\chapter*{Introduction}
\kaitodo[inline]{read chapter/section}
% ##################################################################################################################

\begin{center} \includegraphics[width=0.3\textwidth, angle=0]{frontmatter/figures/MATSimBook} \end{center}

% ##################################################################################################################
%The intense \gls{matsim} development and research during the last few years has generated an extensive body of knowledge with numerous studies, dissertations and projects in research and practice. It is time to put together all the floating pieces and contextualize them in a consistent and coherent manner, as illustrated by the figure above. 
%
%There have been many authors and three editors contributing to this book. Authors come from the \gls{matsim} team but also from the wider community. Chapters have been written by the respective software contributor or researcher whenever possible, which ensures capturing the most complete and detailed knowledge.
%
The book is intended to give new \gls{matsim} users a quick start in running \gls{matsim}. It also provides more experienced \gls{matsim} users and \gls{matsim} developers with 
%details 
information on how to extend \gls{matsim} by plugging in available modules (\eg 
% I changed \ie to \eg since there are extensions beyond the contributions. kai, jan'16
the \glspl{contribution}),  or by programming against the \gls{matsim} \gls{api} to implement their own \gls{matsim} \glspl{extension}. Another of this book's 
%most important 
goals is to contextualize the methods used in \gls{matsim} in a broader theoretical background. By compiling our conceptual insights on \gls{matsim} gained over the years, the book also contributes to methodological discussions on joint \gls{microsimulation} of travel demand and traffic flow, a relatively new field, or---more generally---spatial demand and its congestion generation.

The book is divided into four parts, focused on \emph{using} (Part~\ref{part:using-matsim}), \emph{extending} (Part~\ref{part:extending-matsim}), and \emph{understanding} (Part~\ref{part:understanding-matsim}) \gls{matsim}, 
%as well as 
while simultaneously providing practical, technical, and methodological information. The last part of the book (Part~\ref{part:scenarios}) then presents an impressive array of  \gls{matsim} scenarios that have now been created around the world. %\ah{beschreiben, wie man nach und nach inhaltlich tiefer geht. Verwobenheit des Buches ansprechen.}

\clearpage
% -----------------
\begin{wrapfigure}[6]{r}{0.17\textwidth}
\vspace{-5pt}
  \begin{center}
    \includegraphics[width=0.15\textwidth]{images/DSCF2906.jpg}
  \end{center}
\end{wrapfigure}
\textbf{Part~\ref{part:using-matsim}: Using \acrshort{matsim}}\\
This part enables users to run \gls{matsim} with only the \gls{configfile}, a population and a network. They are given general information to assess whether \gls{matsim} is a suitable tool and method for their specific research question.

Chapter~\ref{ch:introducing} introduces the \gls{matsim} basics, including its underlying co-evolutionary principle and its traffic flow model. 
Chapter~\ref{ch:lgstarted} shows the \gls{matsim} novice how to set up and run a basic \gls{matsim} \gls{scenario}. 
Scoring is central to \gls{matsim}; a full chapter, Chapter~\ref{ch:scoring}, scrutinizes scoring. 
Chapter~\ref{ch:configuring} lists the \gls{configfile}'s options available for basic scenarios containing \gls{configfile}, a population and a network.  


% -----------------
\begin{wrapfigure}[6]{r}{0.17\textwidth}
\vspace{-10pt}
  \begin{center}
    \includegraphics[width=0.15\textwidth]{images/DSCF5871.jpg}
  \end{center}
\end{wrapfigure}
\textbf{Part~\ref{part:extending-matsim}: Extending \acrshort{matsim}}\\
This part
%, divided into 13 sub-parts, % finde ich gefährlich: bleibt es bei 13?  kai, jan'16
presents 
%significant 
technical information on how to extend the base functionality of \gls{matsim} by additional input data beyond \gls{configfile}, population and network, as well as by programming against the \gls{api}. 

Chapter~\ref{ch:modules} introduces \gls{matsim}'s modular architecture. It als explains how to use the available \glspl{module} introduced in Chapters~\ref{ch:first-after-intro-chapter} through \ref{sec:matsim4urbansim}. 
%This is followed by 
Chapter~\ref{ch:discontinued} describes modules that were important in the past but whose development was discontinued.
Chapter~\ref{ch:developmentprocess} briefly describes \gls{matsim} organization, \ie its development process, code structure, the team and the community, and summarizes their development tools. 
Chapter~\ref{ch:extensionpoints} goes one step further and explains to readers how to write their own \gls{matsim} \glspl{extension}, and how to then contribute them to \gls{matsim}, including details about points where \gls{matsim} can be extended; it also digs a bit deeper and provides details about the very central \gls{matsim} concept of \glspl{event}. Explanation about how to 
%add a \lstinline|Listener| and 
%this is really not general enough. kai, jan'16
inject alternative or additional modules and 
%write a customized \lstinline|Controler| 
% Controler is no longer extended/customized. kai, jan'16
how in general to write \gls{matsim} scripts in \gls{java} is also found here.

% -----------------
\begin{wrapfigure}[7]{r}{0.17\textwidth}
\vspace{-10pt}
  \begin{center}
    \includegraphics[width=0.15\textwidth]{images/DSCF5900.jpg}
  \end{center}
\end{wrapfigure}
\textbf{Part~\ref{part:understanding-matsim}: Understanding \acrshort{matsim}}\\
%\ah{Gunnar fragen, ob er hier eine gehaltvollere Zusammenfassung/Vorschau machen möchte}
%\gunnar{Von mir aus ist das hinreichend, es sei denn, jemand m\"ochte, dass ich den kommenden Abschnitt irgendwie ausbaue.}
%\ah{ok, then let us keep it like that}
%
This part presents theoretical aspects underlying the previous two parts. For example, the \gls{matsim} \gls{score} is no longer simply denoted by $S$ without interpretation, but is here contextualized within the discrete choice framework (Chapter~\ref{ch:discretechoice}) and becomes \gls{utility}, commonly denoted by $U$. 
The first chapter, Chapter~\ref{ch:history} starts with a summary of \gls{matsim}'s history, written by Kai Nagel and Kay W.\ Axhausen, \gls{matsim}'s ``fathers''. 
Chapter~\ref{ch:abta} then elaborates on agent-based traffic assignment and qualitatively contextualizes \gls{matsim} within classical concepts. Here, the focus is on development from static to dynamic \gls{trafficassignment} and, finally, agent-based \gls{trafficassignment}.  
Chapter~\ref{ch:montecarlo} quantitatively contextualizes \gls{matsim} within classical concepts by presenting it as a fundamentally stochastic tool, based on random distributions and understandable as a Monte Carlo engine.
Chapter~\ref{ch:kinematicwaves} analyzes \gls{matsim}'s traffic flow model in relation to kinematic waves, while Chapter~\ref{ch:economicEval} provides an economic view on \gls{matsim}. 

% -----------------
\begin{wrapfigure}[7]{r}{0.17\textwidth}
\vspace{-10pt}
  \begin{center}
    \includegraphics[width=0.15\textwidth]{images/DSC00233.jpg}
  \end{center}
\end{wrapfigure}
\textbf{Part~\ref{part:scenarios}: Scenarios}\\
At this point, when readers have a complete picture of \gls{matsim} and are ready to set up their own real-world \gls{matsim} \gls{scenario}, Chapters~\ref{ch:scenarios} through~\ref{ch:yokohama} show them the numerous and highly varied scenarios that have been implemented around the world.

% -----------------
\vskip 1cm
The book concludes with a discussion of promising research avenues (Chapter~\ref{ch:researchavenues}).

% ##################################################################################################################
\section*{Related Material}

The book concentrates on the more stable aspects of \gls{matsim} application and development.  In the future, revisions of Chapters~\ref{ch:introducing} to~\ref{ch:modules} will be presented once a year.  Additional material is referenced from \url{http://matsim.org}, for example under \url{http://matsim.org/docs}, \url{http://matsim.org/javadoc}, \url{http://matsim.org/extensions}, \url{http://matsim.org/faq}, or \url{http://matsim.org/issuetracker}.


%% , while the practical issues, such as specific configurations or code details---usually being part of manuals---can primarily be found in the \gls{matsim} user guide (shipped with the \gls{matsim} releases) and provided on the \gls{matsim} web page \citep[][]{MATSim_Userguide_2015}. \todokai{rm usrguide}

% ##################################################################################################################

% Local Variables:
% mode: latex
% mode: reftex
% mode: visual-line
% TeX-master: "../main"
% comment-padding: 1
% fill-column: 9999
% End: 
