\chapter*{Editors' Foreword}

\editdone{This text has undergone the professional edit.}

% ##################################################################################################################
Developing complex software for over a decade with a heterogeneous group of engineers and scientists (each with widely different skill levels and expertise across multiple locations around the world) requires dedication and mechanisms unusual for a university environment. 

This book is one of these mechanisms. It allows us, collectively, to take stock and present a coherent, state-of-the-system: for us and anyone interested in this approach. It highlights the basics for the student, who wants to undertake a small first research project as part of her degree, provides description of the main functionalities, in detail, for the engineer setting up \gls{matsim} to conduct a policy analysis and, finally, fits the approach into the theoretical background of complex systems in computer science and physics. 

The choice of the e-book format is an advantage, as it allows us to keep the book up-to-date with future chapters, revisions and, if necessary, errata. Equally importantly it allows you, the readers, to select those sections relevant to your needs. 

The book comes at an important time for the system; for most of the first decade, its use was limited to the original developers and users in Berlin and Zürich. It is now much more widely consulted around the world, as we document in the chapter summarizing contributions on \glspl{scenario} up to the present time. 

\Gls{scenario}: this term will occur again and again. In \gls{matsim} context, it is defined as the combination of specific agent populations, their initial \glspl{plan} and \glspl{activitylocation} (home, work, education), the network and facilities where, and on which, they compete in time-space for their slots and modules, \ie behavioral dimensions, which they can adjust during their search for equilibrium. Within these \glspl{scenario}, the user can experiment and explore with behavioral \gls{utilityfunction} parameters, sampling rate of the population between 1\,\% and 100\,\%, and algorithm parameters, \ie share of the sample engaged in \gls{replanning} in any \gls{iteration} and behavioral dimension or exact settings necessary to avoid gridlock due to artifacts of the traffic flow simulation. The creation of a \gls{scenario} is a substantial effort and the \gls{framework} makes a number of tools available to accelerate it: population synthesizers, network editors, network converters between popular formats and the \gls{matsim} representation (\eg \gls{osm}, \gls{gtfs}), semi-automatic network matching to join information, among others.

A large group of colleagues has been involved, and many of them are contributors to this book; this is a list of those involved---both in the past and currently---in Berlin, Singapore and Zürich.  
%
\begin{multicols}{3}
Amit Agarwal \\
Milos Balac  \\
Dr. Michael Balmer\mbox{*} \\
Henrik Becker \\
Joschka Bischoff \\
Patrick Bösch \\
Dr. David Charypar \\
Dr. Nurhan Cetin  \\
Artem Chakirov \\
Dr. Yu Chen \\
Dr. Francesco Ciari \\
Dr. Michal Maciejewski \\
Dr. Christoph Dobler \\
Thibaut Dubernet\mbox{*} \\
Dr. Alexander Erath \\
Dr. Matthias Feil \\
Prof. Dr. Gunnar Flötteröd \\
Pieter J. Fourie\mbox{*} \\
Dr. Christian Gloor \\
Dr. Dominik Grether \\
Dr. Jeremy K. Hackney \\
Dr. Johannes Illenberger \\
Prof. Dr. Johan W. Joubert\mbox{*} \\
Ihab Kaddoura \\
Dr. Benjamin Kickhöfer \\
Dr. Gregor Lämmel\mbox{*} \\
Nicolas Lefebvre \\
Dr. Fabrice Marchal \\
Alejandro Marmolejo \\
Dr. Konrad Meister \\
Dr. Manuel Moyo Oliveros \\
Kirill Müller \\
Dr. Andreas Neumann \\
Dr. Thomas Nicolai \\
Dr. Benjamin Kickhöfer \\
Sergio A. Ordóñez Medina \\
Dr. Bryan Raney \\
Dr. Marcel Rieser\mbox{*} \\
Dr. Nadine Rieser-Schüssler \\
Daniel Röder \\
Mohit Shah \\
Lijun Sun \\
Alexander Stahel \\
Prof. Dr. David Strippgen \\
Dr. Basil Vitins \\
Michael Van Eggermond \\
Dr. Rashid Waraich\mbox{*} \\
Dominik Ziemke \\
Michael Zilske\mbox{*} \\
\end{multicols}
% 
We hope to acknowledge the contributions of more colleagues from other groups, in future versions of this book and in the software.   

Special thanks to members of the "committee" (marked with an \mbox{*}), which makes the final decisions on the \gls{matsim} core, the allocation between core and \glspl{contribution} and the software engineering of the \gls{framework}.

This committee meeting is a further coordination mechanism; without regular face-to-face meetings, one cannot maintain the trust, communication and understanding necessary. So far, an annual user meeting to report on progress, a week-long developer meeting to jointly advance the system and a conceptual meeting to set the medium and longer term agenda have worked as our approach. The user meeting is open to everyone interested, but we co-opt developers and users into the two other meetings to ensure that they are contributing effectively and efficiently to their tasks. 

This effort was funded and supported over the years by numerous agencies. Several particularly important sources are: \gls{eth} Zürich and TU Berlin, through the base funding of the two current core groups, the Deutsche Forschungsgemeinschaft (DFG), the \gls{snf}, the Swiss Bundesamt für Strassen (ASTRA), Volkswagen AG and the Singaporean National Research Foundation (NRF), through their repeated grants and projects supporting different dissertations through the years. This is gratefully acknowledged by all researchers. 

A special thank goes to the Karen Ettlin, for the very precise and productive copy editing. \ah{we extrapolate here from previous collaborations ;)}. The copy editing is funded by the \gls{snf}-grant number ... and the \ah{TUB fund}

We hope that this book is able to engage the interest of more researchers and engineers to be involved in this joint effort to enable us to provide jointly this \gls{framework}, which has to be continuously adapted to our policy needs, so that it stays at the forefront of travel behavior modeling.

The editors

Andreas Horni, 	Kai Nagel,	Kay W. Axhausen

\ah{Reihenfolge diskutieren.}

Zürich, April 2015

% ##################################################################################################################
