\chapter*{Acknowledgments}
\kaitodo[inline]{read chapter/section}
% ##################################################################################################################
A project this dispersed and as long as the MATSim project draws on many sources for its support. We hope that we have not forgotten any institution here. 
We are grateful to all of them, which made this open-source effort possible and we hope that they will continue to do so in the spirit of intellectual discovery and sharing. 

In every case we have to thank our home institutions for providing the basic intellectual and computing infrastructure for our work. 
ETH Zürich was home to Prof.~Nagel and his group when he started to project and continues to be the basis for Prof.~Axhausen and his team. 
TU~Berlin became Prof.~Nagel's new platform after his move. 
Both institutions provided support through base funding of staff, servers and data access, which allow us to provide ongoing support to the overall project. 

The following projects and sponsors funded particular persons and implementations: 

% VSP:
TU Berlin (Kai Nagel, Amit Agarwal, Joschka Bischoff, Dominik Grether, Johannes Illenberger, Benjamin Kickhöfer, Gregor Lämmel, Andreas Neumann, Theresa Thunig, Dominik Ziemke, Michael Zilske) undertook this work in the framework of the following projects: 
%``Detailed evaluation of transport policies using microsimulation'' (DFG NA 682/3-1.); 
%``Optimization and network wide analysis of traffic signal control'' DFG NA 682/7-1);
%``Methods for Modeling and Large-scale Simulation of Multi-destination Pedestrian Crowds'' (DFG NA 682/5-1); 
%``Numerisches Last-mile Tsunami Frühwarn- und Evakuierungsinformationssystem'' by the German Federal Ministry of Education and Research (BMBF) grant 03G0666E and 
%``GIS-basierte Risiko-Analyse-, Informations-, and Planungs-System für die Evakuierung von Gebieten'' by the German Federal Ministry of Education and Research (BMBF) grant 13N1138[1-3].

``COOPERS: Co-operative Networks for Intelligent Road Safety (EU 026814);
``Modelling and simulation approaches for livable cities'' (Volvo Research and Education Foundation SP-2004-49);
``Travel impacts of social networks ans networking  tools Social networks'' (Volkswagen Stiftung I/82 714);
``Numerical Last-mile Tsunami Early Warning and Evacuation Information System'' (BMBF, 03FG0666E);
``Adaptive Traffic Control'' (BMBF 03NAPAI4);
``State Estimation for traffic simulations as coarse grained systems'' (DFG NA 682/1-1);
``Detailed assessment of transport measures using micro-simulation'' (DFG NA 682/3-1);
``Simulation of Multidestination Pedestrian Crowds'' (DFG NA 682/5-1);
``SustainCity: Micro-simulation for the prospective of sustainable cities in Europe (EU 7th Framework 244557);
``Contributions of transport towards the realization of a 2000\,W city'' (DFG NA 682/6-1);
``GRIPS: GIS based risk analysis, information, and planning system for the evaluation of ares'' (BMBF 13N11382);
``Simulation-based system for the sustainable management of electrically powered taxi fleets'' (Einstein Stiftung Berlin A-2012-132);
``Optimization and network wide analysis of traffic signal control'' (DFG NA 682/7-1);
``MAXess: Maesuring accessibiles for policy evaluation'' (BMBF (ERAfrica) 01DG14008); %\ah{siehe MAXess unten?!}
``An agent-based evolutionary approach for the user-oriented optimization of complex public transit systems'' (DFG NA682/11-1);

% IVT:
ETH Zürich (Kay Axhausen, Milos Balac, David Charypar, Francesco Ciari, Christoph Dobler, Thibout Dubernet, Andreas Horni, Nadine Rieser, Rashid Waraich) could also draw on the following grants: 
``A generalized approach to population synthesis'' (SNF 205121\_138270 25); 
``Agent-based modelling of retailers and their reactions to road pricing'' (ETH TH-19042); 
``Agent-based simulation for location-based services'' (KTI 8443.1 ESPP-ES); 
``An investigation of strategies leading to a 2000\,W City using a bottom-up model of urban energy flows'' (SNF 105218-122632~1); 
``Assessment of the impacts of the Westumfahrung Zürich (Kanton Zürich)''; 
``Autonomous Cars---The next revolution in mobility'' (SNF 200021\_159234 43); 
``Choice models for transport modelling: Accounting for similarities between alternatives in large scale choice sets'' (SNF 205120-121889 14); 
``Deriving and assessing strategies for limiting the spread of airborne diseases using a social contact model: The case of influenza'' (SNF); 
``Destination Choice Modeling for Discretionary Activities: Fundamentals of Choice Set Formation and Impacts of Spatial Competition'' (SNF 205121\_132086 20); 
``Dynamic Traffic Self-organization in China: Network Spatial-temporal Methodology and MATSim Simulation'' (SNF IZ69Z0\_131139 17); 
``Integrated modelling and analysis of energy and transport systems'' (ETH TH-22 07-03); 
``Large-scale multi-agent simulation of travel behaviour and traffic flow'' (ETH TH-7959); 
``Large-scale stochastic optimization for agent-based traffic simulations'' (ETH TH-18951); 
``MAXess: Measuring accessibility in policy evaluation'' (ERAfrica IZEAZ0\_154310 37); 
``Models without (personal) data?'' (SNF 200021\_144134 29); 
``Optimising public transport: Making smart cards more useful'' (SNF IZKSZ2\_162185 44); 
``Post Car World'' (SNF CRSII1\_147687 21); 
``SCCER Energy and Mobility'' (KTI 33290); 
``Sharing is Saving: how collaborative mobility can reduce the impact of energy consumption for transportation'' (NFP 407140\_153807 41); 
``Simulation evacuation scenarios and Schwingerfest: Evacuation study'' (BABS); 
``SurPrice - Sustainable mobility through Road User Charging'' (ERA.net); 
``SustainCity: Micro-simulation for the prospective of sustainable cities in Europe'' (EU 7th Framework 244557); 
``THELMA---Technology-centered Electric Mobility Assessment'' (CCEM); 
``TOPDAD---Tool-supported policy-development for regional adaptation'' (EU 7th Framework 308620);
``Travel behaviour in a dynamic spatial and social context: Modelling the Interdependence of Social Network Interactions and spatial choices'' (SNF 105212-112482 10) and 
``Travel impacts of social networks and networking tools'' (VW~Stiftung).

% FCL:
The National Research Foundation of Singapore together with ETH Zurich supported the work of Alexander Erath, Pieter Fourie, Sergio Ordonez Medina, Artem Chakirov and Michael Van Eggermond as part of Future Cities Laboratory.

The co-operation which funded Lun Zhang’s work (Tongji University) was based on two grants (EG01-032010, NIP02-092010) of the Sino-Swiss Cooperation Project Program funded by ETH~Zürich, Switzerland.

The work reported by senozon AG (Michael Balmer, Marcel Rieser, Daniel Röder, Christoph Dobler and Andreas Neumann) is based on projects undertaken since its was set up in~2010, especially noteworthy are the following clients: Public transport authority of Berlin (BVG - Berliner Verkehrsbetriebe), Federal Statistical Office (BFS Bundesamt für Statistik), Openstreetmap.org and their Contributors, Peter Vovsha, Parsons Brinckerhoff, NY, Prof.~Ulrich Weidmann, Transport System group (VS) of the Institute for Transport Planning and Systems (IVT)

University of Pretoria (Johan Joubert) was supported by grants of the South African National Treasury and the National Research Foundation grant FA2007051100019. 

At RMIT (Melbourne) Lin Padgham and Dhirendra Singh were supported by the ARC Discovery DP1093290, ARC Linkage LP130100008 and Telematics Trust grants. They would like to thank Agent Oriented Software for the use of the JACK BDI platform.

The work of Seungjae Lee and Atizaz Ali at the University of Seoul was supported by a grant (11~Higy-tech Urban G06) from High-tech Urban Development Program funded by Ministry of Land, Infrastructure and Transport of Korean government.

At the National Institute for Environmental Studies the research of Daisuke Murakami was supported by the Environment Research and Technology Development Fund (S-10) of Japan's Ministry of the Environment.

The work on the Trondheim scenario by Stefan Flügel, Julia Kern and Frederik Bockemühl was supported by the Research Council of Norway with ``Future Sustainable Transport for Industry and Trade in Norway'' (208420/F40).

The research presented by the University of Poznan was partially supported by the grants PBS1/A6/11/2012 and ERA-NET-TRANSPORT-III/2/2014 from the National Centre for Research and Development (Poland).

At the Universite de Liege (Mario Cools, Jacques Teller, Ismail Saadi) the work was supported by the ARC grant for Concerted Research Actions, financed by the Wallonia-Brussels Federation on ``Landuse change and future flood risk: influence of micro-scale spatial patterns (FLOODLAND)''.

Oleg Saprykin, Olga Saprykina and Tatyana Mikheeva were supported by the Ministry of Education and Science of the Russian Federation at Samara State Aerospace University. 

Chengxiang (Tony) Zhuge (Zhejiang University, Beijing Jiaotong University) and Chunfu Shao’s project ``Evolution Mechanism, Regulation and Control Methods of Urban Transportation Supply and Demand Structure'' was funded by the National Natural Science Foundation of China (51338008).

Sashikanth Gurram and Abdul R. Pinjari and Amy L. Stuart work at the University of South Florida benefited a grant by the National Science Foundation (0846342) on ``Tampa, Florida: High Resolution Simulation of Urban Travel and Network Performance for Estimating Mobile Source Emissions''.

The work of Maxime Lenormand (UIB) and Miguel Picornell (NOMMON) was in the context of a EU~7th~Framework grant (EUNOIA,~318367).
 
The work for Toronto (Adam Weis, Khandker Nurul Habib, Peter Kucirek, Eric Miller, CF Shao) was funded in part by an Natural Sciences and Engineering Research Council (Canada) Discovery Grant and by the sponsors of the University of Toronto Travel Modelling Group: Metrolinx, the Ontario Ministry of Transportation, the Cities of Toronto, Hamilton, Mississauga and Brampton, and the Regional Municipalities of Durham, Halton, Peel and York.

The work at Shinshu University (Rolando Armas) is support by Ecudoran National Secretariat of Higher Education, Science, Technology and Innovation. 

National University of Ireland Maynooth and Dublin (Gavin McArdle, Aonghus Lawlor, Eoghan Furey) were supported by the Science Foundation Ireland by a Strategic Research Cluster grant (07/SRC/I1168) under the National Development Plan.

The work at the University of Melbourne (Nicole Roland) was based on an Australian Research Council grant on ``Integrating Mobility on Demand'' (Linkage Project LP120200130).

Daisuke Fukuda's work at Tokyo Tech was supported by a Grant-in-Aid for Scientific Research from the Japan Society for the Promotion of Science (B) number 25289160 and by the Committee on Advanced Road Technology (CART), Ministry of Land, Infrastructure, Transport, and Tourism, Japan.

The results of Erasmus University Rotterdam (Paul Bouman, Milan Lovric) were made possible by a grant of the Netherlands Organisation for Scientific Research (NWO) funding the Complexity in Public Transport project (ComPuTr). 

The research leading to the results reported by UCL (Camilo Ruiz, Joan Serras, Mike Batty, Melanie Bosredon, Vassilis Zachariadis) has received funding from the UK's Engineering and Physical Sciences Research Council under grant agreement number EP/G057737/1 (SCALE project; 2009–2013), the European Union Seventh Framework Programme FP7/2007–2013 under grant agreement number 318367 (EUNOIA project) and the European Research Council under grant agreement number 249393 (MECHANICITY project; 2010–2015).

The past and ongoing work at KTH Stockholm (Gunnar Flötteröd) was based on the
following grants:
``IHOP2: Flexible coupling of disaggregate travel demand models and network simulation packages'' (Trafikverket TRV 2015/2950); 
``SMART-PT: Smart public Transport'' (Eranet Transport III---Future traveling, VINNOVA 2014-03976) and
``PETRA: Personal Transport Advisor: an integrated platform of mobility patterns for Smart Cities to enable demand-adaptive transportation system'' (EU 7th Framework Program 609042). 
He is supported by the KTH strategic research program in transport TRENoP (Transport research with novel perspectives).

The data sources and support which the authors obtained are too numerous to listen here. Please set the original papers, theses and reports as cited in the various chapters. 

\subsection*{Sponsors: Names and Abbreviations}

\begin{tabular}[]{p{0.1\linewidth} p{0.6\linewidth} p{0.12\linewidth}} % \ah{reduce to <1.0 -> page overflow?!}
\hline
\textbf{ARC} & Australian Research Council  & Australia\\
\hline
\textbf{BABS} & Bundesamt für Bevölkerungsschutz/Federal Office for Civil Protection & Switzerland\\
\hline
\textbf{BMBF} & Bundesministerium für Bildung und Forschung/Federal Ministry of Education and Research & Germany\\
\hline
\textbf{DFG} & Deutsche Forschungsgemeinschaft/German Research Foundation & Germany\\
\hline
\textbf{ERA} & European Research Action & Country consortia\\
\hline
\textbf{KTI} & Kommission für Technologie und Innovation/Commission for Technology and Innovation  & Switzerland\\
\hline
\textbf{NCRD} & National Centre for Research and Development & Poland\\
\hline
\textbf{NFP} & Nationales Forschungsprogramm/National Research Program & Switzerland\\
\hline
\textbf{NRF} & National Research Foundation & Singapore\\
\hline
\textbf{NSF} & National Science Foundation & USA\\
\hline
\textbf{SNF} & Schweizerischer Nationalfonds/Swiss National Research Foundation & Switzerland\\
\hline
\textbf{Trafikverket} & Swedish Transport Administration & Sweden\\
\hline
\end{tabular}

% ##################################################################################################################
