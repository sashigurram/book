\chapter*{Preface}
% ##################################################################################################################

\begin{center} \includegraphics[width=0.25\textwidth, angle=0]{figures/MATSimBook.png} \end{center}

% ##################################################################################################################
\ah{adjust to editors' foreword}

The intense MATSim development and research during the past 10 years has generated a large body of knowledge with a large number of studies, dissertations, and projects in research and practice. Time to put together and contextualize all the pieces in a consistent and coherent book. As an open-source project it relies and benefits from the contributions of its users and developers around the world. The book is also an invitation to new users to use and develop their own ideas, scenarios and modules (contributions) by integrating the many existing elements into one easily accessible place.

There have been many authors and 3 editors contributing to this book. Chapters have been written by the respective software contributor or researcher whenever possible, which ensures capturing the most complete and detailed knowledge. 

The book is intended to give the new MATSim user a quick start in running MATSim. It shall furthermore provide the experienced MATSim user and the MATSim developer with details on how to extend MATSim by plugging in the available modules and by programming against the MATSim API. Last, but not least, the book wants to contribute to the methodological discussion on the relatively new field of the joint microsimulation of travel demand and traffic flow, or more generally of spatial demand and its congestion generation by compiling our conceptual insights on MATSim gained over the years.

To match its aims, the book is divided into the three parts focused on \emph{using, extending} and \emph{understanding} MATSim.

\textbf{Part I: using MATSim} ... \\
Chapter~\ref{ch:introducing} introduces the MATSim basics including its traffic flow model and the underlying co-evolutionary principle. Chapter~\ref{ch:lgstarted} takes the MATSim novice and shows him or her how to set up and run a basic MATSim scenario. Scoring is central to MATSim; a full chapter, Chapter~\ref{ch:scoring}, is thus assigned to scoring. At the point when the readers are ready to set up their own real-world MATSim scenario, Chapter~\ref{ch:scenarios} shows them the scenarios that have been implemented at other places worldwide. 

% -----------------
\textbf{Part II: extending MATSim} ... \\
Chapter~\ref{ch:modules} introduces MATSim modules and explains how to use the available modules introduced in Chapters~\ref{ch:pt} through \ref{ch:businessanalytics}. Chapter~\ref{ch:developmentprocess} shortly describes how new functionality usually enters MATSim. Chapter~\ref{ch:extensionpoints} goes one step further and explains how to write your own MATSim module and how to contribute it to MATSim. It details at which points MATSim can be extended, and it digs a little deeper into MATSim and provides details about the very central MATSim concept of "events". How to add listeners and write a customized controller is also explained here.

% -----------------
\textbf{Part III: understanding MATSim} ... \\
Chapter~\ref{ch:abta} elaborates on agent-based traffic assignment, and Chapter~\ref{ch:montecarlo} continues on that by presenting MATSim as a fundamentally stochastic tool based on random distributions and suitably describable as a Monte Carlo engine. Chapter~\ref{ch:discretechoice} investigates MATSim within the framework of discrete choice modeling, and Chapter~\ref{ch:economicEval} provides an economic view on MATSim. Chapter~\ref{ch:kinematicwaves} analyzes MATSim's traffic flow model in relation to kinematic waves.

\ah{Verwobenheit ansprechen. Pointers nach hinten. Ist jetzt ein essentieller Teil. Schrittweiser Wissensaufbau.}

The book is concluded by a discussion of promising research avenues (Chapter~\ref{ch:researchavenues}).

% ##################################################################################################################
\section*{Related Material}
The book is focused on the more stable aspects of MATSim application and development, while the practical issues, such as specific configurations or code details---usually being part of manuals---can be found in the MATSim user guide (shipped with the MATSim releases) and provided on the MATSim web page \citep[][]{MATSim_Userguide_2014}.

% ##################################################################################################################
