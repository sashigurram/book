\chapter*{Preface}
% ##################################################################################################################

\begin{center} \includegraphics[width=0.25\textwidth, angle=0]{figures/MATSimBook.png} \end{center}

% ##################################################################################################################
MATSim has now been intensely developed for more than 10 years. This has generated a large body of research covering a vast amount of studies, dissertations, and projects in research and practice. High time to put together all the pieces in a summary book.

The book gives the new MATSim user a quick start in running MATSim. It furthermore provides the experienced MATSim user with details on how to extend MATSim by the available modules and by programming against the MATSim API. Last, but not least, the book wants to contribute to the methodological research on the relatively novel field of microsimulation by compiling our insights on MATSim gained over the years. 


Editors and Autors ...: As much as possible and required by the targeted level of detail, functionality is described by the respective software contributor usually having the most detailed knowledge of it. 

Aiming for the purpose defined above, the book is divided into three parts: \emph{I. using MATSim}, \emph{II. extending MATSim}, and \emph{III. understanding MATSim}.

\emph{Part I: using MATSim} ... 
Chapter~\ref{ch:introducing}: description of the co-evolutionary process and an analysis of the necessary processing steps and data requirements when performing a MATSim simulation study. 

Chapter~\ref{ch:lgstarted}

As a core chapter of this book, Chapter \ref{ch:scoring} treats MATSim scoring, probably the most central element in MATSim. 

Finally, existing MATSim scenarios and projects (Chapter~\ref{ch:scenariosprojects}) are presented 

% -----------------
\emph{Part II: extending MATSim} ...
Chapter~\ref{ch:extensionpoints}

and the MATSim modules are listed in Chapter \ref{ch:modules}.

Modules are presented in Chapters \ref{ch:pt} through \ref{ch:coreextensions}.

Chapter~\ref{ch:developmentprocess} 

% -----------------
\emph{Part III}
Chapter~\ref{ch:basicprocedure} gives the mathematical context of the utility-based microsimulation approach

system characteristics and system specification are treated in Chapter~\ref{ch:systemspec}, which include evaluation of the interplay of commonly available data, model resolution, and model sensitivity, where an important element is the assessment of the guidelines provided by the methodological literature. Clearly, these issues are also relevant for single studies and thus they are also touched in part I. Other important analyses focus on microsimulation variability and investigate how the microscopic interactions aggregate on the macroscopic level, such as the macroscopic fundamental diagram or the volume-delay functions.

Chapter~\ref{ch:abta}

Chapter~\ref{ch:montecarlo}

Chapter~\ref{ch:discretechoice}


provides the economic interpretation of MATSim (Chapter~\ref{ch:eco}) 

Chapter~\ref{ch:kinematicwaves}

and localizes it in in the spectrum of traffic analysis tools (Chapter~\ref{ch:relation}). 



Finally, promising research avenues (Chapter~\ref{ch:researchavenues}) are presented.

\ah{ 

Practical hints and \\
where to find more
}

% ##################################################################################################################
\section*{Related Material}

where to find more:
The book is focused on the more stable aspects of MATSim application and development, while the practical issues, such as specific configurations or code details---usually being part of manuals---can be found in the MATSim user guide \citep[][]{MATSim_Userguide_2014} and on the MATSim web page. Clearly, the distinction cannot be strict.

% ##################################################################################################################
\section*{Typographic Conventions}
% ##################################################################################################################

\ah{
``Preface:
Written by the author, the Preface often tells how the book came into being, and is often signed with the name, place and date, although this is not always the case.

Acknowledgments:
The author expresses their gratitude for help in the creation of the book.''
}