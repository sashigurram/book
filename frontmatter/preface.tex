\chapter*{Preface}
% ##################################################################################################################

\begin{center} \includegraphics[width=0.25\textwidth, angle=0]{figures/MATSimBook.png} \end{center}

% ##################################################################################################################
``Preface:
Written by the author, the Preface often tells how the book came into being, and is often signed with the name, place and date, although this is not always the case.

Acknowledgments:
The author expresses their gratitude for help in the creation of the book.''

% ##################################################################################################################
\section*{Dear Reader}

% ##################################################################################################################
\section*{Purpose}
This book is written to fulfill the following three-fold purpose. First, it is an introductory text for new MATSim users, giving a fast overview and a central entry port into MATSim. The relevant literature available for each specific topic is provided in a condensed chapter. Second, due to its comprehensiveness it furthermore serves as a reference for MATSim developers. Relevant material and experiences gained in development are reported, such that a developer planning to implement related or similar functionality can get started quickly. Third, the theorist is served with a starting point for the methodological discussion of microsimulations clearly required for this new method which is not yet theoretically understood very well.

% ##################################################################################################################
\section*{Method}
The book merges and condenses the extensive but fragmented and distributed body of research created by intense MATSim development for over a decade including numerous studies, dissertations, projects in research and practice, seminars, but also developers' experiences not written down as yet. Furthermore, the findings are generalized toward similar frameworks by adopting an inductive research approach. For this meta-analysis supplementary methodological experiments are performed including explicit and quantitative comparison with ALBATROSS \citep[][]{ArentzeTimmermans_TRR_2007} and VISUM \citep[][]{VISUM_Webpage_2013} as an example for a rule-based system and an aggregate model respectively. Thereby, an example analysis with many connection points for comparisons with other frameworks is provided, which will support the methodological transport microsimulation discussion, required for the establishing of a simulation practice with authoritative guidelines and the identification of new microsimulation research avenues. 

The book is focused on the more theoretical aspects of MATSim application and development, while the practical issues, such as specific configurations or code details---usually being part of manuals---can be found in the MATSim user guide \citep[][]{MATSim_Userguide_2014}. Clearly, the distinction cannot be strict.

As much as possible and required by the targeted level of detail, functionality is described by the respective software contributor usually having the most detailed knowledge of it. 

% ##################################################################################################################
\section*{Reader}
% -----------------
Aiming for the purpose defined above, the compendium takes a threefold perspective, namely the application, development and methodological view, and hence addresses the questions in the first place relevant for the microsimulation user or customer, the microsimulation developer, and the microsimulation theorist, respectively. The book is divided accordingly into three parts. Under the customer's (or user's) perspective fall all the issues relevant for specific studies. Examples encompass model calibration, validation, correct handling of microsimulation results' variability and assessing the data requirements given the study purpose and desired model resolution. While the user is responsible to validate the specific model instance calibrated for an individual study, the developer is asked to provide a concise system specification and a thorough validation of the specific microsimulator indeed depended but by no means limited to individual studies. The theorist is more concerned with the research of the microsimulation method as such independent from specific simulators. 



\ah{Zusammenfassung hier! st�rker ausf�hren}

\section*{Content}
% -----------------
\subsection*{Part I} 
Part I adopts the user's perspective. Transport microsimulations are embedded in the context of transport planning and modeling and analyzed as a scientific method (Chapter~\ref{ch:alittletheory}). MATSim's basics (Chapter~\ref{ch:matsimbasics}) include description of the co-evolutionary process and an analysis of the necessary processing steps and data requirements when performing a MATSim simulation study. As a core chapter of this book, Chapter \ref{ch:scoring} treats MATSim scoring, probably the most central element in MATSim. Finally, existing MATSim scenarios and projects (Chapter~\ref{ch:scenariosprojects}) are presented and the MATSim modules are listed in Chapter \ref{ch:modules}. This last chapter also reports about lessons learned in developing the modules and thus builds a bridge to Part II.

% -----------------
\subsection*{Part II}
In part II, of the book taking the MATSim developer's view, system characteristics and system specification are treated in Chapter~\ref{ch:systemspec}, which include evaluation of the interplay of commonly available data, model resolution, and model sensitivity, where an important element is the assessment of the guidelines provided by the methodological literature. Clearly, these issues are also relevant for single studies and thus they are also touched in part I. Other important analyses focus on microsimulation variability and investigate how the microscopic interactions aggregate on the macroscopic level, such as the macroscopic fundamental diagram or the volume-delay functions. Chapter~\ref{ch:cvv} looks at calibration, verification and validation, often mixed up in practice. Part II is concluded by recent extensions concerning the complete MATSim framework and being on the cusp of entering the MATSim core (Chapter~\ref{ch:extensions}).

% -----------------
\subsection*{Part III}
Part III taking a theorist's view, gives the mathematical context of the utility-based microsimulation approach (Chapter~\ref{ch:mathematicalcontext}), provides the economic interpretation of MATSim (Chapter~\ref{ch:eco}) and localizes it in in the spectrum of traffic analysis tools (Chapter~\ref{ch:toolsspectrum}). Finally, promising research avenues (Chapter~\ref{ch:researchavenues}) are presented.

\ah{ 

Practical hints and \\
where to find more
}

% ##################################################################################################################
\section*{Related Material}

% ##################################################################################################################
\section*{Typographic Conventions}
% ##################################################################################################################