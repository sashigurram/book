\chapter*{Preface}
% ##################################################################################################################

\begin{center} \includegraphics[width=0.25\textwidth, angle=0]{figures/MATSimBook.png} \end{center}

% ##################################################################################################################
The intense MATSim development and research during the last years has generated an extensive body of knowledge with a large number of studies, dissertations, and projects in research and practice. Time to put together all the floating pieces and contextualize them in a consistent and coherent manner as illustrated by above figure. 

There have been many authors and three editors contributing to this book. Authors come from the \gls{matsim} core team but also from the extended community. Chapters have been written by the respective software contributor or researcher whenever possible, which ensures capturing the most complete and detailed knowledge.

The book is intended to give the new \gls{matsim} user a quick start in running \gls{matsim}. It shall furthermore provide the experienced MATSim user and the MATSim developer with details on how to extend MATSim by plugging in the available modules, i.e.,\,the \glspl{contribution}, and by programming against the MATSim API to implement their own \gls{matsim} \glspl{extension}. Last, but not least, the book wants to contribute to the methodological discussion on the relatively new field of the joint microsimulation of travel demand and traffic flow, or more generally of spatial demand and its congestion generation, by compiling our conceptual insights on MATSim gained over the years.

To match its aims, the book is divided into the three parts focused on \emph{using, extending} and \emph{understanding} MATSim. \ah{beschreiben, wie man nach und nach inhaltlich tiefer geht.}

\textbf{Part I: using MATSim} ... \\
Chapter~\ref{ch:introducing} introduces the \gls{matsim} basics including its traffic flow model and the underlying co-evolutionary principle. 

Chapter~\ref{ch:lgstarted} takes the MATSim novice and shows him or her how to set up and run a basic \gls{matsim} scenario. 

Scoring is central to MATSim; a full chapter, Chapter~\ref{ch:scoring}, is thus assigned to scoring. 

At the point, when the readers are ready to set up their own real-world MATSim \gls{scenario}, Chapter~\ref{ch:scenarios} shows them the numerous scenarios that have been implemented worldwide. 

% -----------------
\textbf{Part II: extending MATSim} ... \\
Chapter~\ref{ch:modules} introduces \gls{matsim} modules and explains how to use the available modules introduced in Chapters~\ref{ch:pt} through \ref{ch:businessanalytics}. 

Chapter~\ref{ch:developmentprocess} shortly describes the development process of MATSim, i.e.,\,the team and the community and their development tools are summarized. 

Chapter~\ref{ch:extensionpoints} goes one step further and explains how to write your own MATSim extension and how to later contribute it to \gls{matsim}. It details at which points MATSim can be extended, and it digs a little deeper and provides details about the very central \gls{matsim} concept of "events". How to add a \lstinline|Listener| and write a customized \lstinline|Controler| is also explained here.

% -----------------
\textbf{Part III: understanding MATSim} ... \\

\ah{etwas inhaltslos. Gunnars Überlegungen bez. Zusammenhang der Kapitel beschreiben.}

This part provides the underlying theoretical aspects of the previous two parts. For example, the \gls{matsim} score is not just a simple score denoted with $S$ anymore, but, it is here interpreted within the discrete choice framework (Chapter~\ref{ch:discretechoice}) and becomes utility denoted as common with $U$. 

But let us start, from the beginning, with the first Chapter~\ref{ch:history} providing a short summary on \gls{matsim}'s history written by the Kai Nagel, one of \gls{matsim}'s fathers. 

Chapter~\ref{ch:abta} then elaborates on agent-based traffic assignment, qualitatively contextualizes \gls{matsim} within the classical concepts. The development from statical assignment to dynamic assignment and finally agent-based assignment is focused here.  

Chapter~\ref{ch:montecarlo} continues on that by presenting \gls{matsim} as a fundamentally stochastic tool based on random distributions and suitably describable as a Monte Carlo engine. It quantitatively contextualizes MATSim within the classical concepts.

Chapter~\ref{ch:kinematicwaves} analyzes MATSim's traffic flow model in relation to kinematic waves and Chapter~\ref{ch:economicEval} provides an economic view on \gls{matsim}. 

The book is concluded by a discussion of promising research avenues (Chapter~\ref{ch:researchavenues}).

% ##################################################################################################################
\section*{Related Material}
The book is focused on the more stable aspects of MATSim application and development, while the practical issues, such as specific configurations or code details---usually being part of manuals---can be found in the MATSim user guide (shipped with the MATSim releases) and provided on the MATSim web page \citep[][]{MATSim_Userguide_2015}.

% ##################################################################################################################
