\chapter*{Preface}
% ##################################################################################################################

\begin{center} \includegraphics[width=0.25\textwidth, angle=0]{figures/MATSimBook.png} \end{center}

% ##################################################################################################################
The intense \gls{matsim} development and research during the last years has generated an extensive body of knowledge with a large number of studies, dissertations, and projects in research and practice. Time to put together all the floating pieces and contextualize them in a consistent and coherent manner as illustrated by above figure. 

There have been many authors and three editors contributing to this book. Authors come from the \gls{matsim} team but also from the wider community. Chapters have been written by the respective software contributor or researcher whenever possible, which ensures capturing the most complete and detailed knowledge.

The book is intended to give the new \gls{matsim} user a quick start in running \gls{matsim}. It shall furthermore provide the more experienced \gls{matsim} user with details on how to extend \gls{matsim} by plugging in the available modules, i.e.,\,the \glspl{contribution}, and by programming against the \gls{matsim} \gls{api} to implement their own \gls{matsim} \glspl{extension}. An important aim of the book is to contextualize the methods used in \gls{matsim} in a broader theoretical background. Thereby, the book also wants to contribute to the methodological discussion on the relatively new field of the joint \gls{microsimulation} of travel demand and traffic flow, or more generally of spatial demand and its congestion generation, by compiling our conceptual insights on \gls{matsim} gained over the years.

To match its aims, the book is divided into the three parts focused on \emph{using}, \emph{extending} and \emph{understanding} \gls{matsim}, providing practical, technical and methodological information, respectively. %\ah{beschreiben, wie man nach und nach inhaltlich tiefer geht. Verwobenheit des Buches ansprechen.}

\textbf{Part I: using MATSim} ... \\
In this part, users are enabled to run \gls{matsim} and they are given general information to assess if \gls{matsim} is a suitable tool and method for their specific research question.

Chapter~\ref{ch:introducing} introduces the \gls{matsim} basics including its traffic flow model and the underlying co-evolutionary principle. 
Chapter~\ref{ch:lgstarted} takes the \gls{matsim} novice and shows him or her how to set up and run a basic \gls{matsim} \gls{scenario}. 
Scoring is central to \gls{matsim}; a full chapter, Chapter~\ref{ch:scoring}, thus looks closer at scoring. 
At the point, when the readers are ready to set up your own real-world \gls{matsim} \gls{scenario}, Chapter~\ref{ch:scenarios} shows them the numerous scenarios that have been implemented around the world. 

% -----------------
\textbf{Part II: extending MATSim} ... \\
This part comes up with a whole bunch of technical information on how to extend the base functionality of \gls{matsim}.

Chapter~\ref{ch:modules} introduces \gls{matsim} \glspl{module} and explains how to use the available \glspl{module} introduced in Chapters~\ref{ch:destinationchoice} through \ref{ch:businessanalytics}. Chapter~\ref{ch:discontinued} describes modules, that have been important in the past, but whose development is discontinued.
Chapter~\ref{ch:developmentprocess} shortly describes the development process of \gls{matsim}, i.e.,\,the team and the community and their development tools are summarized. 
Chapter~\ref{ch:extensionpoints} goes one step further and explains the readers how to write their own \gls{matsim} \gls{extension} and how to later contribute it to \gls{matsim}. It details at which points \gls{matsim} can be extended, and it digs a little deeper and provides details about the very central \gls{matsim} concept of \glspl{event}. How to add a \lstinline|Listener| and write a customized \lstinline|Controler| is also explained here.

% -----------------
\textbf{Part III: understanding MATSim} ... \\

\ah{Gunnar fragen, ob er hier eine gehaltvollere Zusammenfassung/Vorschau machen möchte}

This part provides the underlying theoretical aspects for the previous two parts. For example, the \gls{matsim} \gls{score} is not just simply denoted by $S$ without interpretation anymore, but, it is here contextualized within the discrete choice framework (Chapter~\ref{ch:discretechoice}) and becomes \gls{utility} commonly denoted by $U$. 
The first chapter, Chapter~\ref{ch:history}, however starts with a summary on \gls{matsim}'s history written by the Kai Nagel, one of \gls{matsim}'s fathers. 
Chapter~\ref{ch:abta} then elaborates on agent-based traffic assignment and qualitatively contextualizes \gls{matsim} within the classical concepts. The development from statical \gls{trafficassignment} to dynamic \gls{trafficassignment} and finally agent-based \gls{trafficassignment} is focused here.  
Chapter~\ref{ch:montecarlo} quantitatively contextualizes \gls{matsim} within the classical concepts by presenting \gls{matsim} as a fundamentally stochastic tool based on random distributions and suitably describable as a Monte Carlo engine.
Chapter~\ref{ch:kinematicwaves} analyzes \gls{matsim}'s traffic flow model in relation to kinematic waves and Chapter~\ref{ch:economicEval} provides an economic view on \gls{matsim}. 

% -----------------
The book is concluded by a discussion of promising research avenues (Chapter~\ref{ch:researchavenues}).

% ##################################################################################################################
\section*{Related Material}
The book concentrates on the more stable aspects of \gls{matsim} application and development, while the practical issues, such as specific configurations or code details---usually being part of manuals---can primarily be found in the \gls{matsim} user guide (shipped with the \gls{matsim} releases) and provided on the \gls{matsim} web page \citep[][]{MATSim_Userguide_2015}.

% ##################################################################################################################
