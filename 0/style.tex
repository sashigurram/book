%%%%%%%%%%%%%%%%%%%%%%%%%%%%%%%%%%%%%%%%%%%%%%%%%%%%%%%%%%%%%%%%%%%%%%
%% Type of document:
%% - Paperformat: letterpaper, a4paper, a5paper, b5paper,
%%   executivepaper, legalpaper
%% - Main font size: 10pt, 11pt, 12pt
%% - Formulae setting: - (centred), fleqn (left-aligned)
%% - Numbering of formulae: - (right-aligned), leqno (left-aligned)
%% - New page after title: titlepage, notitlepage
%% - Number of columns per page: onecolumn, twocolumn
%% - Page style: oneside, twoside
%% - Paper rotation: - (protrait), landscape
%% - Chapter start: openright, openany
%% - Mark overfull boxes: draft, final
\documentclass[a4paper,12pt,fleqn,titlepage,onecolumn,twoside,openany,final]{report}
%%%%%%%%%%%%%%%%%%%%%%%%%%%%%%%%%%%%%%%%%%%%%%%%%%%%%%%%%%%%%%%%%%%%%%
%% borders, margrins and offset
\usepackage[a4paper,left=1.0in,right=1.0in,top=0.5in,bottom=0.5in,includeheadfoot]{geometry}
%%%%%%%%%%%%%%%%%%%%%%%%%%%%%%%%%%%%%%%%%%%%%%%%%%%%%%%%%%%%%%%%%%%%%%
%% providing if-then-else command:
\usepackage{ifthen}
%%%%%%%%%%%%%%%%%%%%%%%%%%%%%%%%%%%%%%%%%%%%%%%%%%%%%%%%%%%%%%%%%%%%%%
\usepackage[english]{babel}%
%%%%%%%%%%%%%%%%%%%%%%%%%%%%%%%%%%%%%%%%%%%%%%%%%%%%%%%%%%%%%%%%%%%%%%
%% Header and footer definition:
\usepackage{fancyhdr}%
\pagestyle{fancy}%
\fancyhf{}%
\fancyhead[R]{\slshape \footnotesize \nouppercase{\myyear}}%
\fancyhead[L]{\slshape \footnotesize \nouppercase{\mytitle}}%
\fancyfoot[C]{\footnotesize \thepage}%
\renewcommand{\headrulewidth}{0.5pt}%
\renewcommand{\footrulewidth}{0pt}%
%%%%%%%%%%%%%%%%%%%%%%%%%%%%%%%%%%%%%%%%%%%%%%%%%%%%%%%%%%%%%%%%%%%%%%
%% paragraph settings:
\setlength{\parindent}{0in}%
\setlength{\parskip}{10pt}
%%%%%%%%%%%%%%%%%%%%%%%%%%%%%%%%%%%%%%%%%%%%%%%%%%%%%%%%%%%%%%%%%%%%%%
%% caption settings:
\usepackage[nooneline,format=hang]{caption}
%%%%%%%%%%%%%%%%%%%%%%%%%%%%%%%%%%%%%%%%%%%%%%%%%%%%%%%%%%%%%%%%%%%%%%
%% Define the depth of numbering parts,chapter,sections and paragraphs:
%%   Numbers representing the depth of sectional units:
%%   -1 = \part    (in book or report document classes)
%%    0 = \chapter (in book or report document classes)
%%    0 = \part    (in article document classes)
%%    1 = \section
%%    2 = \subsection
%%    3 = \subsubsection
%%    4 = \paragraph
%%    5 = \subparagraph
\setcounter{secnumdepth}{3}
%%%%%%%%%%%%%%%%%%%%%%%%%%%%%%%%%%%%%%%%%%%%%%%%%%%%%%%%%%%%%%%%%%%%%%
%% citation style:
\usepackage[round]{natbib}
\bibliographystyle{0/template_ivt-eng}%
%%%%%%%%%%%%%%%%%%%%%%%%%%%%%%%%%%%%%%%%%%%%%%%%%%%%%%%%%%%%%%%%%%%%%%
%% Font:
\usepackage{times}
%%%%%%%%%%%%%%%%%%%%%%%%%%%%%%%%%%%%%%%%%%%%%%%%%%%%%%%%%%%%%%%%%%%%%%
%% To prevent overfull boxes
%%   it is quite nice, but in special cases, you will have too large
%%   speaces between a two words of the same line.
\sloppy
%%%%%%%%%%%%%%%%%%%%%%%%%%%%%%%%%%%%%%%%%%%%%%%%%%%%%%%%%%%%%%%%%%%%%%
%% providing umlauts:
%\usepackage[latin1]{inputenc}
%\usepackage[T1]{fontenc}

%\usepackage[utf8]{inputenc}
\usepackage{ifxetex}

\ifxetex
  \usepackage{fontspec}
\else
  \usepackage[utf8]{inputenc}
\fi

%%%%%%%%%%%%%%%%%%%%%%%%%%%%%%%%%%%%%%%%%%%%%%%%%%%%%%%%%%%%%%%%%%%%%%
%% use hyper-refs for URL's and citations
\usepackage{hyperref}
%% line breaks for URL's
\usepackage{url}
%%%%%%%%%%%%%%%%%%%%%%%%%%%%%%%%%%%%%%%%%%%%%%%%%%%%%%%%%%%%%%%%%%%%%%
%% line spacing
\usepackage{setspace}
\onehalfspacing
%%%%%%%%%%%%%%%%%%%%%%%%%%%%%%%%%%%%%%%%%%%%%%%%%%%%%%%%%%%%%%%%%%%%%%
%% letter spacing
\usepackage{soul}
%%%%%%%%%%%%%%%%%%%%%%%%%%%%%%%%%%%%%%%%%%%%%%%%%%%%%%%%%%%%%%%%%%%%%%
%% no indentation for formulas:
\usepackage[fleqn]{amsmath}
\setlength\mathindent{0pt}
%%%%%%%%%%%%%%%%%%%%%%%%%%%%%%%%%%%%%%%%%%%%%%%%%%%%%%%%%%%%%%%%%%%%%%
%% providing graphics:
\usepackage{graphics}
\usepackage{graphicx}
%%%%%%%%%%%%%%%%%%%%%%%%%%%%%%%%%%%%%%%%%%%%%%%%%%%%%%%%%%%%%%%%%%%%%%
%% sideways figures and tables:
\usepackage{rotating}
%%%%%%%%%%%%%%%%%%%%%%%%%%%%%%%%%%%%%%%%%%%%%%%%%%%%%%%%%%%%%%%%%%%%%%
%% sub-figures:
\usepackage[FIGTOPCAP]{subfigure}
\def\subfigtopskip{0pt}
\def\subfigbottomskip{5pt}
\def\subfigcapskip{0pt}
%%%%%%%%%%%%%%%%%%%%%%%%%%%%%%%%%%%%%%%%%%%%%%%%%%%%%%%%%%%%%%%%%%%%%%
%% figures:
%%   The following are sometimes needed to avoid pushing
%%   the figs to the end of the text.
\def\textfraction{0.0}
\def\topfraction{0.9999}
\def\floatpagefraction{0.9}
%%%%%%%%%%%%%%%%%%%%%%%%%%%%%%%%%%%%%%%%%%%%%%%%%%%%%%%%%%%%%%%%%%%%%%
%% tables:
\usepackage{tabularx}
\usepackage{multirow}
%%%%%%%%%%%%%%%%%%%%%%%%%%%%%%%%%%%%%%%%%%%%%%%%%%%%%%%%%%%%%%%%%%%%%%
%% pretty printing:
\usepackage{listings}
%%%%%%%%%%%%%%%%%%%%%%%%%%%%%%%%%%%%%%%%%%%%%%%%%%%%%%%%%%%%%%%%%%%%%%
%% XML code setup:
\lstloadlanguages{XML}
%%
\lstset {
  showstringspaces=false,
  basicstyle=\ttfamily\footnotesize,
  lineskip=0pt,
  breaklines=true,
  breakatwhitespace=true,
  breakindent=12pt,
  fontadjust=true,
  keywordstyle=\bfseries,
  commentstyle=\bfseries,
  stringstyle=\bfseries,
  xleftmargin=0mm,
  xrightmargin=0mm,
  tabsize=2
}
%%%%%%%%%%%%%%%%%%%%%%%%%%%%%%%%%%%%%%%%%%%%%%%%%%%%%%%%%%%%%%%%%%%%%%
%% convenient referencing:
\usepackage[capitalize]{cleveref}
%%%%%%%%%%%%%%%%%%%%%%%%%%%%%%%%%%%%%%%%%%%%%%%%%%%%%%%%%%%%%%%%%%%%%%

%%%%%%%%%%%%%%%%%%%%%%%%%%%%%%%%%%%%%%%%%%%%%%%%%%%%%%%%%%%%%%%%%%%%%%
%% make definition of abbreviations easier
%% (td: really? I am the only one to use this all the time?)
\usepackage{xspace}
%%%%%%%%%%%%%%%%%%%%%%%%%%%%%%%%%%%%%%%%%%%%%%%%%%%%%%%%%%%%%%%%%%%%%%

%%%%%%%%%%%%%%%%%%%%%%%%%%%%%%%%%%%%%%%%%%%%%%%%%%%%%%%%%%%%%%%%%%%%%%

%%%%%%%%%%%%%%%%%%%%%%%%%%%%%%%%%%%%%%%%%%%%%%%%%%%%%%%%%%%%%%%%%%%%%%
%%%%%%%%%%%%%%%%%%%%%%%%%%%%%%%%%%%%%%%%%%%%%%%%%%%%%%%%%%%%%%%%%%%%%%
%%
%% The following defines language specific words
%%   These are internal commands. They are not used in the main file.
%%   Langugage specific word commands always starts with '\word'
%%
%%%%%%%%%%%%%%%%%%%%%%%%%%%%%%%%%%%%%%%%%%%%%%%%%%%%%%%%%%%%%%%%%%%%%%
%% Quelle/Source
\newcommand{\wordsource}{\iflanguage{english}{Source:\ }{\iflanguage{german}{Quelle:\ }{\langerrmessage}}}
%%%%%%%%%%%%%%%%%%%%%%%%%%%%%%%%%%%%%%%%%%%%%%%%%%%%%%%%%%%%%%%%%%%%%%

%%%%%%%%%%%%%%%%%%%%%%%%%%%%%%%%%%%%%%%%%%%%%%%%%%%%%%%%%%%%%%%%%%%%%%
%% \internmakethreecolumns{entry1}{entry2}{entry3}
%%   creates a table with three columns containing the given entries
\newcommand{\internmakethreecolumns}[3]{
  \noindent
  \begin{tabular*}{\textwidth}{@{}l@{}l@{}l@{}}
    #1 & #2 & #3 \\
  \end{tabular*}
}
%%%%%%%%%%%%%%%%%%%%%%%%%%%%%%%%%%%%%%%%%%%%%%%%%%%%%%%%%%%%%%%%%%%%%%
%% citation
%%   returns the language specific citation of the paper
\newcommand{\interncitation}{%
  \myfirstauthorREF\ (\myyear) \mytitle, \internpapertype, %
  \iflanguage{english}{ETH-Zurich, Zurich.}%
  {\iflanguage{german}{ETH-Z\"urich, Z\"urich.}%
   {\langerrmessage}}}
%%%%%%%%%%%%%%%%%%%%%%%%%%%%%%%%%%%%%%%%%%%%%%%%%%%%%%%%%%%%%%%%%%%%%%

%%%%%%%%%%%%%%%%%%%%%%%%%%%%%%%%%%%%%%%%%%%%%%%%%%%%%%%%%%%%%%%%%%%%%%
%% Figure definition
%%   \createfigure
%%     {<short caption>}
%%     {<long caption>}
%%     {<\label{label}>}
%%     {<\includegraphics[...]{figure}>}
%%     {<source>or<>}
\newcommand{\createfigure}[6][htp]{%
\begin{figure}[#1] % do not indent this line; produces unwanted spaces.  kai, dec'14
    \caption[#2]{#3} #4
    \vspace{0.5em}
    \hrule
    \begin{center}
      #5\\
    \end{center}
    \ifthenelse
      {\equal{#6}{}}
      {}
      {\wordsource #6 \vspace{0.5em}}
    \hrule
  \end{figure}%
}
% The syntax "[6][htp]{...[#1]..." means that the first parameter is
% optional. kai, dec'14 
%
% Kommentar Kai: Mir macht diese private
% Definition einige Probleme, weil mein emacs syntax-parser (der
% z.B. bei den Querverweisen hilft), damit natürlich nicht zurecht
% kommt. :-(
%%%%%%%%%%%%%%%%%%%%%%%%%%%%%%%%%%%%%%%%%%%%%%%%%%%%%%%%%%%%%%%%%%%%%%
%% Sideways figure definition
%%   \createfigure
%%     {<short caption>}
%%     {<long caption>}
%%     {<\label{label}>}
%%     {<\includegraphics[...]{figure}>}
%%     {<source>or<>}
\newcommand{\createsidewaysfigure}[5]{
  \begin{sidewaysfigure}
    \caption[#1]{#2} #3
    \vspace{0.5em}
    \hrule
    \begin{center}
      #4\\
    \end{center}
    \ifthenelse
      {\equal{#5}{}}
      {}
      {\wordsource #5 \vspace{0.5em}}
    \hrule
  \end{sidewaysfigure}
}
%%%%%%%%%%%%%%%%%%%%%%%%%%%%%%%%%%%%%%%%%%%%%%%%%%%%%%%%%%%%%%%%%%%%%%
%% Sub-figure:
%%   \createsubfigure
%%     {<caption>}
%%     {<\includegraphics[...]{figure}>}
%%     {<\label{label}>}
%%     {<\\>or<>}%
\newcommand{\createsubfigure}[4]{
  \subfigure[#1]{%
    #2%
    #3%
  }#4
}
%%%%%%%%%%%%%%%%%%%%%%%%%%%%%%%%%%%%%%%%%%%%%%%%%%%%%%%%%%%%%%%%%%%%%%
%% Table:
%%   \createtable
%%     {<short caption>}
%%     {<long caption>}
%%     {<\label{label}>}
%%     {<\begin{tabular}...\end{tabular}>}
%%     {<source>or<>}
\newcommand{\createtable}[6][htp]{
  \begin{table}[#1]
    \caption[#2]{#3} #4
    \vspace{0.5em}
    \hrule
    \begin{center}
      #5\\
    \end{center}
    \ifthenelse
      {\equal{#6}{}}
      {}
      {\wordsource #6 \vspace{0.5em}}
    \hrule
  \end{table}
}
%%%%%%%%%%%%%%%%%%%%%%%%%%%%%%%%%%%%%%%%%%%%%%%%%%%%%%%%%%%%%%%%%%%%%%
%% Sideways table definition
%%   \createsidewaystable
%%     {<short caption>}
%%     {<long caption>}
%%     {<\label{label}>}
%%     {<\begin{tabular}...\end{tabular}>}
%%     {<source>or<>}
\newcommand{\createsidewaystable}[5]{
  \begin{sidewaystable}
    \caption[#1]{#2} #3
    \vspace{0.5em}
    \hrule
    \begin{center}
      #4\\
    \end{center}
    \ifthenelse
      {\equal{#5}{}}
      {}
      {\wordsource #5 \vspace{0.5em}}
    \hrule
  \end{sidewaystable}
}
%%%%%%%%%%%%%%%%%%%%%%%%%%%%%%%%%%%%%%%%%%%%%%%%%%%%%%%%%%%%%%%%%%%%%%
%% XML-figure
%%   \createxmlfigure
%%     {<short caption>}
%%     {<long caption>}
%%     {<\label{label}>}
%%     {<the/file/with/the/xml/code/to/include>}
%%     {<source>or<>}
\newcommand{\createxmlfigure}[5]{
  \begin{figure}
    \caption[#1]{#2} #3
    \vspace{0.5em}
    \hrule
    \lstset{language=XML}
    \lstinputlisting{#4}
    \ifthenelse
      {\equal{#5}{}}
      {}
      {\wordsource #5 \vspace{0.5em}}
    \hrule
  \end{figure}
}
%%%%%%%%%%%%%%%%%%%%%%%%%%%%%%%%%%%%%%%%%%%%%%%%%%%%%%%%%%%%%%%%%%%%%%
%% Contact
%%   \createcontact
%%     {<Name>}
%%     {<andreas line 1>}
%%     {<andreas line 1>}
%%     {<andreas line 3>}
%%     {<phone number>}
%%     {<fax number>}
%%     {<email address>}
\newcommand{\createcontact}[7]{
  \ifthenelse{\equal{#1}{}}{%
    \noindent\parbox[][][l]{0.33\textwidth}{
    }%
  }{%
    \noindent\parbox[][][l]{0.33\textwidth}{
      #1\newline
      #2\newline
      #3\newline
      #4\newline
      \wordphone: #5\newline
      \wordfax: #6\newline
      #7\newline
    }%
  }%
}

%%%%%%%%%%%%%%%%%%%%%%%%%%%%%%%%%%%%%%%%%%%%%%%%%%%%%%%%%%%%%%%%%%%%%%
%% New titlepage definition
\newcommand{\createtitlepage}{
  \begin{titlepage}
    \begin{center}

      \bigskip
      
      \bigskip

      \bigskip
      
		

      {\textbf{\Large\mytitle}}

      \bigskip

      \bigskip
      
      \bigskip
      
      \includegraphics[width=0.8\textwidth]{figures/matsimcycle}

    \end{center}
  \end{titlepage}
}

%%%%%%%%%%%%%%%%%%%%%%%%%%%%%%%%%%%%%%%%%%%%%%%%%%%%%%%%%%%%%%%%%%%%%%


%%%%%%%%%%%%%%%%%%%%%%%%%%%%%%%%%%%%%%%%%%%%
%%%%%%%%%%%%%%%%%%%%%%%%%%%%%%%%%%%%%%%%%%%%
\newcommand{\createStandardInformationBasic}[4]{{\lstset{basicstyle=\normalsize\tt}%
%
\textbf{Entry point to documentation:}
\\
#1

\textbf{Invoking the module:}
\\
#2

\textbf{Configuration:}
\\
#3

\textbf{Selected publications:}
\\
#4
%
}}

%%%%%%%%%%%%%%%%%%%%%%%%%%%%%%%%%%%%%%%%%%%%
%%%%%%%%%%%%%%%%%%%%%%%%%%%%%%%%%%%%%%%%%%%%
\newcommand{\createStandardInformation}[4]{{\lstset{basicstyle=\normalsize\tt}%
%
\section{Basic Information}
\label{sec:#1-stdInfo}
%
\textbf{Entry point to documentation:}
\\
\url{http://matsim.org/javadoc} $\to$ #1

\textbf{Invoking the module:}
\\
\url{http://matsim.org/javadoc} $\to$ #1 $\to$ #2

\textbf{Configuration:}
\\
\lstinline{#3} section(s) of the config file (only available when the #1 extension is correctly loaded).

\textbf{Selected publications:}
\\
#4
%
}}
%%%%%%%%%%%%%%%%%%%%%%%%%%%%%%%%%%%%%%%%%%%%
%%%%%%%%%%%%%%%%%%%%%%%%%%%%%%%%%%%%%%%%%%%%
