% ##################################################################################################################
\section{Sioux Falls}
\label{ch:scenarios:siouxfalls}
\hfill \textbf{Author:} Artem Chakirov

The Sioux Falls scenario aims to provide a convenient test-case by combining a fully dynamic demand with realistic socio-economic and demographic attributes with a small scale road network and feautiring an integrated public transportation system. Based on the Sioux Falls road network commonly used for tests and demonstration purposes in the transportation literature \citep[][]{BarGera_TNTP_Webpage_2013}, it attempts to allow fast testing  of transportation policies and new software extensions of the MATSim framework with heterogeneous users and a high degree of spatial resolution, but at the same time without significant computational requirements. However, it is important to stress that despite the use of real world data for the generation of the enriched Sioux Falls scenario, it does not aim to replicate the real City of Sioux Falls in South Dakota, US and remains a fictitious test-bed. Detailed report on scenario generation and its characteristics is provided \citet[][]{ChakirovFourie_TechRep_FCL_2014} and can be found at www.matsim.org/scenario/sioux-falls. 

% --------
\paragraph{Demand:}

A realistic, socio-economically and demographically diverse demand population with  heterogeneous use preferences is crucial for unlocking the full potential of an agent-based simulation as MATSim. However, the generation of a disaggregated demand description on individual and household levels with close resemblance of reality, is challenging not only in terms of trip origins and destination, but also with respect to associating travel patterns with socio-demographic characteristics.

In order to address this challenge in case of Sioux Falls scenario and represent the household structure, demographic profile and income distribution as realistically as possible, a synthetic population of households using entropy maximization technique is generated. It matches the aggregate distribution of demographic attributes (age, sex and household income) recorded during the 2010 US Census for the 27 census tracts inside and adjoining the city center of the City of Sioux Falls and is composed of household and person records taken from the (anonymous) 5-year sample (2007-2011) of the American Community Survey, covering 5.0\% of all households.

In order to keep the scenario accessible and to facilitate interpretation and understanding of the possible effects of policies under study, only two simple activity chains are initially included: home – work – home and home – other – home. Activity locations are identified using the data set of real building stock provided by the City of Sioux Falls GIS division. The home location of each household is assigned randomly to a residential unit within the tract it belongs to. As no information on the real number and distribution of work places within the relevant area is easily accessible, the static OD-Matrix from \citet[][]{LeBlancEtAl_TransRes_1975} are taken as an indicator of workplace attraction for each zone. Subsequently, the assignment of work places to individual workers as well as locations of secondary (other) activities is performed using a parameter free radiation model presented by \citet[][]{SiminiEtAl_NAT_2012}.

In order to exploit the full potential of a disaggregated demand and to add another degree of realism to the scenario, a car ownership on the household level is modeled using an ordered probit model, presented by \cite[][]{GiulianoDargay_TransResA_2006} and based on the US Nationwide Personal Transportation Survey (NPTS) 1995. Next to socio-demographic characteristics of a household (number of adults, children, pensioners, household income), the model uses attributes of residential location (population density, public transport access and dwelling type), which allows better accounting for specific characteristics of the Sioux-Falls scenario and its area-wide bus network. 

% --------
\paragraph{Supply:} 

A transportation test network should ensure sufficient complexity of travelers’ choice dimensions while limiting computational effort. To this end, the Sioux Falls test network was introduced by \citet[][]{MorlokEtAl_ResRep_org-fhwa_1973} and later adapted as a benchmark and test scenario in many publications (see \citet[][]{ChakirovFourie_TechRep_FCL_2014} for overview). The structure of this network captures the major arterial roads of the City of Sioux Falls in South Dakota, but was never intended to replicate the real city or all characteristics of its transportation system, such as travel times or mode share. The network is comprised of 76 arcs, 24 nodes and 552 origin-destination pairs. For this scenario the capacities of the roads are adjusted according to values provided by the Highway Capacity Manual \citet[][]{HCM_2010} and other related research publications (e.g. \citet[][]{NgCFSmall_Transportation_2012}). The public transportation network includes 5 bus lines, as initially proposed by \citet[][]{AbdulaalLeBlanc_TransScience_1979}, with bus stops placed in constant distance of 600m away from each other. 

Due to the design of MATSim’s queue simulation, agents are only handled at the beginning and end of each network link, and cannot enter or leave a link along its length. Therefore, origins and destinations located along very long links will lead to a loss of spatial detail, as all origins and destinations along the length of the link are effectively assigned the same coordinate. Consequently, to improve the level of spatial detail, all links of the Sioux Falls network are evenly split into smaller links with maximal length of 500 meters each. Following this operation, the network detail increased from 27 to 282 nodes and the number of links increased from 76 to 334.

In addition to car and bus modes, walking as 'teleported' mode with constant travel speeds and without any interaction with other users is used as the non-motorized mode of transportation. 

% --------
\paragraph{Behavioral parameters:}

The behavioral parameters used in utility functions of activities and trips are based on the estimated demand model for Sydney by \citet[][]{TirachiniHensherRose_TransResB_2014}. Before applying the parameters in an activity-based context, the time related parameters need to be adjusted to account for utility gained from activity performance. Thereby, in order to provide a value for marginal utility of performing an activity, the travel mode with smallest disutility is set as a baseline, under the assumption that traveling with this mode is as good or as bad as idling and doing nothing. The corresponding parameters are split into opportunity costs of time and a mode-specific disutility of traveling, as also done in previous MATSim related publications as e.g. \citet[][]{KickhoeferEtAl_Transportation_2011}. 

% --------
\paragraph{Results, drawbacks and outlook:}

The stability and performance of the Sioux Falls scenario has been tested using two set of activity timing constrains as well as 5 different random seeds. which all delivered stable and realistic results. Furthermore \citet[][]{ChakirovFourie_TechRep_FCL_2014} also investigate the existence and the hysteresis characteristic of the Macroscopic Fundamental Diagram, as discussed  by \cite[][]{GeroliminisDaganzo_TRB_2007, GeroliminisDaganzo_TransResB_2008, GeroliminisSun_TransResA_2011}. 

However, the more recent experience has shown certain drawbacks of the coarse network representing only major arterial roads and neglecting miner neighborhood and collector road links. With an elaborate synthetic population and high demand peaks during rush hours, the network appears to be sensible to network breakdowns under high loading conditions. 

Along with the coarse road network, the coarse level of public transport network and therewith its low level of accessibility for parts of the population represents another drawback, in particular relevant for simulation and evaluation of policies sensible to or requiring certain share of public transport users. 

Substituting, the original Sioux-Falls networks with a finer network obtained from the crowd-sourced Open Street Map and potentially adding additional public transport line to it, would allow to address this problem. However it brings different sets of drawbacks arises from it. First, the significantly larger number of links and nodes in the network increases time and resources for routing and dynamic queue simulation and threatens to erase advantages of a small scale network. Second, the increase in total network capacity leads to the disappearance of congestion during peak hours. 
Still, having substantial benefits, the  suggested improvements are in the process of being implemented. The extended simulation times can be tackled with the new pseudo-simulation methodology, currently developed by \citet[][]{FourieEtAl_TRR_2013}. Furthermore, inclusion of freight and through traffic into the scenario would increase the degree of realism and cater for congested conditions during peak-hours. 


% ##################################################################################################################