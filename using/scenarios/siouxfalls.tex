% ##################################################################################################################
\section{City of Sioux Falls, South Dakota}
\label{ch:scenarios:siouxfalls}
\hfill \textbf{Author:} Artem Chakirov

The Sioux Falls scenario represents a small scale scenario featuring realistic demand with socio-economic and demographic attributes and an integrated public transport system. Based on the  Sioux Falls road network commonly featured as a test-case in the transportation literature \citep[][]{BarGera_TNTP_Webpage_2013}, the scenario aims to provide a fully dynamic demand with heterogeneous users and a high degree of spatial resolution. It can serve as a convenient test-case for the study of different transportation policies as well as a test bed for the software extension and development. However, it is important to stress that despite the use of real world data for its generation, the scenario does not aim to replicate the real City of Sioux Falls in South Dakota, US and remains a fictitious test-case scenario. Detailed report on scenario generation and its characteristics is provided \citet[][]{ChakirovFourie_TechRep_FCL_2014} and can be found at www.matsim.org/scenario/sioux-falls. 

Often used test scenario . Not aimed at replicating the real City of Sioux Falls, South Dakota.

% --------
\paragraph{Demand:}

A realistic, socio-economically and demographically diverse demand population with  heterogeneous preferences is crucial for unlocking the full potential of an agent-based simulation. However, the generation of a disaggregated, agent-based demand description, which closely resembles reality, not only in terms of trip origins and destination, but also with respect to associating travel patterns with socio-demographic characteristics, is challenging.

In order to address this challenge in case of Sioux Falls scenario and represent the household structure, demographic profile and income distribution as realistically as possible, a synthetic population of households using entropy maximization technique is generated. It matches the aggregate distribution of demographic attributes (age, sex and household income) recorded during the 2010 US Census for the 27 census tracts inside and adjoining the city centre of Sioux Falls and is composed of household and person records taken from the (anonymous) 5-year sample (2007-2011) of the American Community Survey, covering 5.0\% of all households.

On the demand side of the activity-based model only two simple activity chains are initially included, in order to keep the scenario accessible and to facilitate interpretation and understanding of the possible effects of policies under study. Modelling only home – work – home and home – other – home activity chains, destinations are assigned using a parameter-free radiation model (Simini et al. 2012).

Activity locations are identified using the data set of real building stock provided by the City of Sioux Falls GIS division. The home location of each household is assigned randomly to a residential unit within the tract it belongs to. As no information on the real number and distribution of work places within the relevant area is easily accessible, the static OD-Matrix from LeBlanc et al. (1975) are taken as an indicator of workplace attraction for each zone. Subsequently, the assignment of work places to individual workers was performed using a parameter free radiation model presented by Simini et al. (2012).

In order to exploit the full potential of a disaggregated demand and to add another degree of realism to the scenario, a car ownership model is applied to the synthetic population of households. An ordered probit model estimated by Giuliano and Dargay (2006)based on the US Nationwide Personal Transportation Survey (NPTS) 1995. Next to socio-demographic characteristics of a household (number of adults, children, pensioners, household income), the model uses attributes of residential location (population density, public transport access and dwelling type), which allows to better account for characteristics of the Sioux-Falls scenario with its added area-wide bus network. 

% --------
\paragraph{Supply:} 

A transportation test network should ensure sufficient complexity of travellers’ choice dimensions while limiting computational effort. To this end, the Sioux Falls test network was introduced by Morlok et al. (1973) and later adapted as a benchmark and test scenario in many publications, as e.g. LeBlanc et al. (1975), Abdulaal and LeBlanc (1979a), Suwansirikul et al.  (1987), Friesz et al. (1992), Meng et al. (2001), Bar-Gera et al. (2013) to name but a few. The structure of this network captures the major arterial roads of the City of Sioux Falls, but was never intended to replicate the real city or all characteristics of its transportation system, such as travel times or mode share. The network os comprised of 76 arcs, 24 nodes and 552 origin-destination pairs. For this scenario the capacities of the roads were adjusted according to values provided by the Highway Capacity Manual (TRB, 2000) or other related research publications (e.g. Ng and Small, 2011). The public transportation network includes 5  bus lines, as initially proposed by Abdullal and LeBlanc (1979), with bus stops placed in constant distance of 600m away from each other. 

Due to the design of MATSim’s queue simulation, agents are only handled at the beginning and end of each network link, and cannot enter or leave a link along its length. Therefore, origins and destinations located along very long links will lead to a loss of spatial detail, as all origins and destinations along the length of the link are effectively assigned the same coordinate. Consequently, to improve the level of spatial detail, all links of the Sioux Falls network are evenly split into smaller links with maximal length of 500 meters each. Following this operation, the network detail increased from 27 to 282 nodes and the number of links increased from 76 to 334.

In addition to car and bus modes, walking as 'teleported' mode with constant travel speeds and without any interaction with other users is used as the non-motorized mode of transportation. 

% --------
\paragraph{Behavioural parameters:}

The behavioural parameters used in utility functions of activities and trips are based on the estimated demand model for Sydney by Tirachini, Hensher, and Rose (2012). The time-related parameters derived in their model have to be adjusted for application in the activity-based context. In order to provide a value for marginal utility of performing an activity, the travel mode with smallest disutility is set as a baseline, under the assumption that travelling with this mode is as good or as bad as idling and doing nothing. Therefore, the corresponding parameters are split into opportunity costs of time and a mode-specific disutility of travelling, as was done previously by Kickhöfer et al. (2011), Kickhöfer et al. (2013), Kaddoura et al. (2013), Kickhöfer and Nagel (2013), who employed the same parameters within their MATSim based studies. 

% --------
\paragraph{Results, drawbacks and outlook:}

The stability and performance of the Sioux Falls scenario has been tested using two set of activity timing constrains as well as 5 different random seeds. which all delivered stable and realistic results. Furthermore \citet[][]{ChakirovFourie_TechRep_FCL_2014} also investigate the existence and shape of the Macroscopic Fundamental Diagram, as demonstrated by Geroliminis and Daganzo (2007, 2008). 

However, the more recent experience has shown certain drawbacks of the coarse network representing only major arterial roads and neglecting finer neighbourhood access road network. Combined with the realistic synthetic population and high demand peaks during rush hours, the network appears to be sensible to network breakdowns under high loading conditions. 

Along with the coarse road network, the coarse level of public transport network and therewith its low level of accessibility and attraction represents another drawback, in particular relevant for simulation and evaluation of policies sensible to or requiring certain share of public transport users. 

Substituting, the original Sioux-Falls networks with its only 76 links with a finer Open Street Map network and adding additional public transport line to it, allows to address this problem, but brings another drawbacks at the same time. First, it increases time and resources for routing and dynamic queue simulation due to the significantly larger number of links and nodes present in the network. Second, the increase in total network capacity leads to the disappearance of congestion during peak hours.

The extended simulation times can be improved by using the new pseudo-simulation methodology, currently developed by Fourie (2014). The larger network capacity can be addressed by the inclusion of freight and through traffic into the scenario. 

Drawbacks: course resolution of the network
public transport is not attractive and accessible enough, especially for experiment design in european context
no freight of through traffic


% ##################################################################################################################