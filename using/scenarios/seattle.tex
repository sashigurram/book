% ##################################################################################################################
\section{Seattle Region}
\label{sec:seattle}
\hfill \textbf{Author:} Kai Nagel

% ##################################################################################################################
A \acrshort{matsim} model of the Seattle region -- more precisely of the Puget Sound Regional Council (PSRC) area -- was developed during a sabbatical stay of K.\ Nagel with the UrbanSim team in Seattle in 2008. The model resulted from a prototypical integration of the UrbanSim software \citep[e.g.][]{WaddellEtc2003UrbanSim} with \acrshort{matsim}. 

The base was an existing PSRC UrbanSim model, which used an existing EMME/2 model \kai{citation!} \ah{Etwa so? \citep[][]{RolleEtAl_ITE_2007, EMME_Webpage_2008}} as a travel model. It was investigated how difficult it would be to replace the EMME/2 model by \acrshort{matsim}.

The network was taken by conversion from the existing EMME network, resulting in 15478~links and 5025~nodes with attributes length, free speed, and capacity.

The demand was generated as output from UrbanSim. Evidently, the UrbanSim simulation already contained a full synthetic population of the area. The UrbanSim model was also set up with workplace choice, so that each synthetic person with status ``working'' had a workplace assigned. Since that version of UrbanSim worked on the parcel level, this meant that home-to-work trips could be extracted directly, with coordinates, from the model. As so often for initial \acrshort{matsim} studies, this home-to-work demand was completed into home-work-home plans.

The configuration used the standard \acrshort{matsim} scoring parameters, a workplace opening time of 07:00:00, and a latest work start time of 09:00:00. The iterations were run with re-routing and time mutation enabled until convergence. Since this was an exercise in rapid prototyping, only a 1\% sample of the full synthetic population was used; the flow and storage capacities of the road network were scaled down accordingly. Figure~\ref{fig:seattle-snapshot.left} shows a result. Figure~\ref{fig:seattle-snapshot.right} shows households most affected by a hypothetical closure of the Alaskan Way viaduct, which bypasses the Seattle downtown area on the waterfront side to the west.

\createfigure%
{Seattle Region Scenario}%
{Seattle Region Scenario}%
{\label{fig:seattle-snapshot}}%
{%
  \createsubfigure%
  {Simulated congestion patterns in Seattle at 7:30am}%
  {\includegraphics[width=0.49\textwidth,angle=0]{using/figures/seattle-snapshot-7h30.pdf}}%
  {\label{fig:seattle-snapshot.left}}%
  {}%
  \createsubfigure%
  {10\% households most affected by closure of the so-called Alaskan Way Viaduct (in red)}%
	{\includegraphics[width=0.49\textwidth,angle=0]{using/figures/seattle-top-10pct-0it.pdf}}%
  {\label{fig:seattle-snapshot.right}}%
  {}%
}%
{}

% Removed only for consistency:
%
%\begin{figure}[!t!]
%\centerline{%
%\includegraphics[height=0.8\hsize,trim=0 0 0 0,clip]{using/figures/seattle-snapshot-7h30.pdf}
%%
%\includegraphics[height=0.8\hsize,trim=0 0 0 0,clip]{using/figures/seattle-top-10pct-0it.pdf}
%}
%\caption{Left: Simulated congestion patterns in Seattle at 7:30am. Right: 10\% households most affected by closure of the so-called Alaskan Way Viaduct (in red).}
%\label{fig:seattle-snapshot}
%\end{figure}

% ##################################################################################################################


% Local Variables:
% mode: latex
% mode: reftex
% mode: visual-line
% TeX-master: "../../main"
% comment-padding: 1
% fill-column: 9999
% End: 
