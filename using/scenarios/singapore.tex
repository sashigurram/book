% ##################################################################################################################
\section{Singapore}
\label{ch:scenarios:singapore}
\hfill \textbf{Author:} Alexander Erath, Artem Chakirov

The MATSim Singapore scenario is implemented and maintained at the Future Cities Laboratory, a research programme of the Singapore-ETH Centre for Global Environmental Sustainability (SEC) and part of the Singapore National Research Foundation initiative CREATE (Campus for Excellence and Technological Enterprise). The scenario covers the whole area of the island city-state of Singapore, involving population of approximately 5 million inhabitants. As a fairly densely populated island with an extensive public transport infrastructure and advanced transportation policy and pricing policies, Singapore provides an excellent study case for an agent- and activity based modelling approach. 
Limitations in availability of data being commonplace in other countries on the one hand and existence of alternative, unique and exhaustive data sets on the other, lead to number of experimental modelling approaches applied in this implementation.  A detail description of the modelling process and methodology is also provided by \citet[][]{ErathEtAl_TechRep_FCL_forth, Erath_unpub_UniSeoul_2011}.

\subsection{Demand}
In the absence of a full-population census for Singapore, the agent population is generated by combining data from national travel diary survey (HITS 2008, reported by \citet[][]{Choi_JOUR_2010}) and the publicly available breakdowns of Singapore’s most recent population census. Applying iterative proportional updating (IPU) and weighted random sampling methods, the travel survey data containing 1 \% sample of Singapore’s permanent residents serves as a base for generation of a full scale synthetic population. Further details on this approach and used methodology can be found in \citet[][]{FourieMueller_HKSTS_2011} and \citet[][]{CasatiEtAl_TechRep_IVT_2014}. 
Singapore’s quotas licence system for car ownership with extremely high ownership cost but comparably low usage cost, requires an accurate modelling of car license holdership and vehicle ownership on household level. The model presented in \citet[][]{VanEggermondEtAl_IATBR_2012} includes not only socio-economic but also spatial variables and have proved to be essential feature of Singapore’s transport model, leading to accurate mode choice and mode share predictions. 
Activity locations are defined at a single building level, whereby the information on building and facility types is compiled from various sources as land-use master plan, commercial and government websites as well as building footprints from navigational NAVTEQ network. With no business census being available, modelling of work places requires an alternative, innovative approach.  Drawing from the full smart card data record of public transport  journeys and enriching it with the information on land-use and building floor areas, estimation of work place locations and there capacities is performed. Thereby, the method is based on identification of workers arriving at each public transportation stop or station using probabilistic model calibrated with travel interview survey (HITS) and considering time of day, activity duration and land-use around each stop or station. After accounting for mode shares of 53 different regions, an optimization technique employing accessibility computation is applied in order to distribute work places to a single building level.  \citet[][]{ChakirovErath_IATBR_2012} and \citet[][]{OrdonezErath_TRR_2013} describe the developed methodology and its practical application in detail. 
Assignment of home locations to individual households from the synesthetic population is performed on a building level using information on residential developments and unit types within each land-use planning zone. For assignment of work locations to a working population a gravity-model approach is applied, using the approximated number of work places in each building as an additional constraint.  Activity chains are assigned based on their observed frequency in HITS, with home – work – home chain being significantly dominant and accounting for approx. 50% of the trips. 
Additional activity plans for freight, cross border and tourist travel demand are derived based on the set of origin destination matrices provided by the Singapore Land Transport Authority (LTA). The matrices are converted into special daily plans and trips are distributed spatially and temporally using methods described in detail in Section 6.1.6 of \citet[][]{ErathEtAl_TechRep_FCL_forth}.

\subsection{Supply}
For the Singapore model two road networks are combined in order to achieve a sufficient level of detail with accurate capacity values of expressways and arterial roads. Using a semi-automatic map-matching algorithm a high-resolution NAVTEQ navigation network is map-matched to and enhanced with the lane and capacity information from a more granular but also more accurate planning network, provided by the LTA. Thereby the travel delays result only from the network capacity restrictions and the probabilistic approach of link space distribution among agents inherent to the queue based traffic simulation. In the current version of the scenario, no traffic light model is included due to the unavailability of data on signal timings and schedules. 
Public transport network, including bus and train routes, stops and stations, is also obtained from the LTA and is matched with the road network using yet another map-matching algorithm presented by \citet[][]{Ordonez_HKSTS_2011, Ordonez_Webpage_2011_4}. A more recent feature of the public transport simulation in the Singapore scenario is the realistic public transport schedule derived from the full record of public transport journeys – the same data set as already used for the identification of work locations \citet[][]{Fourie_TechRep_FCL_2014}. The accurate schedule allows realistic modelling of vehicle dispatch frequencies and headways, which undergo continuous adjustments and in some cases can substantially deviate from the published schedule. Furthermore, additional features of the public transport simulation in Singapore’s model include advanced bus dwell time model \cite[][]{SunEtAl_TransResA_2014} as well as an approximation of distance based public transport fare scheme. The high degree of attention given to the public transport simulation reflects the importance of interaction between private and public transport in Singapore’s context. Large number of buses accounts for a significant share of road traffic and simulating dynamic effects as bus bunching is crucial for obtaining realistic travel times and mode shares. Other modes, as walk and bike are ``teleported'' with constant travel speeds and without any interaction with other users. 

\subsection{Policy}
This scenario also makes use of MATSim capability to model cordon tolls, as they are implemented in in Singapore with the Electronic Road Pricing (ERP) scheme featuring time and vehicle dependent road pricing. Based on two data sets indicating the location and time dependent price levels, prevailing tolls are specified for 69 network links with the installed tolling gantries. 
Dedicated bus lanes are another noticeable policy measure, which cannot be neglected in Singapore’s context. In order to account for them, addition links, used only by buses, are added to the network. With some of the bus lanes being exclusively reserved for public transport only during peak hours, such simplified setup leads to the underestimation of actual road capacity during time periods when bus lanes are also open for private cars. However, as most links featuring bus lanes consist of three or more lanes, the effect on modelled traffic conditions during off-peak hours appears to be low.

\subsection{Calibration and validation}
For validation road count data on an hourly base are available for 200 stations in Singapore. The availability of full record of public transport journeys provides an additional dimension for validation. In particular opening of new mass rapid transit (MRT) lines in the meantime presents a unique opportunity for comparison the observed ridership with predicted ridership in the model. However, this exercise and its detailed evaluation are yet to be conducted. 

% ##################################################################################################################