% ##################################################################################################################
\section{Patna}
\label{sec:patna}
\hfill \textbf{Author:} Amit Agarwal

% ##################################################################################################################
Patna is a medium-size city in eastern India. As in other developing nations, traffic conditions are heterogeneous, composed of: a large number of bikes (37\,\%, including 4\,\% cycle rickshaws) and motorbikes (14\,\%). When this scenario was composed, public transport accounted for 18\,\% and walk for 29\,\%; only 2\,\% of all trips were made by car. Therefore, the \gls{matsim} queue simulation was modified to simulate travel demand under mixed traffic conditions.

A detailed Patna scenario description can be found in \citet[][]{AgarwalEtcMixedTraffic}. The scenario was created using household survey data from a comprehensive Patna mobility plan \citep[][]{TrippItransVks2009PatnaReport}, using the area within the Patna Municipal Corporation. The scenario consisted of 72\,zones, with a population of about 1.57\,million (year 2008), \gls{matsim} demand was generated using trip diaries, with car, motorbike and bike used as main congested modes (Figure~\ref{fig:patna0}). Passenger car units for vehicles were derived using effective area occupied by vehicles. To allow overtaking of slower vehicles (bike), by faster vehicles (car and motorbike), pre-existing, state-of-the-art \gls{fifo} queue simulation was overridden using earliest link exit time as shown in Figure~\ref{fig:patna1}. The behavior of traffic in modified queue simulation was then analyzed by plotting fundamental diagrams and space time trajectories for car, motorbike and bike.

%\ah{Could you please elaborate a little on computation time issue, i.e.,\,mention that there are none, if so.}
To rectify some of the artifacts of the travel time distributions for Patna, \gls{matsim} utility function was calibrated such that a mode share from real world data is replicated in the model. This was performed by allowing agents to switch modes. The model was validated using traffic count data and modal travel time distributions. The main shortcoming of the model seemed to be overly small average travel time of motorbikes. Although, no explicit experiment was performed to analyze the computational performance, during the simulations no noticeable loss of performance was found and therefore, the model seems to be useful for many areas, where mixed traffic conditions predominate.

\createfigure%
{Patna: Various vehicles on network, car in red, motorbike in blue and bike in green}%
{Patna: Various vehicles on network, car in red, motorbike in blue and bike in green}%
{\label{fig:patna0}}%
{\includegraphics[width=0.99\textwidth, angle=0]{using/figures/vehiclesOnNetwork}}%
{}

\createfigure%
{Patna: \protect\gls{fifo} approach and passing of bicycle by car on a link (not to scale)}%
{Patna: \protect\gls{fifo} approach and passing of bicycle by car on a link (not to scale)}%
{\label{fig:patna1}}%
{\includegraphics[width=0.99\textwidth, angle=0]{using/figures/FIFOandPassing}}%
{}

% ##################################################################################################################