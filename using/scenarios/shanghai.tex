% ##################################################################################################################
\section{Shanghai}
\label{sec:shanghai}
\hfill \textbf{Author:} Lun Zhang

% ##################################################################################################################
Shanghai, a city of about 20 million residential population, 6073 square kilometers land area, is the biggest metropolis in China. To fully integrate the activity-based demand modeling and further public transport models, the full implementation of MATSim for Shanghai is built to forecast precise traffic demand on network as well as scientific policy evaluation. The scenario contains 200 thousand synthetic persons and they are simulated on a network with 50 thousand links. Key features of Shanghai scenario are as follows.

The 1\% sample of the actual population, about 0.2 million agents, is used. To generate the population individual with personal attributes, MC (Monte Carlo method) are used to disaggregate available census data form the 4th Travel Survey of Residents.

The demand generation is based on 24 hour OD matrices generated from the GPS data and synthetic population. These OD are then disaggregated into individual trips. The activity-based modeling is used to generate initial population plans by five steps: activity chain choice, duration choice, mode choice, destination choice, and route choice, in which the MNL model is used to estimate and serial choices of agents. During the simulation, activity replanning are still introduced to learn the better travel plans, while scoring for a plan is modeled by using utility-based approach.

The street network of Shanghai has been extracted from the whole OpenStreetMap network and then merged with Shanghai expressway network. Road attributes such as the number of lanes per direction or the flow capacities are set via the specification of road classifications. To simplify the original network, the optimization rules for are designed to remove the useless information that increases the computational burden.

All facilities from OD pairs are classified into the particular zones via their geographical coordinates. Three main facilities types, home, work/education and leisure, are used. The names of origin and destination facilities are obtained via reverse geocoding and these facilities are classified by their names. The resolution of facilities is hectare, in which facilities and types are randomly created according to their coordinates.

Simulated modes are as follows. A public transport system, subways and buses, is integrated with both motor traffic and non-motorized traffic. Whether individuals go to their destination public transits is base on the transit schedules. And a transport mode decision model used to make mode choice by a car or by transit is developed based on the relative utility of travel time of the two modes.

Activity replanning are used to optimize activity plans of agents and the stable state of simulation system is reached after 100 iterations of replanning procedures. The effectiveness of the Shanghai MATSim transport simulation model is validated against the the observed counts from vehicle detectors and mode split from travel survey. Extensive simulation results indicate that most of the traffic simulation volumes match quite well with the observed counts, and the potential of MATSim for large-scale dynamic transport simulation has been demonstrated. It can provide researchers and policy makers a useful tool to evaluate traffic policies. 

The specific algorithms of integrating new data in Shanghai with MATSim inputs such as synthetic population, facilities and network are separately designed according to data characteristics. To see more detailed work about Shanghai scenario, please see the publications in reference \citet[][]{ZhangLEtAl_TRR_2014}.

% ##################################################################################################################