% ##################################################################################################################
\section{Shanghai}
\label{sec:shanghai}
\hfill \textbf{Author:} Lun Zhang

\editdone{This text has undergone the professional edit. Please no grammatical changes anymore! They are most-probably wrong.}

% ##################################################################################################################
Shanghai, with a population of about 20\,million and 6\,073\,square kilometers land area, is the biggest metropolis in China. To fully integrate activity-based demand modeling and further public transport models, the full implementation of \gls{matsim} for Shanghai was built to forecast precise traffic demand on network, as well as scientific policy evaluation. The scenario contained 200\,000\,synthetic persons, simulated on a network with 50\,000\,links. Shanghai scenario key features are as follows.

A 1\,\% sample of the actual population, about 0.2\,million agents, was used. To generate the population individual with personal attributes, the Monte Carlo method was used to disaggregate available census data from the 4th Travel Survey of Residents.

Demand generation was based on 24\,hour \gls{od} matrices generated from the \gls{gps} data and synthetic population; the \gls{od} were then disaggregated into individual trips. The activity-based modeling was used to generate initial population plans in five steps: activity chain choice, duration choice, mode choice, destination choice and route choice, where the \gls{mnl} model was used to estimate and serialize choices of agents. During the simulation, activity \gls{replanning} were introduced to discern better travel plans; while scoring for a plan was modeled using a utility-based approach.

The Shanghai street network was extracted from the overall \gls{osm} network and then merged with the Shanghai expressway network. Road attributes, such as  number of lanes per direction, or  flow capacities, were set through road classification specification. To simplify the original network, optimization rules were designed to remove unneeded information that increased computational burden.

All facilities from \gls{od} pairs were classified into particular zones using their geographical coordinates. Three main facilities types, home, work/education and leisure, are used. Origin and destination facilities' names were obtained via reverse \gls{geocoding}; these facilities are classified by their names. The unit resolution of facilities was the hectare, in which facilities and types are randomly created according to their coordinates.

Simulated modes were as follows. A public transport system, subways and buses, was integrated with both motor traffic and non-motorized traffic. Transit schedules were considered for public transport. A travel time based transport mode choice model---between car and public transport---was developed. % \ah{summarized and corrected this not-easily-understandable paragraph}

Activity replanning was used to optimize activity plans of agents; stable simulation system state was reached after 100\, replanning procedures iterations. The effectiveness of the Shanghai \gls{matsim} transport simulation model was validated against observed counts from vehicle detectors and mode split from travel surveys. Extensive simulation results indicate that most traffic simulation volumes matched quite well with observed counts, which demonstrated \gls{matsim}'s potential for \gls{largescale} dynamic transport simulation. It provides researchers and policy makers with a useful tool to evaluate traffic policies. 

Specific algorithms integrating new data in Shanghai with \gls{matsim} inputs, such as synthetic population, facilities and network, were separately designed according to data characteristics. To see more detailed work about the Shanghai scenario, please see \citet[][]{ZhangLEtAl_TRR_2014}.

% ##################################################################################################################