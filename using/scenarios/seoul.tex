% ##################################################################################################################
\section{Seoul}
\label{sec:seoul}
\hfill \textbf{Authors:} Seungjae Lee, Atizaz Ali

\editdone{This text has undergone the professional edit. Please no grammatical changes anymore! They are most-probably wrong.}

% ##################################################################################################################
The \gls{matsim} model of \gls{sma} was developed in 2012, as a result of long-term research collaboration between the University of Seoul (Prof.\ Seungjae Lee) \& \gls{eth} Zürich (Prof.\ Kay W.\ Axhausen). The model was updated yearly and demand was generated based on 2012 \gls{hhtsd}. Demand statistics (input) are summarized as follows. 

Study area was the \gls{sma} (Gyeonggi-do province, with emphasis on the Seoul Metro,  comprised of 25\,main administrative districts). A population synthesizer was developed to generate the \gls{matsim} input demand, based on \gls{hhtsd} 2012. Total population of \gls{sma} was 21.5\,million; therefore, a 10\,\% sample was generated and simulated (2.15\,million agents). A detailed nodes and links network was generated, capturing all details (16\,384 nodes and 32\,768 links) for railways, highways, arterials, pedestrians, expressways and bus-only lanes. \gls{emme2} network was converted to \gls{matsim} format. The 2012 Korean Transport Database was utilized to generate transit schedules and vehicle definitions, according to bus types, railway and metro lines. Total number of routes was 1\,317 (contained regional buses, inter-city buses, feeder line buses and metro lines, etc.). In collaboration with \gls{senozon}, a more realistic door-door demand was generated in Seoul City in July, 2014. Data source was the Korean \gls{gis} department.

In Seoul, \gls{matsim} has been widely used for various research purposes to aid policy evaluation \citet[e.g.,][]{KimEtAl_IJHE_2012, LeeAli_unpub_IWUTSCD_2014}.

A master's thesis on transit demand generation and calibration using smart card data in \gls{sma} is currently underway by this section's second author, sequenced as follows. A video is available from the authors on request.
%
\begin{compactitem}
\item Data mining (trimming off non-useful data)
\item	Converting disaggregate transactions (\gls{od}) to individual trips and trip segments based on user \lstinline|ID|
\item	Activities inference and assignment in \gls{spss} database
\item	Generating transit demand (\gls{matsim} input format)
\item	Updated transit network \& schedule for running the simulation
\item	Model calibration (in process)
\end{compactitem}
%
\gls{matsim} tutorials were also presented during the fall semester (2014) to help Department of Transportation Engineering undergrad and grad students gain a thorough working knowledge of \gls{matsim}.

\createfigure%
{Seoul scenario}%
{Seoul scenario}%
{\label{fig:seoul}}%
{\includegraphics[width=0.99\textwidth, angle=0]{using/figures/seoul}}%
{}

% ##################################################################################################################
 
 
