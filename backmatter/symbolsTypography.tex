% ##############################################################################################################################
\chapter*{Symbols and Typographic Conventions}
\label{ch:conventionsSymbols}
% ##############################################################################################################################
\section*{Symbols}
\ah{Sollte das in (verschiedene Glossaries rein?}
% ============================================================================================
\subsection*{Variables}
\ah{Bez�glich Gross- und Kleinschreibung habe ich jetzt grad ein bisschen ein Durcheinader mit Statistik, Algebra und Analysis. U wie Utility ist bei uns gross. t ist aber klein. Meist wird U aber als Funktion angegeben. Brauchen wir hier Konsistenz bis zum Kopfschmerz oder lassen wir die gewachsenen Bezeichnungen so?}


\begin{tabular}{l l}
	$c$ & VSP: monetary costs \\
  $d$ & distance \\
  $m$ & monetary costs \\
  $t$ & time \\
  $U$ & utility variable ($V + \epsilon$) \\
  $V$ & systematic component of utility variable \\
  $\beta$ & utility function coefficient \\
  $\hat{\beta}$ & estimated utility function coefficient \\
  $\epsilon$ & random component of utility variable \\
  $\phi$ & replanning share \\
  $\sigma$ & scale parameter of the multinomial logit model \\
  
\end{tabular}

% ============================================================================================
\subsection*{Indices and Subscripts}
\begin{tabular}{l l}
	$?$ & index of plans \\
  $i$ & index of agents. VSP: index of a plan activities\\
  $j$ & VSP: index of agents \\
  $q$ & index of plan activities \\
\end{tabular}

% ##############################################################################################################################
\section*{Typographic Conventions}
blabla

% ##############################################################################################################################
