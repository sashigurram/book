\chapter{Some history of \acrshort{matsim}}
\label{ch:history}
% ##################################################################################################################

\hfill \textbf{Authors:} Kai Nagel, Kay W.\ Axhausen

\begin{center} \includegraphics[width=0.3\textwidth, angle=0]{figures/MATSimBook.png} \end{center}

\kai{Auch dieses Module nur ein Vorschlag.  Ich nehme es erstmal als Ablage für Absätze, die ich in diese Richtung geschrieben habe.  Johan war daran mal sehr interessiert.}

%%%%%%%%%%%%%%%%%%%%%%%%%%%%%%%%%%%%%%%%%%%%
%%%%%%%%%%%%%%%%%%%%%%%%%%%%%%%%%%%%%%%%%%%%

%%TRANSIMS (already mentioned in Chapter~\ref{??}) was a collection of stand-alone modules, coupled by a script.  For example, the population synthesizer would generate a population file, the activity generate would take the population file as input and generate an activities file as output, etc.  Iterations were done by running the traffic microsimulation based on plans and outputting average link travel times, and then the router based on link travel times and outputting plans.

The first version of \acrshort{matsim} was, similar to the original TRANSIMS (already mentioned in Chapter~\ref{...}, a collection of stand-alone modules coupled by scripts.  For example, the router would read plans and events, and replace some of the plans by other plans with modified routes.  The program flow was organized with shell scripts and makefiles.  Later it was possible to start all modules simultaneously and they would use messages to interact \citep[also see][]{GloorNagel2005ped-att04-birkh}, but the file-based and scripted interaction always remained available.

That approach had, as a consequence, very clearly defined interfaces, i.e.\ the files.  Exchanging information that was not exchanged before meant changing the readers and writes on \emph{both} side, which was in consequence rarely done, and the stand-alone modules rather tried to live with the information they had.

When \acrshort{matsim} was re-implemented in \gls{java} around 2006--7, it was re-implemented as one system.  As a result, now everything could interact with everything.  For example, a router would modify the network, route on the modified network, and then modify it back -- including the danger of not getting things completely right in changing it back.  

What made even more problems, however, were extensions to the program flow.  The program flow was, as it still is, organized by the \lstinline$Controler$ class.  Originally, everybody who wanted to change the program flow, in particular insert his or her own research modules, would inherit from \lstinline$Controler$, override some methods, and insert his or how own instructions.  This did, however, have the consequence that it was impossible to combine the extensions without possibly massive manual interventions (see Figure~\ref{fig:do-not-extend-controler}).

\begin{figure}\footnotesize
\hrule
For example, assume the core method as
\begin{lstlisting}
class Controler {
   void run() {
      ...
      aMethod() ;
      ...
   }
   void aMethod() {
      doA() ;
      doB() ;
   }
}
\end{lstlisting}
Also assume an extension called \protect\lstinline$MyControler$
from one researcher, and another extension called \protect\lstinline$YourControler$ by another researcher:
\begin{lstlisting}
class MyControler extends Controler {
   @Override
   aMethod() {
      doA() ;
      doMyStuff() ;
      doB() ;
   }
}
\end{lstlisting}
\begin{lstlisting}
class YourControler extends Controler {
   @Override
   aMethod() {
      doA() ;
      doYourStuff() ;
      doB() ;
   }
}  
\end{lstlisting}
You can neither say \protect\lstinline$YourControler extends MyControler$ nor \protect\lstinline$MyControler extends YourControler$, since either way one of the two extensions gets lost.  In this simple case, it may be possible to address the problem by manual intervention, but in more complicated situations this is no longer possible without extensive additional testing.
\hrule
\caption{Example why extending Controler does not allow combination of extensions.}
\label{fig:do-not-extend-controler}
\end{figure}

%%%%%%%%%%%%%%%%%%%%%%%%%%%%%%%%%%%%%%%%%%%%
%%%%%%%%%%%%%%%%%%%%%%%%%%%%%%%%%%%%%%%%%%%%
% Local Variables:
% mode: latex
% mode: reftex
% mode: visual-line
% TeX-master: "main"
% comment-padding: 1
% fill-column: 9999
% End: 
