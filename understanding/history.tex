\chapter{Some History of MATSim}
\label{ch:history}
% ##################################################################################################################

\hfill \textbf{Authors:} Kai Nagel, Kay W.\ Axhausen

%\begin{center} \includegraphics[width=0.3\textwidth, angle=0]{frontmatter/figures/MATSimBook} \end{center}

% ##################################################################################################################
%\kai{Auch dieses Module nur ein Vorschlag.  Ich nehme es erstmal als Ablage für Absätze, die ich in diese Richtung geschrieben habe.  Johan war daran mal sehr interessiert.}
%
%\ah{Finde es je länger desto besser dieses Kapitel drinzuhaben und es hier drinzuhaben. Wenn wir schon über die verweneten Konzepte refektieren, dann ist auch eine Art Reflektion über das MATSim-Unternehmen an sich nötig. ... wo auch die frühere History-Einsprengsel nochmals auf einen Blick und im Fluss vorkommen}
%
%\kai{Finde ich eigentlich auch.  Nehme das mal zum Anlass, da etwas mehr zu schreiben.}

\ah{Wo wir sonst noch History drinhaben und Konsistenz garantieren müssen: Editor's Foreword, Preface, Section\ref{sec:howitstarted}, Chapter~\ref{ch:developmentprocess}}

% ##################################################################################################################
\section{Scientific Lineage of \gls{matsim}}
\label{sec:streams}
%% \section{Outside View}
%% \kai{``outside'' vs.\ ``inside'' view entspricht irgendwie nicht meiner Intuition.  Im Moment ist dies hier eher eine ``wissenschaftliche Herkunft'' (scientific lineage), während der nächste Teil eher meine eigene Historie darstellt.}
%% \ah{Wenn man den Text liest/las konnte man genau diesen Perspektivenwechsel sehen. Die Titel waren ein Schritt Richtung benötigter Verallgemeinerung. ``Scientific Lineage'' hätte auch ein externer Beobachter schreiben können, ``what Kai Nagel originally wanted'' aber eher nicht ;) Das hiesst aber überhaupt nicht, dass diese Titel besonders hilfreich/passend waren; die jetzt gewählten Titel finde ich besser! :)}

As sketched earlier 
%% \kai{wo?  Das ist doch jetzt weg, oder?}
%\ah{
(Section\ref{sec:howitstarted})
 %% (The Beginnings). Da kein Hintergrundwissen nötig.}
\gls{matsim} derives from the following research streams:

% ............................................
\paragraph{Microscopic Modeling of Traffic} 
Microscopic modeling had been 
%the 
a basis of traffic flow 
%simulation 
theory from the start 
%in the 1970’s 
\citep[e.g.,][]{theGang,Seddon_Simulation_1972,Wiedemann_PhDThesis_1974}, but the work 
%on traffic flow 
limited itself to individual links or small sequences of links and could therefore not address equilibrium as the aggregate assignment models could do from the 1970’s onward \citep[see][]{Sheffi_1985,OrtuzarWillumsen_2011}. The expansion to whole and large networks came with the increasingly more powerful computers in the 1980’s and fast and still accurate enough flow models \citep[e.g.,][]{Schwerdtfeger_VolmulerHamerslag_1984, NagelSchreckenberg1992CA, Daganzo_TransResPartB_1994,Gawron_IJMPC_1998}.
%% , but these 
%% %agent-based 
%% microscopic simulations were not used to find an \gls{equilibrium} at first. 

% ............................................
\paragraph{Computational Physics} 
For \gls{matsim}, this development was helped by insights from Computational Physics, which often adopts simple and thereby very fast models of physical processes and has thus performed simulations with $10^8$ and more particles since the 1980s \citep[for a contemporary review see][]{BeazleyEtcMolec-dyn}.
%% \gls{matsim}'s microscopic modeling of traffic as well as of demand, with $10^7$ or more ``particles'', brings it into this field of many-particle systems. %\ah{numerical analysis}
It was thus clear from the beginning that urban or regional systems with $10^7$ or $10^8$ persons or vehicles could be simulated microscopically; the research thus could focus on the question along which lines the necessary compromises would have to be made.

% ............................................
\paragraph{Microscopic Behavioral Modeling of Demand/Agent-Based Modeling} 
%% \kai{Habe im folgenden das mit "agent-based" expliziter diskutiert.  Pls chk.}
According to \citet[][p.~53]{RusselNorvig2010ArtificialIntelligence}, an \gls{agent} is ``anything that can be viewed as perceiving its environment through sensors and acting upon that environment through actuators''. In that sense, both the models of \citet{Seddon_Simulation_1972} and of \citet{Wiedemann_PhDThesis_1974} can be classified as agent-based, and this holds even for the simple cellular automata models of \citet{NagelSchreckenberg1992CA} since here the driver-vehicle units perceive the distance to the vehicle ahead, and act by adjusting their velocity.  
% Bei Schwerdtfeger oder Daganzo würde ich das NICHT mehr sehen, hier kommt die Geschwindigkeit nämlich aus einer Dichte, und das ist eine aggregierte Systemgröße und keine agenten-spezifische Perzeption mehr. kai, dec'14

Agent-based behavior can also be found at the level of demand modeling, where aggregate models such as the gravity model \citep{Wilson1971SpatialInteraction} can be replaced by person-centric formulations. In that sense, agent-based modeling of travel demand had been developed in Germany since the 1970s \citep[see the references in][]{AxhausenHerz_JTE_1989}, but then also in English speaking countries after the seminal book of \citet[][]{JonesEtAl_1983}.  
%% \kai{Da steht ``then'', aber das Datum bei Jones ist deutlich früher. ??}  
%% %
%% \kwaah{Axhausen und Herz fassen Arbeiten seit den 1970iger Jahren zusammen. 
%
%% Die englisch-sprachigen Kollegen beginnen erst in den 80iger Jahren. 
%% }
%
While the anglophone authors focused on sample enumeration methods to estimate the total demand with their 
%agent-based 
activity-based demand models (see \citet[][]{BradleyBowman_TRBTDF_2006} for the North American, mostly discrete choice model based, developments, and \citet[][]{ArentzeTimmermans_2000} for an alternative Dutch approach), the simpler German approach was linked with an integral mesoscopic traffic flow simulation in \citet[][]{Axhausen_PhDThesis_1988}, but not used for equilibrium search. It had, however, already a simple description of the total utility of the daily schedule.

%% \kai{according to Russel and Norvig, an agent is ``something that perceives and acts''.  Ich würde insbesondere ``perception'' bei fast allen der Arbeiten oben bezweifeln.  Außer vielleicht bei Wiedemann, aber dort richtet sich die Perzeption auch auf einen anderen Aspekt (nämlich driving). Darf ich durch die Absätze nochmal durchgehen und sie etwas abschwächen/differenzieren?}

%% \ah{gerne}

%% \kwaah{"perceives and acts" ist natuerlich eine Frage des Anspruchs an das Modell der Wahrnehmung. 

%% ORIENT/RV liess die Agenten within-trip umplanen, wenn sie laenger als erwartete Fahrtzeiten hatten; alle 
%% Verkehrsflussmodelle reagieren auf die lokale Umwelt. Auch Zumkeller’s MOBITOPP laesst die Agenten
%% reagieren. 

%% Ich wuerde bei dem Begriff bleiben.
%% }

% ............................................
\paragraph{Complex Adaptive Systems/Co-Evolutionary Algorithms}
Nash-equilibrium-like approaches had been developed in transport assignment since the seminal \citet[][]{Wardrop_PICE_1952} paper. These aggregate flow based approaches were expanded to account for perception errors of the user and for the social optimum \citep[see][]{DaganzoSheffi_TransScience_1977}.
%
\kwaah{In the late 1990's transport science addressed the process of learning in the context of the new possibilities of 'intelligent transport systems'. It used various smoothing techniques to integrate data from iteration to iteration reflecting the tradition of the field. Examples include \citet{%
%
ChangMahmassani_ICTB_1989,%
KaufmanEtc-91,%
HatcherMahmassani_TRR_1992, %
SmithEtc1995TRANSIMSSeattle,%
AxhausenEtAl_ETF_1995,%
Nagel1995phd,%
Nagel1996NRW,%
Gawron_IJMPC_1998,%
MahmassaniAndLiu_TransResC_1999,%
PolakOladeine_TechRep_ICL_2002,%
ArentzeTimmermans_TransResB_2004}
%
} 
%\kai{obige Referenzen (1) nach ivt.bib, (2) chronologisch sortieren.}
%\ah{Kommt noch. Wurde zwischenzeitlich unterbrochen.}
%\ah{so, nun erledigt}
%
%% but their reformulation for disaggregate agent-based solutions had to wait until the late 1990’s with (*) 
%
These approaches translated the logic of the Nash equilibrium
%genetic \glspl{algorithm} 
into co-evolutionary search schemes, which efficiently identified the optima of each agent’s daily schedule.
%
%% At the end of the 1990’s \gls{matsim} merged some of these strands into a computationally efficient tool.
%% , that is increasingly used all over the world.

% ##################################################################################################################
%\section{Kai Nagel's Perspective}
\section{Stations of Development}
\subsection{Kai Nagel's Perspective}
%% \kai{Wäre vielleicht gut, wenn wir das hinterher zusammenführen könnten.}

% -----------------------------------------------------------------------------------
\subsubsection{Fast Microscopic Modeling of Traffic Flow (University of Cologne/Los Alamos National Laboratory)}
% Dose! Auf deutsch "Dosis".  Wenn nicht gut, dann lieber ganz reformulieren.
\label{sec:history-u-of-cologne-phase}

Kai Nagel originally wanted to do his \acrshort{phd} 
% das ist ein PhD und kein ``Philosophieae Doctor''. :-) kai
%\ah{Weil es komisch klingt? ;)}
in meteorology.  When the funding did not come together, he started exploring alternatives, and applied with Prof.\ A.\ Bachem at the University of Cologne for a position in insurance modeling.  He was instead offered a position in operations research, solving problems such as dynamic vehicle routing with time windows. 

Having had some training in Computational Statistical Physics, he soon became a bit skeptical if it made sense to optimize up to the last second of a closing time window while at the same time being faced with highly stochastic transport system.  He thus embarked on using his training to build a microscopic model of the transport system, in particular single-lane~\citep{NagelSchreckenberg1992CA,Nagel1999flowTheoTrr} road traffic on long links, as well as combining such links to large-scale network-based simulations where each vehicle follows its own individual route~\citep{Nagel1996NRW}, including an adaptive dynamics being influenced most heavily by \cite{ArthurBar}.  Already that paper \citep{Nagel1996NRW} describes what is still the main \gls{matsim} architecture, where agents have many different plans, they keep trying them out, and eventually settle on the best option.  In contrast to the current approach, in that paper all plans were pre-computed, \ie there was no innovation during the iterations.  This was possible because the network was much coarser than what we use today, and thus computing route plans with enough diversity was easy.

% -----------------------------------------------------------------------------------
\subsubsection{TRANSIMS (Los Alamos National Laboratory/Santa Fe Institute)}
\label{sec:history-lanl-phase}
Some of the above \acrshort{phd} work was already done at Los Alamos National Laboratory (LANL).  After his \acrshort{phd}, Kai Nagel then fully moved to LANL, where he worked with the \gls{transims} \citep[see, e.g.,][]{SmithEtc1995TRANSIMSSeattle} team, under the leadership of Chris Barrett.
%
The \gls{transims} project inherited some of the above design, most notably the cellular automata approach to road traffic modeling, which was thus extended to multi-lane traffic \citep{NagelWolfEtAl1998TwoLaneSystematic}, to intersections \citep{NagelEtc1997flow-char}, and to massive parallel computing \citep{NagelRickert2001parallel}.

%% However, it became quickly clear that if the goal was realistic long-term traffic forecasting, demand modelling needed to correspond to what the network loading could do.  This was the time when activity-based demand modelling gained ground \citep{Axhausen1988PhD,Bowman1998PhD,KitamuraEtcSequential}, and it was thus expected that the activity-based demand would eventually come from somewhere else.  

In terms of software design, \gls{transims} was a collection of stand-alone modules, coupled by a script.  For example, the population synthesizer would generate a population file, the activity generator would take the population file as input and generate an activities file as output, etc.  Iterations were done by running 
the traffic \gls{microsimulation} (in \gls{matsim} called \gls{mobsim})
%% \ah{replace by \gls{mobsim}?}
based on plans and outputting average link travel times, and then running the router based on link travel times and outputting plans.

% -----------------------------------------------------------------------------------
\subsubsection{MATSim in \protect\gls{cpp} (ETH Zürich Computer Science)}
\label{sec:history-ethz-phase}
%\ah{somehow latex on ci.matsim.org does not like gls in section titles (?). See Build \#432. I'll try to find a work-around.}
%\ah{got it: \protect\gls{...}}

Kai Nagel moved to \gls{eth} Zürich Computer Science in 1999.  Here, it turned out to be difficult to continue with \gls{transims}, in part because \gls{transims} was not under an open source license at that time, and also because \gls{transims} fell under U.S.\ technology export restrictions for some time.  As a result, \gls{matsim} was started.

\gls{matsim} had two important differences to \gls{transims} from the beginning: (1) it tried to be more ``lightweight'', \ie running much faster, in particular by using the queue model \citep{GawronPhd,Gawron1998IterativeAlgorithmto} rather than the cellular automata model for network loading; and (2) other than \gls{transims}, agent properties such as demographic data, activity patterns or routes were no longer distributed across multiple files, but contained in one hierarchical \gls{xml} file.

Another difference followed eventually, which was to go back to the approach of \citet{Nagel1996NRW}, but this time truly following \citet{ArthurBar} by giving each individual agent its own memory \citep{RaneyNagel2006traf-framework}.  After experimentation with relational databases such as MySQL \citep{mysql-wikipedia} or Oracle \citep{oracle}, it was eventually decided to implement \gls{matsim} as an object-oriented database ``in memory'', \ie by first reading in all \gls{xml} files, modifying the data in computer memory\footnote{Random access memory (RAM)} during the life-time of a run, and writing the data back from memory to \gls{xml} files at the end of the run.  The decision was based on the observation that the \gls{matsim} data model was described much better by \gls{xml} files, and conversion to the relational format was impractical, prone to errors, and also too slow if not kept in memory during iterations. 

%========================================================================================================
\subsubsection{MATSim in \protect\gls{java} (TU Berlin transport engineering)}
\label{sec:matsim-in-java}

Michael Balmer wrote his dissertation at \gls{eth} (see below) about demand modeling for \gls{matsim}, \ie about the upstream process that leads to initial plans
\citep{Balmer_PhDThesis_2007}.  That project, other than the main \gls{matsim} code at that time, was written in \gls{java}.  Together with the assessment that \gls{java} would be the better language than \gls{cpp} to continue development%
%at TU Berlin transport engineering
, it was decided to use Michael Balmer's code as starting point for a \gls{java} version.  Arguments for \gls{java} included:
%
\begin{itemize}\styleItemize
\item  \gls{java} is more restrictive. For example, in \gls{java} objects are always passed by reference,\footnote{%
%
We abstract away from the notion that \gls{java} ``passes object references by value'' \citep{...}.
%
} while in \gls{cpp} one has the choice between passing a pointer, a reference, or a deep copy of the object.  Since in academic environments standards are difficult to enforce, a more restrictive language seemed (and still seems) the better choice.

\item \gls{java} runs well on many platforms.  This allowed (and still allows) us to let people work on their favorite platforms, be it \gls{linux}, \gls{windows}, or \gls{mac}.

\item There is good non-commercial support for \gls{java}, for example the \gls{eclipse} \gls{ide}.

\item The \gls{java} compiler is easier to handle.  For example, there is no extraction of header files, and the \gls{java} compiler by itself sorts out the sequence in which modules need to be compiled.

\item For our applications, \gls{java} consistently was \emph{not} slower than \gls{cpp}.  This assessment was based on several years of teaching a \gls{matsim} class at \gls{eth} Zürich, where computer science students implemented simple versions of \gls{matsim} in a programming language of their choice.  It typically came out that while the fastest \gls{cpp} code may have been 30\,\% faster than the fastest \gls{java} code, the slowest \gls{cpp} code typically was a factor of 3\,slower than the slowest \gls{java} code.  In other words, while \gls{cpp} gives more opportunities for optimization, it also gives more opportunities for very serious performance degradation.  The assessment is somewhat corrobated in the literature \citep{Prechelt1999EfficiencyJavaVsCpp}, where for one example it is shown that interpersonal differences within the same language are of the same magnitude as the differences between the languages.

%% Again, in an academic environment where much of the programming style is 
%% %left to the 
%% done by (\acrshort{phd}) students, it seemed (and still seems) more important to avoid important performance degradations than to go for the last 30\,\%.


In addition, it seems that the gap between \gls{cpp} and \gls{java} has narrowed further since then.  Important differences remain in numerical applications, in part also because \gls{cpp}, other than the \gls{java}, allows operator overloading.\footnote{See \url{http://en.wikipedia.org/wiki/Operator_overloading}.}  However, the agent-based approach of \gls{matsim} means that the handling of complex objects happens much more frequently than true numerical computations.

\item One reason for using \gls{cpp} was that it could be combined with \gls{mpi}, which is a reliable message passing standard for parallel computing.  Parallel computing was necessary both for performance reasons and to be able to run simulations that needed more than about 4\,\gls{gb} of memory---the maximum that could be addressed with the 32\,bit architecture that was standard at that time.  \gls{mpi} is also available for \gls{java}, but it is much less well maintained.

With the advent of the 64\,bit architectures, the second reason for parallel computing became obsolete. In addition, Kai Nagel now being at a transport engineering department, it seemed that making conceptual progress was more important than keeping the parallel computing edge, especially since the maintenance of parallel code \emph{permanently} consumes additional resources.

With the decision to give up on parallel computing, it was no longer necessary to maintain compatibility with \gls{mpi}, and thus the move to \gls{java} was facilitated.

\end{itemize}
%
In terms of language, \gls{csharp} might have been an alternative to \gls{java}.  However, \gls{csharp} depends much more on the \gls{windows} platform, and community support is not as good as it is for \gls{java}.

Clearly, the code by Michael Balmer already had all the necessary data classes, readers and writers.  This was used as a starting point to re-implement \gls{matsim} in \gls{java}.  Nevertheless, many important elements such as the \gls{mobsim}, the events architecture, scoring, routing, and the co-evolutionary architecture, had to be re-implemented.  It took about two years from taking that decision to the first plausible run of \gls{matsim} in \gls{java}.

%% In the early days of \gls{matsim}, the code was programmed in \gls{cpp} (see Chapter~\ref{ch:history}). Later, however, the code was migrated to \gls{java}. Performance of \gls{java} and \gls{cpp} had converged to a reasonable extent, such that the much higher developer friendliness of \gls{java} could be exploited. A fast access to the program is essential for \gls{matsim} as many \gls{api}-users have their background in computer science but in fields like transport planning or social sciences. 

%% Up to 2007, when \gls{matsim} was brought to \gls{sourceforge}, \gls{matsim} was provided by a download website and \gls{svn} repository hosted on a VSP server.

%% \gls{matsim} development furthermore greatly benefits from the possibility to run large-scale scenarios. Therefore, several servers with up to 512\,\gls{gb} of \gls{ram} and 40\,\gls{cpu} cores are available.  
%
% part of that argument is included further above (from 32 to 64 bit mem architecture)


% ========================================================================================================
\subsubsection{Code Reorganization}
\label{sec:matsim-core-reorg}
The \gls{cpp} version of \gls{matsim} was, similar to the original \gls{transims}, a collection of stand-alone executables coupled by scripts.  For example, the router would read plans and events, and replace some of the plans by other plans with modified routes.  The program flow was organized with shell scripts and makefiles.  Later it was possible to start all modules simultaneously and they would use messages to interact \citep[also see][]{GloorNagel2005ped-att04-birkh}, but the file-based and scripted interaction always remained available.

That approach had, as a consequence, very clearly defined interfaces, \ie the files. Exchanging information that was not included in the files meant changing the readers and writers on \emph{both} sides, which was in consequence rarely done, and the stand-alone modules rather tried to live with the information they had.

When \gls{matsim} was re-implemented in \gls{java} around 2006--7, it was re-implemented as one system.  As a result, now everything could interact with everything.  For example, a router could modify the network, compute routes on the modified network, and then modify it back. Clearly, it could make an error in the process, thus erroneously modifying the network.  In this way, any module could modify any data of \gls{matsim}, greatly increasing the scope for misunderstandings and errors.

What made even more problems, however, were extensions to the program flow.  The program flow was, as it still is, organized by the \lstinline$Controler$ class.  Originally, everybody who wanted to change the program flow, in particular insert his or her own research \glspl{module}, would inherit from \lstinline$Controler$, override some methods, and insert his or how own instructions.  This did, however, have the consequence that it was impossible to combine the \glspl{extension} without possibly massive manual interventions, illustrated as follows.
% (see Figure~\ref{fig:do-not-extend-controler}). \ah{reasons for this being a figure?}

%\begin{figure}\footnotesize
%\hrule
%\strut For example, assume the core method as
For example, assume the core program flow as
\begin{lstlisting}
class Controler {
   void run() {
      ...
      aMethod() ;
      ...
   }
   void aMethod() {
      doA() ;
      doB() ;
   }
}
\end{lstlisting}
Also assume an extension called \protect\lstinline$MyControler$
from one researcher, and another extension called \protect\lstinline$YourControler$ by another researcher:
\begin{lstlisting}
class MyControler extends Controler {
   @Override
   aMethod() {
      doA() ;
      doMyStuff() ;
      doB() ;
   }
}
\end{lstlisting}
\begin{lstlisting}
class YourControler extends Controler {
   @Override
   aMethod() {
      doA() ;
      doYourStuff() ;
      doB() ;
   }
}  
\end{lstlisting}
%
If you wanted to combine both approaches, you could neither say \protect\lstinline$YourControler extends MyControler$ nor \protect\lstinline$MyControler extends YourControler$, since either way one of the two extensions gets lost.  In this simple case, it may be possible to address the problem by manual intervention, but in more complicated situations this is no longer possible without extensive additional testing. %\strut
%\hrule
%\caption{Example why extending \lstinline|Controler| does not allow combination of extensions.}
%\label{fig:do-not-extend-controler}
%\end{figure}

Therefore, in 2008 a decision was made to make \gls{matsim} more modular.  The first step in that direction was a decision to submit the whole \gls{matsim} repository to frequent refactorings, \ie to \emph{not} leave the code alone as much as possible but rather make the community get used to frequent changes of code while maintaining functionality.  In order to facilitate that approach, the coverage by automatic regression tests on the build server was hugely increased, and all developers were encouraged to write automatic regression tests for their own code and their own projects. 

The changes since then are too numerous to be listed here.  They include, in particular, fairly restrictive data classes that are no longer extended or modified by every scientific project, and well-defined extension points both in the iterative loop and inside the \gls{mobsim}.  Please see Chapter~\ref{ch:extensionpoints} for the currently existing extension points.

%% Since the huge refactoring greatly improving the code base structure and establishing, the \gls{matsim} project code is divided into the \emph{core} \lstinline|org.matsim| and \gls{api} packages \lstinline|org.matsim.api| and remaining packages. This refactoring was undertaken by the Berlin group starting at the end of 2007 and ended in the beginning of 2009 before the very first user meeting \ah{stimmt das so?}.

%\ah{And here enters the second author of MATSim the scene ...}
% ========================================================================================================
\subsection{Kay~W.~Axhausen's Perspective}
\subsubsection{ORIENT/RV: Parking in Travel Demand Models (Karlsruhe University)}
Kay Axhausen returned in 1984 to Karlsruhe University%
\footnote{
Now: Karlsruhe Institute of Technology (KIT)
}
after two years for an MSc at the University of Wisconsin to start his \acrshort{phd} at the \gls{ifv}. At that time the \gls{ifv} already had a long tradition of traffic flow analysis \citep[][]{Leutzbach1972Buch} and of its agent-based simulation as pioneered by \citet[][]{Wiedemann_PhDThesis_1974} \citep[see also][]{LeutzbachWiedemann_TEC_1986}. In this environment \citet[][]{Sparmann_TechRep_1980} had implemented a sample enumeration based simulation of traffic demand in the spirit of \citet[][]{PoeckZumkeller_PTRC_1978}. This approach had taken the daily schedule of the traveler and simulated it activity-by-activity including the necessary travel. Neither the traffic flow or travel demand simulations aimed for equilibrium, but in line with discussion at the time both were more interested in the underlying behaviors \citep[e.g.,][]{JonesEtAl_1983}.

Faced with a project to simulate parking as an extension of Sparmann's ORIENT approach, it became clear to him that sample enumeration approaches cannot account for the temporal and spatial competition for parking spaces, but that the event-oriented approaches of the traffic flow model naturally can. 
Merging the two approaches was the natural solution, as he then designed it for ORIENT/RV \citep[][]{Axhausen_TechRep_IFV_1989}. 
Given the need to model the flow of traffic on the roads as part of the daily dynamics the approach of Schwerdtfeger, an \gls{ifv} colleague, was a natural and computationally-efficient choice. 
\citet[][]{Schwerdtfeger_VolmulerHamerslag_1984} had developed a meso-scopic simulation of traffic flow, which retained the agent-resolution, but employed macroscopic link-performance functions to calculate the speeds on the links. 

The work of \citet[][]{Swiderski_CarpenterJones_1983}, a second \gls{ifv} colleague, had alerted him to the need to account for the constraints imposed by the mental maps of travelers. As a full implementation of a mental map is impossible even with today's computers he chose to condition the route choices of the travelers on their travel time expectations, which were based on shortest-paths over an empty network initially. The agents reconsidered their routes at every junction, if the experienced travel time deviated beyond an adaptive threshold from the expected travel times. 
%
%% \st{This architecture is mirrored in MATSim's architecture of modeling competition  on the network and for facilities} \textcolor{gray}{\tiny \citep[][]{HorniEtAl_TRR_2009}} \st{and then replanning before the next iteration.} 
%% \kai{Sehe ohne weitere Erläuterungen im Moment nicht die Beziehung zwischen HorniEtAl und entweder ``constraints imposed by mental maps'' oder ``reconsidered their routes at every junction''.  Vorschlag: streichen} 
%% \kwaah{Würde nur den letzten Teil streichen. Das Zitat Horni ist die Arbeit zur Wahl des Einkaufsorts und damit das erste Beispiel des Wettbewerbs um 'facilities'} 
The framework was used to iterate \citep[][]{Axhausen_Jones_1990} the expectations via shortest-paths based on stored mean travel times from the last iteration, but no formal tests of equilibrium were conducted, nor was the number of iterations extensive. 
%% \st{(See} \textcolor{gray}{\tiny \citet[][]{Dobler_PhDThesis_2013}} \st{that such an approach can approximate the equilibrium travel times with rather few iterations.)}
%% \kai{Ich sehe die Verbindung, würde aber, wenn der Pointer gewünscht wird, hier eine Formulierung vorziehen wie ``This was taken up again by Dobler ...''.}
%% \kwaah{
%% This was taken up in 
%%   \citet[][]{Dobler_PhDThesis_2013}, where he showed the such an approach can approximate the equilibrium in a small number of iterations.
%}

\kai{@Andreas: Habe die beiden KWA-Kommentare (auskommentiert) mal rausgezogen wie folgt.  Das erscheint mir eigentlich einfacher, als sie in obigem Text unterzubringen, zumal die Referenz auf Horni am Ende von ``within-day replanning'' kommt und nicht hinter ``competition for facilities''.  Weiß nicht, ob wir das abstimmen müssen.}
In the \gls{matsim} context, the competition for facilities for taken up by \citep[][]{HorniEtAl_TRR_2009}.  Reconsidering routing decisions while already being en-route was taken up by \citet[][]{Dobler_PhDThesis_2013}, where he showed the such an approach can approximate the equilibrium in a small number of iterations.

% -----------------------------------------------------------------------------------
\subsubsection{From EUROTOPP to \gls{matsim} (Karlsruhe, Oxford, London, Innsbruck, Zürich)}

%% \kai{section slightly re-written, pls chk}

The  first framework program of the European Union gave the chance to continue with the work in a larger context, but unfortunately this extended version of ORIENT/RV never went beyond the design stage \citep[][]{AxhausenGoodwin1991}. The EUROTOPP approach was later implemented in a changed form at the \gls{ifv} again by Zumkeller and his students, who also had been one of the partners of the first framework project \citep[][]{SchnittgerZumkeller_ETC_2004}.

%% \ah{Merging this somewhere below ...}
%\kwaah{
Moving to Oxford, London, Innsbruck and then Zürich in relatively short order kept him from initiating serious work on a \gls{largescale} simulation system. 
The focus switched to data collection and choice modeling. 
The collaboration on travel demand simulation with Kai Nagel began when Kai Nagel also joined \gls{eth} in 1999. 
While this was low key at the beginning, the move of Michael Balmer and David Charypar to Kay Axhausen's group after Kai Nagel's departure to TU Berlin jump-started the further work, which is now documented in %the rest of 
this book.
%% }

% -----------------------------------------------------------------------------------
%% \ah{Following needs to be merged into ``Joint Work'' but with a specialized focus. ``Focus Berlin''/''Focus Zürich''?}

\subsubsection{``Best Response'' and Further Choice Dimensions (ETH Zürich transport engineering)}
\label{sec:zhgroup_matsim}

%% \kai{Die Logik hier ist jetzt in etwa: Es gibt eine Liste, siehe Figure~\ref{fig:dimensions}, und die bildet den roten Faden für die Arbeiten am IVT.  ``History'' ist insofern alles, was von dieser Liste bereits bearbeitet ist.  Ok so?  Sonst bitte anpassen, wobei ich gerne ``history'' und ``future'' ($=$ research avenues) auseinander halten würde.}

%\ah{maybe change to bullets}
% ----------
\createfigure[!t!]%
{Behavioral dimensions to be included in a fuller scheduling model}%
{Behavioral dimensions to be included in a fuller scheduling model}%
{\label{fig:dimensions}}%
{
\begin{minipage}[l]{0.9\textwidth}
Number and type of activities \\
Sequence of activities
\begin{itemize}[label=$\bullet$]
\item {\color{red}Start and duration of activity}
\item Composition of the group undertaking the activity
\item Expenditure division
\item Location of the activity
\begin{itemize}[label=$\bullet$]
	\vspace{5pt}
	\item Movement between the sequential locations
	\vspace{5pt}
			\begin{itemize}[label=$\bullet$]
				\item Location of access and egress from the mean of transport
				\item Parking type
				\vspace{5pt}
				\item {\color{red}Vehicle/means of transport}
				\item {\color{red}Route/service}
				\item Group traveling
				\item Expenditure division
			\end{itemize}
	\end{itemize}
\end{itemize}
\end{minipage}
}%
{\citet[][]{Axhausen_unpub_NYC_2014, Axhausen_SSRL_2006, Axhausen_unpub_TRB_2009} 
%\kai{was ist rot und was ist schwarz?} 
%\ah{Ich vermute: rot = erledigt. Müsste man wohl noch 1-2 Punkte rot färben.}
\kai{@Andreas: Bin etwas skeptisch bzgl.\ ``vehicle''.  Wo finde ich das im core?}
}
% ----------

%% While \gls{matsim} was designed from early on as a system where synthetic travellers could maintain multiple plans \citep{RaneyNagel2004agdb}, choice dimensions beyond route choice were investigated and added one by one: departure time fairly early \citep{BalmerRaneyEtAl2005act-times}, mode choice a bit later \citep{RieserGretherNagel2008modeChoiceCalculations}.

\textbf{Departure time, mode, and route choice} are at the core of the transport modeling enterprise and were addressed in \gls{matsim} from the start \citep{RaneyNagel2004agdb} or early on \citep{BalmerRaneyEtAl2005act-times,RieserGretherNagel2008modeChoiceCalculations}.
%
The direction of the work in Zürich aimed to address further behavioral dimensions 
%beyond the core of mode and route choice (See 
as shown in Figure~\ref{fig:dimensions}.
%\kwaah{The figure should go to the research avenues or the introduction maybe}\kai{Finde ich nicht zwangsläufig, s.o.}). 
%% While destination, departure-time, mode and route choice are the core of the transport modeling enterprise and were addressed in \gls{matsim} from the start in one-way or another,
\kwaah{Red indicates dimensions which are handled in core \gls{matsim} today, while the others are subject to ongoing research or past studies, which did not produce stable enough code for general use.}
It is clear that there are more dimensions to consider, and those listed in the figure are only the more prominent ones, as for example the class of service on a rail journey is ignored or the intensity of activity engagement. 

\gls{matsim} at present takes the activity chain, the schedule, as given from the initial demand generation process, while modern ``activity based-models'' make it sensitive to accessibility, understood as the logsum term of the included destination and mode choice model (route choice is generally excluded in those models) \citep[see][for an early example]{BenAkivaEtAl_Transportation_1996}.
% (See Ben-Akiva, Bowman and Gopinath, 1996 for the first example). 
The computational overhead costs of calculating the set of non-chosen alternatives becomes prohibitive at the scale for which \gls{matsim} is designed,
%% \kai{Kann man das so sagen, wenn CEMDAP oder TASHA das inzwischen hinbekommen?}
%% \kwaah{Die Modelle lösenm das, in dem sie nicht den ganzen Baum erzeugen, sondern partielle Bäume (Teilmodelle) nutzen.
%% TASHA ist als heuristisches Modell anders aufgestellt.}
 so alternative approaches were explored. 
%
Meister developed a \textbf{genetic algorithm on a household-basis to find optimal schedules for all members simultaneously} \citep[reported in][]{MeisterEtAl_Transportation_2005},
% (reported in Meister, Frick and Axhausen, 2005), 
but only its time-of-day choice element saw use in later scenarios \citep{MeisterEtAl_IATBR_2006}.
%% (Meister, Balmer, Axhausen and Nagel, 2006). 
%
Feil set himself the task to find a \textbf{best-response, but computationally fast approach to the optimization of the number and sequence of activities into a schedule} \citep[][]{Feil_PhDThesis_2010}. While he made substantial progress using a tabu search and a cloning approach, it is still too slow as it currently stands. Fourie's PSim (see Chapter~\ref{ch:psim}) might remove that constraint.

While the approaches of Meister and of Feil, as well as the standard \gls{matsim} routing algorithm, attempt to directly provide best response solutions, also the standard \gls{matsim} evolutionary algorithm moves in the direction of good or best response (also see Section~\ref{sec:stat-weight-of-plan}).  With these approaches, it is not possible to directly model \textbf{destination choice}, since the best response destination would just be the closest possible destination, in consequence under-estimating travel distances \citep{HorniEtAl_TRR_2009}.  The problem is that destinations which are similar from the analyst's point of view are quite different from each person's point of view, for example allowing different types of leisure activity.  As further explained in Chapter~\ref{ch:destinationchoice}, the problem was addressed by attaching randomness directly to each person-alternative-pair \citep[also see][]{HorniEtAl_TRB_2012}.


%% \ah{destination choice (Chapter~\ref{ch:destinationchoice}).}

%% As best-response approaches these two are at odds with the general co-evolution approach of \gls{matsim}. One would have to show with extensive tests that these comprehensive optimizations contribute to a faster convergence of the overall population then more limited random searches, \eg local switches in order, or local additions or subtractions of activities. Still, as Nagel (xxxx) \kwaah{Das Argument, dass die ln Funktion mit dem induzierten Verkehr Schwierigkeiten macht taucht ja früh auf. Was sollte hier zitiert werden?}\kai{Ich sollte solche Sachen wohl (auch) aufschreiben, bin mir aber ziemlich sicher, dass ich das hier niemals gemacht habe.} and \citet{Feil_PhDThesis_2010} point out (see also Section~\ref{sec:future-of-scoring-function}), the basic \gls{matsim} scoring function with its log-transformation of activity duration cannot cope with addition and subtraction of activities, as it favors a solution with many shorter activities. Given the policy interest of induced demand effects, a solution will need to be found in further work. 
%
%habe obigen Absatz auskommentiert, m.E.\ research avenues. kai, apr'15

The need to address \textbf{parking} is obvious and even more obvious when combined with the consideration of electric vehicles and their current necessity to be recharged during the course of a day. 
Waraich addressed both aspects by 
%finally 
integrating a local search into the overall \gls{matsim} iteration scheme to identify the preferred parking space in the vicinity of the final destination (Chapter~\ref{ch:parking}). 
Dobler's approach \citep[][]{Dobler_PhDThesis_2013} to evacuation is similar, but 
%for obvious reasons 
does not iterate since that is not relevant for the modeling of evacuation.
%In this local search he is similar to Horni's destination choice approach (Chapter~\ref{ch:destinationchoice}) and it can be extended with personalized values of walking time. \ah{dc != lcoal search} 
His local search can be extended with personalized values of walking time.

The \textbf{group composition for joint travel and joint activities} is crucial to make progress on a number of fronts, but especially to understand destination choice and activity generation. 
%Gliebe and Koppelman (2005), 
\citet{GliebeKoppelman_Transportation_2005} or
%Zhang, Timmermans and Borgers (2005), 
\citet{ZhangEtAl_TransResB_2005},
for example, have proposed discrete choice models for household activity allocation. However, these approaches cannot be integrated easily into \gls{matsim} because of their computational costs. They are also too restrictive with their exclusive focus on the household.  Based on parallel empirical work on social networks 
%(see Larsen, Axhausen and Urry, 2006; Kowald et al., 2013) 
\citep[see][]{LarsenEtAl_MOB_2006,KowaldEtAl_JTG_2013},
Dubernet is currently exploring new game theoretic approaches to co-ordinate the timings and activities of households and wider social networks.  These social networks, in turn, are constructed using the approach of 
%Arentze, Kowald and Axhausen (2013), 
\citet{ArentzeEtAl_SN_2013},
as earlier simulation-based approaches to network formation were found to be too computationally expensive \citep[e.g.,][]{Hackney_PhDThesis_2009}.

The question of \textbf{expenditure division} is a promising research avenue (Section~\ref{sec:ra-choice-dimensions} untouched by transport planning up to now, and obviously interacting with joint activity participation and travel. 
%\kai{Ist das ``history'' oder eher ``research avenues''?  Letztendlich aber Eure Entscheidung.}
%\kwaah{Ich denke, dass wir das in 'Research Avenues' wieder aufgreifen können, aber so haben wir alle Zeilen der Abbildung besprochen.}
%\ah{adapted accordingly}

% ##################################################################################################################
%%%%%%%%%%%%%%%%%%%%%%%%%%%%%%%%%%%%%%%%%%%%
%%%%%%%%%%%%%%%%%%%%%%%%%%%%%%%%%%%%%%%%%%%%
% Local Variables:
% mode: latex
% mode: reftex
% mode: visual-line
% TeX-master: "../main"
% comment-padding: 1
% fill-column: 9999
% End: 
