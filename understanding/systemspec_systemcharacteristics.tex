\chapter{System Characteristics and Relation to Other Transport Modeling Approaches}
\label{ch:systemspec}
% ##################################################################################################################

\hfill \textbf{Author:} Andreas Horni

\begin{center} \includegraphics[width=0.3\textwidth, angle=0]{figures/MATSimBook.png} \end{center}

% ##################################################################################################################
\section{Best Response Versus Random Choice}
\label{sec:bestResponseOrRandom}
Kai hatte das im Conceptual Meeting angesprochen -> innovative Module müssen keine exakten Lösungen vorschlagen. 
Aber: etwas anders dürfte es bei best response sein!


Methodische Diskussion:
route choice -> random walk führt zu nichts
man könnte best response plus Fehlerterm versuchen

Routing in Hypernetzen?
Wie kann man Suchraum begrenzen? -> siehe auch equilibration bei Zielwahl.


\ah{RandomizedRouter (Gauteng) -> siehe Mail von A. Neumann}


% ##################################################################################################################
\section{Scale and Interfaces}
-> siehe Workshop-Paper Nagel and Axhausen

% ========================================================================================
\subsection{Landuse}

% ========================================================================================
\subsection{Cellular Automaton}

% ##################################################################################################################
\section{Emergence and Volume-Delay Function}
Netzweite Volume-Delay Function

Webster
Acelik
Wu and Brillon (nicht-signalisierte Knoten)
Kimber Hollis
Geroliminis and Zheng (STRC) -> 3-dimensional MFD
Peter Fourie für Singapur

% ----
Marcel: Interface für Geschwindigkeitsbestimmung wird reingehängt werden.
-> allenfalls $\epsilon$ draufschmeissen.

% ----
MFD:
Geroliminis and Daganzo 2008
MFD for complete city
Yokohama; taxis

% ----
Fourie
aggregate analysis (MFD) of district
-> derive max. number of drivers
Hysterese

-> MFD konnte für frühe Iterationen schon reproduziert werden, für Enditeration wurde hinterer Teil von MFD wegoptimiert. Etwas ähnliches hat David beobachtet (?)

% ----
MFD: siehe auch Simionis!

% ##################################################################################################################
\section{Variability}
\label{sec:variability}

siehe Horni Papers

and \citet[][]{Neumann_PhDThesis_2014}: AN: ``Sensitivitätsruns (nenne das ``ensemble runs'') zur Abschätzung des Signalrauschens, d.h. z.B. 10 Runs mit gleichem Setup aber verschiedenen Random Seeds. siehe z.B. Abbildungen 7.15/7.16 und Tab 7.2 der Diss (Seiten 133-135)''

% ##################################################################################################################
% ##################################################################################################################
a little theory hier unterbringen



\section{The 4-Step Procedure and Aggregate Forecasting Methods}
blablabla blablabla blablabla

blablabla blablabla

blablabla

% ##################################################################################################################
\section{Assignment Methods: Successors of Beckmann et al.}

% ##################################################################################################################
\section{Analytical and Other Numerical Methods (Variational Inequalities, Fixed Points)}

% ##################################################################################################################
\section{Other Microsimulation Frameworks}
\subsection{Rule-Based vs. Equilibrium-Based Frameworks}

(agent-based) microsims

% ##################################################################################################################
\section{Microsims in the Context of Different Types of Equilibria}
\label{sec:types}
% ------------------------------------------------------------------
\copied{\citet[][]{Horni_PhDThesis_2013}}

% ------------------------------------------------------------------
\textbf{Deterministic User Equilibrium (UE):}

% ------------------------------------------------------------------
\textbf{Stochastic User Equilibrium (SUE):}
 
% ------------------------------------------------------------------
\textbf{Dynamic User Equilibrium (DUE):}

% ------------------------------------------------------------------
\subsection{Existence, Uniqueness, Stability and Behavioral Basis of Equilibria}
For microsimulations, very little is known about the targeted equilibria. These models are highly dynamic, stochastic and disaggregate with many user classes and behaviorally rich decision principles. In Section \ref{sec:MATSimEquilibrium}, the MATSim equilibrium is discussed. As, in general, for the interpretation of results but also for model development (see equilibration discussion in Section \ref{sec:equilibration}), knowledge about the characteristics of the equilibrium searched is helpful, further research is strongly suggested.


\todo{
Gunnar \\
Daniel R. Leeds
}