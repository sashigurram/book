\chapter{Organization: Development Process, Code Structure \& Contributing to MATSim}
\label{ch:developmentprocess}
% ##################################################################################################################

%% \kaitodo[inline]{read chapter/section}

\hfill \textbf{Authors:} Marcel Rieser, Andreas Horni, Kai Nagel

\begin{center} \includegraphics[width=0.5\textwidth, angle=0]{extending/figures/ConceptualMeetingVillaHatt.jpg} \end{center}

% ##################################################################################################################
This chapter describes how new functionality enters \gls{matsim}. It describes the \gls{matsim} team and community, the different roles existing in the \gls{matsim} project, development drivers and processes, and tools used for integration. The goal is to 
%give you 
provide an overview of the development process such that 
%you 
one quickly finds access to the \gls{matsim} community and such that 
%you are 
one is able to efficiently contribute to \gls{matsim} according to one role or another, if you commendably like to.

% ##################################################################################################################
\section{MATSim's Team, Core Developers Group, and Community}
The \imp{\gls{matsim} team} 
%the heart of the \gls{matsim} community, 
currently consists of three research groups and a spin-off company, namely 
\begin{itemize}\styleItemize
\item the \gls{vsp} group at the \gls{ils}, \gls{tu} Berlin, led by Prof.~Dr.~Kai Nagel,
\item the \gls{vpl} group at the \gls{ivt}, \gls{eth} Zürich, led by Prof.~Dr.~Kay~W.~Axhausen, 
\item the recently founded Mobility and Transportation Planning group at the \gls{fcl}, based in Singapore and led by Prof.~Dr.~Kay~W.~Axhausen, and 
\item \gls{senozon}, based at Zürich with a subsidiary in Germany, founded by former \acrshort{phd} and research students. 
\end{itemize}

As common in research, the university groups' composition changes frequently. Over the last decade more than 50\,people, as listed earlier, contributed to \gls{matsim}.

A small group of the \gls{matsim} team defines the \imp{\gls{matsim} core developers group}, maintaining \gls{matsim}'s core as defined below in Section~\ref{sec:extending-core}.

Additionally, there is a \imp{\gls{matsim} community} composed of closely connected research groups (such as Stockholm, Pretoria, Poznan, and Jülich),
% and Toronto) 
%\kai{habe Toronto hier rausgenommen.  Spricht etwas dagegen?}
%\ah{nein. Danke! }
and more loosely connected external users coming together, \eg at the annual \gls{matsim} user meeting (see Figure~\ref{fig:team}).   

\gls{matsim} is open-source software under the \gls{gplv2}, thus, you are also very welcome to contribute to the code base as described in Section~\ref{sec:yourcontribution}. New contributors are mentored in the beginning 
%\citep[][]{MATSIM-BecomingAContributor_Webpage_2015}  \ah{unavailable}
to become familiar with the project and the coding conventions. 
%\citep[][]{MATSIM-CodingGuide_Webpage_2015}. \ah{unavailable}

%Picture~\ref{fig:team} shows a collage of MATSim events with the MATSim community members as participants.
%
% ------------
\createfigure%
{MATSim events and community}%
{MATSim events and community}%
{\label{fig:team}}%
{\includegraphics[width=0.99\textwidth, angle=0]{extending/figures/MATSimCommunity}}%
{\copyright Dr.~Marcel Rieser, \gls{senozon}}
% ------------

% ##################################################################################################################
\section{Roles in the MATSim Community}
\label{sec:roles}
There are the following roles in the \gls{matsim} community.
%
\begin{itemize}\styleItemize
\item The \imp{\gls{matsim} user} uses the official releases or nightly builds and runs the \gls{matsim} core with the \gls{configfile} (Section~\ref{sec:using-core-only}). He or she does not write computer code. Part~I of the book is dedicated to the \gls{matsim} user. On the web page he or she finds relevant information in the \emph{user's guide} section and in the user's mailing list \lstinline|users@matsim.org|.\footnote{%
%
During the writing of this book, the information that had so far been contained in the user's guide was moved to this book.  In consequence, the user's guide section on the web page is currently essentially empty, and may be removed.
%
} There is also a list of questions and answers under \url{http://matsim.org/faq}.

Users should also remember to consult the files \lstinline{logfileWarningsErrors.log} and \lstinline{output_config.xml.gz}, as also explained in Section~\ref{sec:survival}.  The former file is an extract from \lstinline{logfile.log}, but only contains the warnings and errors.  The latter is a complete dump of the currently available configuration options, including comments to most options.
%
\item The \imp{\gls{matsim} power user} is a \gls{matsim} user with knowledge on how to use the additional modules presented in the book's part~II. He or she does \emph{not} program but knows how to use \gls{matsim} scripts-in-\gls{java} prepared by others or her/himself, as shown in Section~\ref{sec:writing-scripts-java}. Part~I and~II of the book are helpful to the \gls{matsim} power user.  Information about \glspl{extension} can be found under \url{http://matsim.org/extensions}.  Most \glspl{extension} come with an example script-in-\gls{java}.  Again, questions and answers are under \url{http://matsim.org/faq}.
%
\item The \imp{\gls{matsim} 
%\gls{api} 
developer} extends \gls{matsim} by programming against the \gls{matsim} \gls{api} (Section~\ref{sec:writing-your-own-extensions}). He or she also finds his or her information in Part~II of the book, in particular in Chapter~\ref{ch:extensionpoints}, on the web page in 
%the \emph{\gls{api} user's guide} and 
% does no longer exist.  kai, aug'15
the \emph{developer's guide}, and in the mailing list \lstinline|developers@matsim.org|.
%
%\item The \imp{\gls{matsim} \gls{api} developer} also programs against the \gls{matsim} \gls{api}, but additionally he or she is part of the \gls{matsim} code base by having his or her own playground or \gls{contribution} being part of the refactorings. 
%\ah{contribs auch, oder?} \kai{ja}. 
%The \gls{api} developer finds information in Part~II of this book, on the web page in the \emph{developer's guide} and in the mailing list \lstinline|developers@matsim.org| 
%
\item There are relatively few \textbf{\gls{matsim} core developers} in the \gls{matsim} team. 
%This role is an extension of the \gls{matsim} \gls{api} developer and 
These 
%guys 
persons make necessary modifications of the core (as defined in Section~\ref{sec:extending-core}), usually after having discussed them in the issue tracker (\url{http://matsim.org/issuetracker}), in the \gls{matsim} committee, or at a developer meeting (see below). 
%% Part~II and part~III of this book might be particularly interesting for them.
%
% \kn{Sehe nicht, warum ausgerechnet die core developers neben dem software engineering auch noch die mathematisch-konzeptionelle Arbeit übernehmen müssen. --> weglassen}
% \ah{Dachte eher an Hintergrund: Damit man auch weiss, was man in einem weiteren Sinn eigentlich tut. Gibt aber verschiedene Blickwinkel drauf ... und deinen finde ich mittlerweile viel naheliegender.}
\end{itemize}
%
% ##################################################################################################################
\section{Code Base}
\label{sec:core-contribs-playgrounds}
The various pieces of \gls{matsim} are delineated by \gls{maven} projects and sub-projects. 
The \gls{maven} layout corresponds to the layout of the \gls{git} repository.\footnote{
\gls{matsim} is currently at \gls{github} under \url{https://github.com/matsim-org/matsim}. 
The exact path name may change in the future, \eg because of changes at \gls{github}.
}
Note that the \gls{java} package structure does \emph{not} directly correspond to the \gls{maven}/\gls{git} layout.

%---------------------------------------------------------------
\subsection{Main Distribution}
\label{sec:extending-main}
The ``\gls{matsim} main distribution'' corresponds to the ``matsim'' part of the \gls{git} repository.
%% It is maintained by the \gls{matsim} (core) team.
It is the part of the code that the \gls{matsim} team primarily feels responsible for.
%
At the time of writing, the \gls{matsim} main distribution contains following packages:
\begin{itemize}\styleItemize
\item \lstinline{org.matsim.analysis.*}, containing certain analysis packages that are added by default to every \gls{matsim} run.
\item \lstinline{org.matsim.api.*}, see Section \ref{sec:extending-core}
\item \lstinline{org.matsim.core.*}, see Section \ref{sec:extending-core}
%\footnote{Please note, that at the moment, the core is \emph{not} restricted to this large package as one might naturally assumes. \kai{chk.  Ich hatte das immer anders gesehen: core nur org.matsim.core.* und org.matsim.api.*.}}
\item \lstinline{org.matsim.counts.*}, see Section~\ref{sec:extending-counts}
\item \lstinline{org.matsim.facilities.*}, see Section~\ref{sec:extending-facilities}
\item \lstinline{org.matsim.households.*}, see Section~\ref{sec:extending-households}
\item \lstinline{org.matsim.jaxb.*}, containing automatically or semi-automatically generated adapter classes to read \gls{xml} files using \gls{jaxb} 
\item \lstinline{org.matsim.lanes.*}, see Chapter~\ref{ch:signalslanes}
\item \lstinline{org.matsim.matrices.*}, containing  (somewhat ancient) helper classes to deal with matrices, in particular origin-destination-matrices
\item \lstinline{org.matsim.population.*}, mostly containing a collection of algorithms that go through the population and modify persons or plans
\item \lstinline{org.matsim.pt.*}, see Chapter~\ref{ch:pt}
\item \lstinline{org.matsim.run}, see Section \ref{sec:extending-core}
%% \item \lstinline{org.matsim.signalsystems.*}, see Chapter~\ref{ch:signalslanes}
% now a contrib. kai, sep'15
\item \lstinline{org.matsim.utils.*} containing various utilities such as the much-used ObjectAttributes
% deliberately not lstinline(ObjectAttributes} in order to keep the lstings font for the package names here.  kai, mar'15
(see Section~\ref{sec:objectAttributes-and-customizable})
\item \lstinline{org.matsim.vehicles.*}, see Section~\ref{sec:extending-vehicles}
\item \lstinline{org.matsim.vis.*}, containing helper classes to write \gls{matsim} information, in particular from the \gls{mobsim}, to file.  This has to a large extend been superseded by the \gls{via} visualization package (see Chapter~\ref{ch:via})
\item \lstinline{org.matsim.visum.*}, containing code to input data from \gls{visum}
\item \lstinline{org.matsim.withinday.*}, see Chapter~\ref{ch:withinday}
\item \lstinline{tutorial.*}, containing example code of how to use \gls{matsim}, referenced throughout this book
\end{itemize}

%---------------------------------------------------------------
\subsection{Core}
\label{sec:extending-core}

The so-called core is part of the main distribution (see the previous Section~\ref{sec:extending-main}).  In contains material that is considered basic and indispensable, and resides in the
%three % nein, im Java Sinne sind die sub-packages eigene packages. kai, dec'15
packages
%Most of the core material is in packages that start with 
\begin{itemize}\styleItemize
\item \lstinline{org.matsim.api.*}
\item \lstinline{org.matsim.core.*}
\item \lstinline{org.matsim.run.*}
%% \item \lstinline{tutorial.*}
\end{itemize}
%
The \gls{matsim} core is maintained by the \gls{matsim} core developers group.

%\kai{Andreas, ich würde die tutorial package nicht zum core rechnen.  Es ist zwar richtig, dass das core team sich weitgehend darum kümmert.  Aber erstens hätte ich nichts dagegen, wenn das andere Leute übernehmen würden.  Und zweitens könnte man dann auch behaupten, dass jira zum Core gehört.}
%\ah{Ok. Hatte das, glaube ich, irgendwoher kopiert.}
 
%There is additional material in that part of the repository, for example under \lstinline{org.matsim.withinday.*} or \lstinline{org.matsim.pt.*}, that could as well be moved into the ``contrib'' section, but for the time being has remained in the core \ah{see below}, mostly because the maintainers of these parts have moved to \gls{senozon} and from there continue to maintain these functionalities.

%\ah{we should make the difference more clearer between \lstinline{org.matsim.core.*} and \lstinline{org.matsim.*}. 
% I have the impression that we use the word ``core'' interchangeably, and even if not, then having \lstinline{org.matsim.core.*} inside ``core (?) MATSim'' is confusing.}
%\ah{see now below and issues \url{https://matsim.atlassian.net/browse/MATSIM-327} and \url{https://matsim.atlassian.net/browse/MATSIM-323}}

%---------------------------------------------------------------
\subsection{Contributions}

%% \kai{@Andreas: M.E.\ hast Du die package names gar nicht geprüft.}
%% \ah{? Sind keine package names. Genau darum steht ja kein Pfad dran. Habe die contrib projects abgetippt, irgendwo musst ich mich ja festhalten.}
%% \kai{Erstens ist es ``contrib'' und nicht ``contribs''.}
%% \ah{? Titel? Ist kein Package Pfad!}
%% \kai{Zweitens folgen manche gar nicht dieser Konvention, z.B.\ gtfs2matsimtransitschedule.  Drittens ist es bei manchen ``garbled'', z.B.\ fehlt bei roadpricing das ``contrib''.  Viertens ist die Groß-/kleinschreibung der contrib nicht identisch mit der im package name. }
%% \ah{Schlimm genug, also sollte man zuerst diese Leute blamen und dann mich. CamelCase in Packages ist ja auch eher unansehnlich ... und die Webseite ist auch nach einem verlochten Samstag immernoch nicht konsistent damit.} 

%% \kai{Etc.\ -- MZ und ich waren eigentlich zu dem Schluss bekommen, dass wir das gar nicht (mehr) durchsetzen ... soll jeder contrib Programmierer sich in seinem eigenen Name space austoben (von uns auch auch org.ahorni....). }
%% \ah{... oder so.} 
%% \ah{Wie auch immer, das elende Contrib-Gewurstel generiert bei mir mittlerweile zuverlässig schlechte Laune.}

%% \kai{Die URL Referenzierung erfolgt nur noch über \url{http://matsim.org/javadoc} $\to$ /extension/.  Habe das daher auch so gemacht, also einfach nur die Namen der repositories zu listen.}

The idea of the \glspl{contribution} part of the repository is to host community contributions.
Historically, most contributors are from the \gls{matsim} team, but this is not a requirement.%
\footnote{
It is currently at \gls{github} under \url{https://github.com/matsim-org/matsim/tree/master/contribs}.  
}
The code is maintained by the corresponding contributor. 
Code in this section of the repository is considered more stable than code in playgrounds.
The \gls{java} packages often have the root \lstinline{org.matsim.contrib.*}, but this is not mandatory.

At the time of writing, there are the following \glspl{contribution} ($=$ extensions which are in the ``contrib'' part of the repository), listed in alphabetical order:
\begin{itemize}\styleItemize
\item \lstinline{accessibility},  
presented in Chapter~\ref{ch:accessibility}.
\item \lstinline{analysis}, presented in Chapter~\ref{sec:contrib-analysis}.
\item \lstinline{cadytsIntegration}, 
presented in Chapter~\ref{ch:cadyts}.
\item \lstinline{common} is not a true contrib, \ie it does not provide additional functionality by itself.  Instead, it is a place where code that is used by several contribs but that has not yet made it into the main distribution is located.  It also contains some long-running integration tests that are run at each build (\ie more often than those contained in the \lstinline{integration} contrib described below).
\item \lstinline{dvrp}, presented in Chapter~\ref{ch:dts}.
\item \lstinline{emissions}, presented in Chapter~\ref{ch:emissions}.
\item \lstinline{freight}, presented in Chapter~\ref{ch:freight}.
\item \lstinline{freightChainsFromTravelDiaries}, presented in Chapter~\ref{sec:freightChainsFromTravelDiaries}.
\item \lstinline{grips}, presented in Chapter~\ref{ch:evacuation}.
\item \lstinline{gtfs2matsimtransitschedule}, presented in Chapter~\ref{sec:SemiTool}
\item \lstinline{integration} is not a true contrib, \ie it does not provide additional functionality.  Instead, it is a place where integration tests that should run daily or weekly (instead of as often as possible) can be committed.
\item \lstinline{locationchoice}, presented in Chapter~\ref{ch:destinationchoice}.
\item \lstinline{matrixbasedptrouter}, presented in Chapter~\ref{sec:matrix-based-pt-router}.
\item \lstinline{matsim4urbansim}, presented in Chapter~\ref{sec:matsim4urbansim}.
\item \lstinline{minibus}, presented in Chapter~\ref{ch:minibus}.
\item \lstinline{multimodal}, presented in Chapter~\ref{ch:multimodalsim}.
\item \lstinline{networkEditor}, presented in Chapter~\ref{ch:contrib-networkEditor}.
%% \footnote{This contribution is not presented in the book. Instead, two external editors also developed by MATSim team members are presented in Chapter~\ref{ch:networkeditor}} 
%presented in Chapter~\ref{ch:networkeditor} \ah{afaik: the 2 presented ones are external}
%% \kai{@Andreas: bekommen wir noch einen Eintrag unter ``contributions in short'' hin?} \ah{Halte das für schlecht investierte Zeit. Das Ding ist doch veraltet oder? Würde es eher löschen. Die beiden anderen anderen Editoren sind ja ausführlich beschrieben.}
\item \lstinline{otfvis}, presented in Chapter~\ref{ch:otfvis}.
\item \lstinline{parking}, presented in Chapter~\ref{ch:parking}.
\item \lstinline{roadpricing}, presented in Chapter~\ref{ch:roadpricing}.
%\item \lstinline{org.matsim.contribs.signals.*} presented in Chapter~\ref{ch:signalslanes} \ah{only a placeholder}
\item \lstinline{socnetgen}, presented in Chapter~\ref{sec:contrib-socnetgen}.
\item \lstinline{socnetsim}, presented in Chapter~\ref{ch:jointtrips}.
\item \lstinline{transEnergySim}, presented in Chapter~\ref{ch:elvehicles}.
\item \lstinline{wagonSim}, presented in Chapter~\ref{ch:wagonSim}.
\end{itemize}

%% in the code base 
%% as \gls{matsim} contributions (\lstinline|package org.matsim.contrib.*|). Such modules are usually provided by \gls{matsim} \gls{api}-users or developers. %

%---------------------------------------------------------------
\subsection{Playgrounds}
Another element of the \gls{matsim} repository are the ``playgrounds''. 
These are meant as a service to programmers. 
The have historically grown from the fact that \gls{matsim}'s object classes and in consequence the interfaces between them have evolved and grown over time, and thus a stable \gls{api} was not available.  Regular code-wide refactorings, along the lines discussed, \eg by \citet{Fowler2004refactoring}, thus were the norm for many years.

At this point, the extension points described in Chapter~\ref{ch:extensionpoints} should be somewhat stable, and development against them should be possible without major changes from release to release.  
Anybody who needs tighter integration with the project should still apply for a playground.

% ====================================================================================
\subsection{Contributions and Extensions}
Congruent with the structure of this book, the \gls{matsim} code structure contains a core which allows to run basic \gls{matsim} using the \gls{configfile}, a population and a network. Packages going beyond this basic functionality are \glspl{extension}, where three different kind of extensions exist:
\begin{itemize}\styleItemize 
\item Extensions in the main distribution.%
\footnote{At the time of writing it is unclear if these extensions might become \glspl{contribution} one day, shrinking the \gls{matsim} main distribution to its core.
}
\item Extensions contributed by the \gls{matsim} community known as \glspl{contribution}.
\item Any code written anywhere published or unpublished extending the \gls{matsim} core.
\end{itemize}
Extensions are listed at \url{http://matsim.org/extensions}.

% ====================================================================================
\subsection{Releases, Nightly Builds and Code HEAD}
\label{sec:releases-builds}
%\kai{@Andreas: Meine starke Präferenz wäre immer noch, hier \emph{keine} sourceforge-Pfade anzugeben.  Sondern auf die entsprechenden Stellen in matsim.org zu verweisen.  Die Sourceforge-Pfade sind unserer Erfahrung nach nicht stabil.}
%\ah{Sourceforge-Pfade sind halt spezifischer.}

Releases, nightly builds and the code head can be obtained 
%by going further 
from \url{http://matsim.org/downloads}.

\gls{matsim} releases are published approximately annually.
%, which can be downloaded from \gls{sourceforge}. 
%at \url{http://sourceforge.net/projects/matsim/files/}. 
Usually, \gls{matsim} users and \gls{matsim} power users as defined above in Section~\ref{sec:roles} work with releases. 

MATSim 
%is based on 
uses continuous integration and, thus, nightly builds are available without stability guarantee under \url{http://matsim.org/downloads/nightly}. Nightly builds might be used by \gls{matsim} \gls{api} developers that depend on a very recent feature. 

Both \gls{maven} releases and \gls{maven} snapshots are available, see \url{http://matsim.org/downloads} for details.

\gls{matsim} \gls{api} developers or core developers often work on the code's HEAD version that can be checked out from \gls{github}.
%, which is available from \gls{sourceforge}.
% at \url{http://sourceforge.net/p/matsim/source/HEAD/tree/}.

Nightly builds and maven snapshots are only generated when the code compiles and passes the regression tests.  They are, in consequence, somewhat ``safer'' than the direct download from the HEAD.

% ##################################################################################################################
\section{Drivers, Organization and Tools of Development}
Important drivers of the \gls{matsim} development are the projects and dissertations of the \gls{matsim} team. 
New features are developed as an answer to requirements of these dissertations and projects, where projects range 
%
from purely scientific ones---often sponsored by \gls{snf} or \gls{dfg}---%
%
via projects for governmental entities
%administration 
%
and projects where science and industry contribute equally---\eg \gls{cti} projects---%
%
to purely 
%industrial 
commercial projects, which are managed by \gls{senozon} in the majority of cases. 
%
A significant amount of innovation is also introduced by the collaboration with external researchers.

Systematic code integration is mainly performed by the Berlin group and by \gls{senozon}. 
This includes continuous code review and integration upon request of the community, but also comprehensive code refactorings to clean up
%degenerated 
code and to 
improve modularity.  Refactorings are discussed and documented in the \gls{matsim} issue tracker (\url{http://matsim.org/issuetracker}).

The development process is supported by a standing \gls{matsim} committee discussing software and sometimes conceptual issues on a regular basis (\url{http://matsim.org/committee}). 
%
Another element bringing in innovation but also organization are the annual meetings. 
Right from the beginnings, there has been a \gls{matsim} developer meeting focused on coding issues (\url{http://matsim.org/devmeeting}, open only to authorized users).
%
Later, a user meeting giving insights into current work by the community has been added (\url{http://matsim.org/usermeetings}), sometimes combined with a tutorial.
%
Finally, a conceptual meeting is now hold every year,
%has been added strictly dedicated to 
concentrating on issues that go beyond pure software engineering (\url{http://matsim.org/conceptual-meetings}, also only open to authorized users). 
The developer meeting and the conceptual meeting 
%have shown to establish an effective 
together establish the road map that guides development for the remainder of the year. 
%
%% The internal development is organized by regular group meetings, in Zürich for example by the \gls{matsim} ``group meeting''.
%hab' das jetzt mal rausgestrichen.  So wichtig erscheint es mir nicht, und es scheint ja auch eher abnehmend zu sein.
%% and in Berlin by the "jour fix". 

%% Due to the heterogeneous character of the community and its individual research groups, with every member being engaged in his individual dissertation work, organizing the development in an agile and somewhat ad hoc manner has been regarded beneficial over adopting an established and strict software project management method.
%
%not sure what the above sentence means (MZ says that compared to the software company he worked for before we are quite well organized; compared to Bertrand Meyer/Eiffel we are possibly rather disorganized) --> removing it. kai, aug'15

% ====================================================================================
%\subsection{Development Tools}
%\kai{Moved some material from here to ``history''.}
%\ah{ok. Adapted title, as there was nothing about "core" here anyamore.}

\gls{matsim} development makes use of a 
%whole bunch 
large number of tools, hopefully leading to better software quality.  Historically, many of those tools ran from automated scripts and were made available at \url{http://matsim.org/developer}.  Nowadays, most of them are automatically available from the build server (see \url{http://ci.matsim.org:8080}) and/or from the repository (\url{https://github.com/matsim-org/matsim}), so that many of them are scheduled for removal from \url{http://matsim.org/developer}.  Some of these tools are:
%% Summarized and accessible from :
%---with private access---
%
%m.E. war das nie non-public.  Evtl. nicht auf der home page ausgewiesen. kai, dec'14
a change log; an issue tracker; the \gls{javadoc} documentation; static code analyses performed by \emph{FindBugs} and \emph{PMD}; test code coverage analysis; copy paste analysis; code metrics; \gls{maven} dependencies; 
%the complete code linked by \gls{xref}; % scheduled to go away. kai, aug'15
and information about the nightly test results. These nightly test results are generated by the \gls{matsim} build server based on the \gls{jenkins} software. 

Furthermore, there is a \gls{matsim} benchmark at \url{http://matsim.org/files/benchmark/benchmark.zip}. For results see \url{http://matsim.org/benchmark}.
%\url{http://www.matsim.org/docs/devguide/optimization}. \kai{link funktioniert nicht.  Wir hatten neulich schon danach gefahndet, aber das Material ist einfach in den latex user guide gewandert.  Der Benchmark selber ist unter \url{http://matsim.org/files/benchmark/benchmark.zip}.}
%\ah{danke}
%% \todokai{fix benchmarks www page}

Most \gls{matsim} developers use \gls{eclipse} as an \gls{ide}. The \gls{matsim} documentation is tailored to this \gls{ide}. Team development is currently based on \gls{git} as revision control system. External library dependencies are managed by \gls{maven}.

% ##################################################################################################################
\section{Documentation, Dissemination and Support}

The main documentation is now this book.  Additional information, incuding tutorials, can be found under \url{http://matsim.org/docs}.
%
%% can be found on the \gls{matsim} web page \citep[][]{MATSIM-Docu_Webpage_2015}, where information is provided in the three guides mentioned above. Additionally, a handful of tutorials is available, with the ``Quickstart''-tutorial to give fast access to \gls{matsim} and the ``Learning \gls{matsim} in 8\,Lessons''-tutorial, which is used for the user's meetings and hands-on tutorials in Berlin, Zürich and occasionally elsewhere, also listed on that page. \todokai{revise tutorial process}
%
Code documentation in form of \gls{javadoc} can be found unter \url{http://matsim.org/javadoc}.
%at \citet[][]{MATSIM-Javadoc_Webpage_2015}, 
%% as \gls{xref} view \url{http://matsim.org/xref} \todokai{wurde nicht gerade beschlossen, xref abzuschaffen?} and as doxygen documentation \url{http://matsim.org/doxygen}. 
%\kai{gerne auch xref und doxygen URLs: \url{http://matsim.org/xref} and \url{http://matsim.org/doxygen}.}

For fast application of \gls{matsim}, some small-scale example scenarios are provided in the code base (\lstinline|folder: examples|), where recently an extended version of the well-known benchmark scenario for the City of Sioux Falls has been added \citep[][]{ChakirovFourie_TechRep_FCL_2014} (Chapter~\ref{ch:siouxfalls}). %\kai{Auf den test input folder sollte m.E.\ auf keinen Fall verwiesen werden ... test inputs bleiben normalerweise auf dem Status, mit dem sie erstellt wurden, passen sich also nicht an neue Entwicklungen an.  Stattdessen bitte ``folder: examples''.} \ah{danke}
Additional example datasets, including Berlin datasets, can be obtained via \url{http://matsim.org/example-datasets}.

Further information is disseminated at the afore-described annual user meetings and \gls{matsim} mailing lists, see \url{http://matsim.org/mailinglists}.  Support is provided by the \gls{matsim} team via these mailing lists and via \url{http://matsim.org/faq}, both on a best effort basis. 
%% Particularly active in support are the Berlin group and \gls{senozon}. 
Many components of \gls{matsim} are documented by the numerous papers published in international journals and presented at worldwide conferences; information about such publications can, \eg be obtained from \url{http://matsim.org/publications}
%\citep[][]{MATSIM-Publications_Webpage_2015} \kai{@Andreas: Das sieht irgendwie weder schön noch hilfreich aus.  Hast Du mal so etwas wie author=\{MATSim www page\} im bibtex file probiert?} 
%% and in this book's part~II.
and from this book.
% not only part II. kai, aug'15

%% The information on \gls{matsim} is quite extensive, however, getting a complete picture as a new \gls{matsim} user or developer requires a substantial literature search. Minimization of this effort is the goal of the book at hand.

% ##################################################################################################################
\section{Your Contribution to MATSim}
\label{sec:yourcontribution}
The technical details, \ie the \gls{matsim} extension points, where to hook with \gls{matsim} are detailed in the following Chapter~\ref{ch:extensionpoints}.  Here, the different ways of contributing to \gls{matsim} according to the roles presented in Section~\ref{sec:roles} are introduced.

As a \gls{matsim} user, power user, or \gls{api} developer, you are warmly welcome to make an important impact by reporting your achievements, needs and problems with, or bugs of, the software via the users mailing list, the issue tracker, the FAQ, or at the annual \gls{matsim} user meeting. 

If you would like to directly contribute to the code base of \gls{matsim} you are welcome to become part of the \glspl{contribution} repository.

If you are the type of person that likes to change the core system, you can, although it is a long way, become a member of the \gls{matsim} core developers group. Core developers are usually picked from the \gls{matsim} team. Prerequisites are a strong computer scientist background, several years of experience with \gls{matsim} and a deep understanding of large software projects.

% ##################################################################################################################
%\section{Discussion and TODOs}
%\label{sec:development_process}
%Will be commented, when chapter is finished. Make final results traceable.
%
%% --------------
%\kai{Ich würde gerne ``Development process'' und ``Contributing to MATSim'' auftrennen.  Ersteres beschreibt Dinge wie core development, regression tests, etc.  Aber ``contributing to MATSim'' sollte sich auf die extension points konzentrieren.  Vielleicht zusammen mit dem vorhergehenden Kapitel.}
%
%\kai{Bin mir allerdings beim zweiten Nachdenken nicht mehr sicher, ob das so kategorisch sein muss.  Wir sollten erstmal versuchen, ``contributing to matsim'' als logische Folge des development process zu schreiben.  Dann sind ``minimally invasive contributions'' ja gerade solche, welche die extension points verwenden.}
%
%% --------------
%\ah{Vieles noch etwas zu implizit dargestellt! -> Aufzählungen einfügen. Playgrounds! Testing! Verantwortlichkeiten!}
%
%% --------------
%\ah{Classes und Methods stabil seit Langem: 
%auf was deutet das hin? Zu wenig Innovation oder Clean-Code? Steckt zuviel in den Playgrounds, Contribs oder ist das genau richtig?
%Wurde Kernfunktionalität eingefügt und dafür weniger Zentrales (wie Evacuation) nach und nach in Contrib verschoben?}
%
%\ah{Der Drop Ende 2007, war das schon Core, API Umstruktutierung oder war das die Entfernung von G. package?}
%\kai{Entfernung von Gunnar's package.}
%
%\createfigure%
%{Code development since 2005}%
%{Code development since 2005}%
%{\label{fig:codedev}}%
%{%
  %\createsubfigure%
  %{...}%
  %{\includegraphics[width=0.95\textwidth,angle=0]{extending/figures/nof_classes.png}}%
  %{\label{fig:codedev0}}%
  %{}%
  %\createsubfigure%
  %{...}%
	%{\includegraphics[width=0.95\textwidth,angle=0]{extending/figures/avg_method.png}}%
  %{\label{fig:codedev1}}%
  %{}%
%}%
%{}
%
%
%\ah{noch weiter eingehen auf Code Struktur? Resources, XML, ect.???}
%\ah{approx. number of playgrounds}
%\ah{approx number of code lines /classes at the different locations (core, playgrounds, contributions etc.}

% ##################################################################################################################
% Local Variables:
% mode: latex
% mode: reftex
% mode: visual-line
% TeX-master: "../main"
% comment-padding: 1
% fill-column: 9999
% End: 
