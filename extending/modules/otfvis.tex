\chapter{OTFVis}
\label{ch:otfvis}
% ##################################################################################################################

\hfill \textbf{Author:} ??

\begin{center} \includegraphics[width=0.25\textwidth, angle=0]{figures/MATSimBook.png} \end{center}

% ##################################################################################################################

\begin{compactitem}
\item Invoking the module: run \lstinline|org.matsim.contrib.otfvis.OTFVis|
\item Configuration: \lstinline|otfvis| configuration file section
\item Code: \lstinline|org.matsim.contrib.otfvis|
\end{compactitem}

Before senozon via the OTFVis (on-the-fly visualizer) \citep[][]{Strippgen_PhDThesis_2009} was the main visualization tool of MATSim. OTFVis is able to visualize results during runtime of a MATSim run. It is available as a contribution.

It is implemented in java, however, to make use of sophisticated visualization of the OpenGL framework, it is also heavily based on jogl. This requires a couple of user actions to get jogl running on his specific platform. This platform dependency has lead to some issue with the programs stability. It is thus recommended to use the senozon via. visualizer.
\michaz{The jogl and stability issues aren't true anymore, it runs out of the box everywhere and hasn't broken for quite some time. I also recommend using via, but only because otfvis is otfvis, not because of jogl. :-)}

% ################################################################################################################





