\chapter{OTFVis Visualizer}
\label{ch:otfvis}
% ##################################################################################################################

\hfill \textbf{Authors:} Kai Nagel, Michael Zilske ???

\begin{center} \includegraphics[width=0.25\textwidth, angle=0]{figures/MATSimBook.png} \end{center}

\createStandardInformationBasic{otfvis}{run \lstinline|org.matsim.contrib.otfvis.OTFVis|}{otfvis}{\citet[][]{Strippgen_PhDThesis_2009}}

% ##################################################################################################################
\begin{compactitem}
\item Invoking the module: 
\end{compactitem}

Before senozon via the OTFVis (on-the-fly visualizer) \citep[][]{Strippgen_PhDThesis_2009} was the main visualization tool of \gls{matsim}. OTFVis is able to visualize results during runtime of a \gls{matsim} run. It is available as a contribution. It is implemented in \gls{java}, however, to make use of sophisticated visualization of the OpenGL framework, it is also heavily based on jogl. This requires a couple of user actions to get jogl running on his specific platform. 

%This platform dependency has lead to some issue with the programs stability. It is thus recommended to use the senozon via. visualizer.
%\michaz{The jogl and stability issues aren't true anymore, it runs out of the box everywhere and hasn't broken for quite some time. I also recommend using via, but only because otfvis is otfvis, not because of jogl. :-)}

% ################################################################################################################





