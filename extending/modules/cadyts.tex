\chapter{Cadyts}
\label{ch:cadyts}
% ##################################################################################################################

\hfill \textbf{Authors:} Kai Nagel, Michael Zilske, Gunnar Fl\"otter\"od

\begin{center} \includegraphics[width=0.5\textwidth, angle=0]{extending/figures/cadyts/cadyts} \end{center}

\editdone{This text has undergone the professional edit. Please no grammatical changes anymore! They are most-probably wrong.}

\createStandardInformation{cadytsIntegration}
{\lstinline{RunCadyts4CarExample} class}
{cadytsCar}
{\citet[][]{cadyts-manual, floetteroed-2010e, FloetteroedChenEtAl2011BehavioralCalibAndAnaNETS, Floetteroed2008PhD, Moyo2013PhD}}

% ##################################################################################################################

%\ah{adopted from \citet[][]{Floetteroed2009CadytsSTRC}. Please check, in particular copy rights of figure}
%\gunnar{Checked. -- Die Abbildung taucht in dem NETS paper gar nicht auf. Sollte so gesehen unproblematisch sein.}
%\ah{ok}

% ##################################################################################################################
\section{Introduction}

\gls{cadyts}\footnote{\url{http://people.kth.se/~gunnarfl/cadyts.html}}
---licensed under \gls{gplv3}---calibrates disaggregate travel demand models 
of \gls{dta} simulators from traffic counts and vehicle re-identification data. 
\gls{cadyts} is broadly compatible with \gls{dta} microsimulators,
into which it can be hooked through sparse interfaces.

As explained officially in Chapter~\ref{ch:abta} and~\ref{ch:montecarlo}, 
\gls{dta} targets consistency between a travel demand dynamic model---  
\gls{matsim}, represented by the agents' plans---
and a network supply dynamic model, capturing spatiotemporal
network flows and congestion evolution.

\gls{cadyts} adjusts all agents' plan choice probabilities,
resulting in simulated network conditions consistent with 
measured real-world data, while maintaining behavioral 
plausibility of the underlying travel demand model. 
Within MATSim, plan choice probabilities adjustment is realized 
by adjusting plan scores, as explained in the next section.

% ##################################################################################################################
\section{Adjusting Plans Utility}

When traffic counts are the empirical source, plan-specific score corrections 
are composed of link- and time-additive terms $\Delta S_a(k)$ for each link $a$
and each calibration time step $k$ (often one hour). 
When congestion is light and traffic counts are independently and normally distributed, 
these correction terms become 
%\cite[p.487]{FloetteroedChenEtAl2011BehavioralCalibAndAnaNETS}
%
\begin{equation}
\label{eq:cadyts:correction}
\Delta S_a(k) = \frac{y_a(k) - q_a(k)} {\sigma_a^2(k)}
\end{equation}
%
where $y_a(k)$ is real-world measurement on link $a$ in time step $k$, 
$q_a(k)$ is its simulated counterpart, and $\sigma^2_a(k)$ is (an estimate of) 
the real measurement variance  (assuming its expected value 
coincides with the prediction $q_a(k)$ of a perfectly calibrated simulator).

Score correction of an agent's given activity-travel plan is calculated as 
the sum of all $\Delta S_a(k)$, when following that plan 
implies entering link $a$ within time step $k$. 
%\citep{FloetteroedChenEtAl2011BehavioralCalibAndAnaNETS}.
With this, the \textit{a posteriori} choice probability of agent $n$'s plan $i$, given
the count data ${\bf y} = \{y_a(k)\}$, becomes:
%
\begin{equation}
P_n(i\mid{\bf y}) \sim 
\exp\left(S_n(i)+ \sum_{ak\in_{}i} \Delta S_a(k) \right)
\,\,=\,\,
\exp\left(S_n(i)+ \sum_{ak\in_{}i} \frac{y_a(k) - q_a(k)} {\sigma_a^2(k)} \right)
\label{eq:cadyts:selection}
\end{equation}
where $S_n(i)$ is the \textit{a priori} score of plan $i$ of agent $n$, as 
calculated---for example---with Equation~(\ref{eq:matsimUTF}) and $ak\in i$ reads: 
``following plan $i$ implies entering link $a$ in time step $k$''.

Intuitively, if the simulation value $q_a(k)$ is smaller than the real measurement 
$y_a(k)$, then a score increase, and thus a choice increase 
probability, results. 
$\sigma^2_a(k)$ denotes level of trust in that specific measurement
---a large variance $\sigma^2_a(k)$ implies low trust level, taking effect through
a large denominator in the corresponding score correction addend.

\citet[][]{floetteroed-2010e} is the key methodological reference on \gls{cadyts}.
Its calibration approach is derived from a Bayesian argument, providing
more technical information, such as a more general functional utility form 
correction in Equation~(\ref{eq:cadyts:correction}), that also applies when congestion is present. 
A  lighter presentation is 
\citet[][]{FloetteroedChenEtAl2011BehavioralCalibAndAnaNETS}, where the 
formulas above are discussed in somewhat greater detail.

% ##################################################################################################################
\section{Hooking Cadyts Into MATSim}
Hooking \gls{cadyts} into \gls{matsim} is based on the following operations:
\begin{enumerate}\styleEnumerate
\item Initialization: When the calibration is started, it requires all available 
traffic counts and some further parameters. 
For this, the \gls{cadyts} function \lstinline|void addMeasurement(...)| is called once for every 
measurement, before the simulation starts. It registers a certain measurement type, which 
has been observed on a specific link.
\item Iterations: The calibration is run jointly with the simulation until (calibrated) 
stationary conditions are reached.
	\begin{enumerate}[label=\emph{\alph*})]
	\item Demand simulation: The calibration needs an access point in the simulation 
	to affect the plan choice. There are various ways to realize this, depending on the  
	simulator.	
	%\gunnar{Trifft Folgendes f\"ur MATSim zu? Ich habe diesen Funktionsaufruf aus der
	%CadytsScoring-Klasse herausgefischt, bin mir aber nicht sicher, wie das auf der MATSim-Seite
	%zusammenspielt.}
	%\ah{m.E. geschieht genau das in \lstinline|org.matsim.contrib.cadyts.general.CadytsPlanChanger.selectPlan(...)|}\kai{Ja.  Noch einfacher in CadytsScoring.}
	Before a \gls{matsim} agent chooses a plan, he or she asks the calibration through the 
	\gls{cadyts} function
	\lstinline|double calcLinearPlanEffect(Plan plan)| for all of its plans' score offsets.
    The agent then chooses a plan based on accordingly modified scores.
	\item Supply simulation: The calibration must observe simulated network conditions 
  to evaluate their deviation from real traffic counts.
	For this, the \gls{cadyts} function \lstinline|void afterNetworkLoading(SimResults simResults)| 
	is called once after each network loading. It passes a container object to the calibration, 
	providing information about most recent network loading results, particularly on 
	simulated flows at measurement locations.
	\end{enumerate}
\end{enumerate}

% ##################################################################################################################
\section{Applications}
\gls{cadyts} has been successfully applied in studies like % FloetteroedEtAl2009IatbrCalibration, 
\citet[][]{ZiemkeNagelBhatIntegratingCemdapMatsimTransferability,ZilskeNagelPhoneTracesAndCadyts, FloetteroedChenEtAl2011BehavioralCalibAndAnaNETS}. 
Zürich scenario results illustrate its efficiency, as shown in \citet[][Slide~8]{FloetteroedEtAl_unpub_MATSimUserMeeting_2011}, 
reproduced in Figure~\ref{fig:cadyts}.

\createfigure%
{Zürich case study results: mean relative error in link volumes}%
{Zürich case study results: mean relative error in link volumes}%
{\label{fig:cadyts}}%
{\includegraphics[width=0.8\textwidth, angle=0]{extending/figures/cadyts/cadyts}}%
{\citet[][Slide~8]{FloetteroedEtAl_unpub_MATSimUserMeeting_2011}}

% ##################################################################################################################
% Local Variables:
% mode: latex
% mode: reftex
% mode: visual-line
% TeX-master: "../../main"
% comment-padding: 1
% fill-column: 9999
% End: 
