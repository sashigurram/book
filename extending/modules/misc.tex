\chapter{Miscellaneous Contribs in Short}
\label{ch:misccontribs}
% ##################################################################################################################

\hfill \textbf{Author:} Kai Nagel

% ##################################################################################################################
\section{Analysis}
\label{sec:contrib-analysis}
\createStandardInformationBasic{{\url{http://matsim.org/javadoc} $\to$ analysis}}{\lstinline|RunKNEventsAnalyzer| class}{Analysis tools without standard configuration.}{-}

\lstinline{org.matsim.contribs.analysis.*}

% ##################################################################################################################
\section{freightChainsFromTravelDiaries}
\label{sec:freightChainsFromTravelDiaries}
\createStandardInformationBasic{{\url{http://matsim.org/javadoc} $\to$ freightChainsFromTravelDiaries}}{Currently not possible.}{No standard configuration.}{
%
\cite{Schneider2011PhD}
%
}

Sebastian Schneider has done a Ph.D.\ dissertation about generating freight vehicle chains by essentially re-sampling the information contained in the German survey KiD \citep{SteinmeyerWagner2005KiD}.  Since the KiD is essentially an activity-based travel diary, the method should also be applicable to other situations.
Since Sebastian has left science for the time being, he allowed us to take his code and integrate it into the repository, under the GPL. For the time being, it will just ``sit'' here until someone attempts to make it work.


%##################################################################################################################
%\section{gtfs2matsimtransitschedule}
%\label{sec:gtfs2matsimtransitschedule}
%
%\kai{todo not kai}

%\ah{wird in in Section~\ref{sec:SemiTool} abgehandelt}

% ##################################################################################################################
\section{matrix based pt router}
\label{sec:matrix-based-pt-router}
\createStandardInformationBasic{{\url{http://matsim.org/javadoc} $\to$ matrixbasedptrouter}}{\lstinline|RunMatrixBasedPTRouterExample| class}{No standard configuration.}{-}
\kai{todo kai}

The matrix based pt router reads a list of public transit (pt) stops, and constructs pt routes using the stops nearest to origin and destination. Travel times and travel distances between pt stops can be given by corresponding matrices.

% ##################################################################################################################
\section{MATSim4UrbanSim}
\label{sec:matsim4urbansim}

\editdone{This text has undergone the professional edit. Please no grammatical changes anymore! They are most-probably wrong.}

\subsection{Basic Information}
\label{sec:matsim4urbansim-stdInfo}

\createStandardInformationBasic{%
\url{http://matsim.org/javadoc} $\to$ matsim4urbansim
%
}{%
%
The module is invoked from a live UrbanSim implementation.
%
}{%
%
In the UrbanSim config file.
%
}{%
\citet[][]{Nicolai2011Couplinganurbansimulationmodelwithatravelmodel}; \citet{NicolaiNagelHiResAccessibilityMethod}; \citet{NicolaiNagel2013HandbookIntegrationOfTransportLanduse}
%
}

% ##################################################################################################################
\subsection{Summary}
``MATSim4UrbanSim'' is an adapter package for using \gls{matsim} as a travel model plug-in to \gls{urbansim}, a well-known land use simulation \citep[e.g.][see \url{http://www.urbansim.org}]{WaddellEtc2003UrbanSim}.
\gls{urbansim} has, for example, submodels for residential location choice, commercial location choice, or development and building construction, thus creating synthetic  potential urban or regional development scenarios under various conditions and constraints. 
Traffic infrastructure plays a significant role in such developments; for example, very accessible areas are more attractive as residences and for commercial activities. 
Since accessibility is reduced by congestion, and congestion can only be realistically modeled through a sophisticated model of demand and supply interaction, \gls{urbansim} does not have its own travel model, but delegates that task to external models, such as \gls{matsim}.

To use MATSim4UrbanSim, one first needs to have a running \gls{urbansim} installation. 
From there, one can add \gls{matsim} to that installation; see Section~\ref{sec:matsim4urbansim-stdInfo} for more information. 
Basic \gls{matsim} parameters are configured from the \gls{urbansim} configuration file by adding an appropriate section; again, see Section~\ref{sec:matsim4urbansim-stdInfo} for more information. 
It is possible to add a standard \gls{matsim} \gls{configfile} allowing use of the extended \gls{matsim} features, including those added after the adapter package was designed.

The module was applied by \citet[][]{ZoelligRenner_PhDThesis_2014}, but it has not been maintained since 2013 and it is unclear how much of it currently works.

% ##################################################################################################################

% Local Variables:
% mode: latex
% mode: reftex
% mode: visual-line
% TeX-master: "../../main"
% comment-padding: 1
% fill-column: 9999
% End: 
