\chapter{Accessibility \who{Ziemke}}
\label{ch:accessibility}
% ##################################################################################################################

\hfill \textbf{Author:} Dominik Ziemke

\begin{center} \includegraphics[width=0.25\textwidth, angle=0]{figures/MATSimBook.png} \end{center}

% ##################################################################################################################
\ah{please delete everything, that you do not need anymore!}

The term accessibility is widely used in the context of transport and
infrastructure planning. The improvement of accessibility is often stated as a
central goal of discussed transport or infrastructure schemes. In contrast to the
ubiquitous use of the term accessibility, however, approaches to the quantitative analysis
of accessibility are comparatively rare. This is striking since first such
approaches appear to have been undertaken at the latest in the 1950s (cite Hansen).
Likewise, it is notable that the development of gravity models, which have since
then been the most prominent and widely-applied type of destination choice models,
has originally been intertwined with quantitative approaches to accessibility
analysis.

Today, methods to assess the quality of accessibility (cite BBSR, TUHH, TUM,
Curtis, ...) are mainly used in superordinate planning procedures like regional
planning, where a central goal is to provide citizens with a certain quality of
access to various services. Curtis et al. (cite Curtis...) also argue that
accessibility-based analysis methods may be suitable to overcome the deficits that
traditional transport analysis methods possess in terms of analyzing transport
policies that are not construction-based, e.g. transport demand management schemes.

Depending on the concrete definition of quantitative accessibility measures, the
explanatory power of the result varies widely.
%
% following based on NicolaiNagel
%
As pointed out by Nicolai and Nagel (cite NicolaiNagel2013....) also see \citep[e.g.][]{GeursRitsema2001AccessibilityMeasures,Geurs2004AccessibilityReview}, quantitative indicators can rely on the following approaches:
%
\begin{enumerate}
\item An \textbf{activity-based} or \textbf{land-use-based} approach focuses on
the distribution of possible activity locations (land use). One can, for instance,
base calculations on the number and spatial distribution of activity opportunities
like shopping locations or workplaces within a certain distance.
%
\item An \textbf{in\-fra\-struc\-ture-based} or \textbf{transport-based} approach takes
into account the effort to travel from a given origin to a given destination and 
can be based on performance characteristics of the transport system, e.g. the
average speed by mode at certain locations. If one considers, for instance, the
number of shopping locations or workplaces under a defined travel time threshold,
the activity-based and the infrastructure-based approaches can be combined.
%
\item A \textbf{temporal} component, which considers the availability of activities
at different times-of-day, may be added.
%
\item An \textbf{individual} approach to accessibility computations can be
obtained by that addressing different needs and opportunities of different socio-economic groups, e.g. different income groups.
%
\item A \textbf{utility-based} measurement of accessibility reflects the
(economic) benefits, as the maximum expected utility, that someone gains
from access to spatially distributed opportunities
\citep{GeursRitsema2001AccessibilityMeasures,deJongEtAl2007LogsumTRA}. The
typical example is the logsum term, which is discussed in ??????.
\end{enumerate}
%
Established and frequently applied
approaches to accessibility analyses often calculate travel times to
the next facility that offers a certain type of service (e.g. travel time to the
next airport or next hospital) and define this travel time as the accessibility. As
such, accessibilities are oftentimes activity-specific.
%, i.e. calculated with regard to a certain type of service.
One can argue, however, that the activity-specific
accessibility is in reality not only defined by the impedance to reach \texttt{the nearest}
facility serving a particular need. Instead, the measure will become more
insightful, if a certain number of facilities serving the same need is taken into
account because different facilities of the same type may offer a given service in
different qualities. Likewise, services offered may be complementary. For instance,
a person wishing to make a holiday trip by airplane will likely take into account several airports in her/his vicinity when planning a journey instead
of just looking at the flights offered from the nearest airport. Therefore, the
accessibility to airports should be made dependent on the impedances to reach all of these airports
instead of just the impedance to reach the nearest one. Similarly, the accessibility to
medical services will in reality be perceived as better if several hospitals, potentially offering
different kinds of medical treatments, are reachable from a given location. Therefore, all reachable
hospitals should be considered when calculating accessibilities of medical
facilities instead of just taking into account the closest hospital. The latter
would only be sufficient in cases where it can be assumed that \texttt{any}
hospital can provide a suitable treatment. In order to account for the multitude of facilities serving a
given need, a (weighted) sum over the accessibilities of these facilities appears
reasonable. The accessibility extension in MATSim is based on such weighted sums.

Another question is the calculation of accessibilities from a given location to facilities of a ...



congestion-effects

free data

\section{Invocation}

\subsection{Minimal}

...



\subsection{Invocation as ``script in Java''}

The above \lstinline$org.matsim.accessibility.run.Main$ class can also be used as a starting point for one's own ``script in Java'':
\begin{lstlisting}
public static void main(String[] args) {

		if ( args.length==0 || args.length>1 ) {
			throw new RuntimeException("useage: ...Main config.xml") ;
		}
		Config config = ConfigUtils.loadConfig( args[0] ) ;
		
		Scenario scenario = ScenarioUtils.loadScenario( config ) ;
		
		// the run method is extracted so that a test can operate on it.
		run( scenario);
		
		
	}

	public static void run(Scenario scenario) {
		
		List<String> activityTypes = new ArrayList<String>() ;
		ActivityFacilities homes = FacilitiesUtils.createActivityFacilities("homes") ;
		for ( ActivityFacility fac : scenario.getActivityFacilities().getFacilities().values()  ) {
			for ( ActivityOption option : fac.getActivityOptions().values() ) {
				// figure out all activity types
				if ( !activityTypes.contains(option.getType()) ) {
					activityTypes.add( option.getType() ) ;
				}
				// figure out where the homes are
				if ( option.getType().equals("h") ) {
					homes.addActivityFacility(fac);
				}
			}
		}
		
		log.warn( "found activity types: " + activityTypes ); 
		
		// yyyy there is some problem with activity types: in some algorithms, only the first letter is interpreted, in some other algorithms,
		// the whole string.  BEWARE!  This is not good software design and should be changed.  kai, feb'14
		
		Map<String, ActivityFacilities> activityFacilitiesMap = new HashMap<String, ActivityFacilities>();
		Controler controler = new Controler(scenario) ;
		controler.setOverwriteFiles(true);

		for ( String actType : activityTypes ) {
			
//			if ( !actType.equals("w") ) {
//				log.error("skipping everything except work for debugging purposes; remove in production code. kai, feb'14") ;
//				continue ;
//			}
			
			ActivityFacilities opportunities = FacilitiesUtils.createActivityFacilities() ;
			for ( ActivityFacility fac : scenario.getActivityFacilities().getFacilities().values()  ) {
				for ( ActivityOption option : fac.getActivityOptions().values() ) {
					if ( option.getType().equals(actType) ) {
						opportunities.addActivityFacility(fac);
					}
				}
			}
			
			GridBasedAccessibilityControlerListenerV3 listener = 
				new GridBasedAccessibilityControlerListenerV3(opportunities, scenario.getConfig(), scenario.getNetwork());

			// define the modes that will be considered
			// the following modes are available (see AccessibilityControlerListenerImpl): freeSpeed, car, bike, walk, pt
			listener.setComputingAccessibilityForMode(Modes4Accessibility.freeSpeed, true);

			// this has to do with the additional population density column:
			listener.addAdditionalFacilityData(homes) ;

			listener.generateGridsAndMeasuringPointsByNetwork(100.);
			// yy todo: meaningful error message if this is not set
			// yy todo: make sure that cell size is taken from config.
			
			listener.writeToSubdirectoryWithName(actType);
			
			controler.addControlerListener(listener);
		}
					
		controler.run();
	}


#######################################



public static void main(String[] args) {
	// load the config, telling it to "materialize" the road pricing section:
	Config config = ConfigUtils.loadConfig(args[0], new RoadPricingConfigGroup());
	
	// load the scenario:
	Scenario scenario = ScenarioUtils.loadScenario(config) ;

	// instantiate the controler:
	Controler controler = new Controler(scenario) ;

	// instantiate road pricing and add it as controler listener:
	RoadPricing roadPricing = new RoadPricing() ;
	controler.addControlerListener( roadPricing ) ;

	// run the controler:
	controler.run() ;
}
\end{lstlisting}

\subsection{More information}

\url{http://ci.matsim.org:8080/job/MATSim_contrib_M2/org.matsim.contrib$accessibility/javadoc/?}

\ah{
-> Vortrag Wiepersdorf
Modules in the config: 
\begin{itemize}
	\item \lstinline|accessibility|
\end{itemize}

Package: 
\begin{itemize}
	\item \lstinline|org.matsim.contrib.emissions|
\end{itemize}

Usage: Contrib via config

http://www.matsim.org/accessibility

% http://ci.matsim.org:8080/job/MATSim_contrib_M2/org.matsim.contrib$accessibility/javadoc/?

hängt auch noch mit MATSim4Urbansim zusammen: %http://www.google.ch/url?sa=t&rct=j&q=&esrc=s&source=web&cd=1&ved=0CB4QFjAA&url=http%3A%2F%2Fmatsim.org%2Fuploads%2FMATSim4UrbanSim.pdf&ei=XAjJU7-VE6Gr0QWx14DADA&usg=AFQjCNGdMwxY-q9VuXOC6oFwIyW7LVKTBg&sig2=74HJB3wmajqkYWEDYHg2Ng&bvm=bv.71198958,d.d2k&cad=rja
}
%
\kai{M.E.\ nicht.  matsim4urbansim verwendet accessibility, aber nicht anders herum.}


% ##################################################################################################################
