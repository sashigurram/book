\chapter{Discontinued Modules}
\label{ch:discontinued}
% ##################################################################################################################

\hfill \textbf{Authors:} Kai Nagel, Andreas Horni

\editdone{This text has undergone the professional edit. Please no grammatical changes anymore! They are most-probably wrong.}

% ##################################################################################################################
This chapter lists modules that were important for several projects in the past, but which are no longer being developed.

% ################################################################################################################
\section{DEQSim}
\label{sec:deqsim}
\gls{deqsim} was used for project \emph{Westumfahrung} \citep[][]{BalmerEtAl_ResRep_bdktzrh_2009}. It was a queue-based, event-based parallel simulation written in \gls{cpp} \citep[][]{CharyparAxhausenEtAl2007Event-DrivenQueueBasedTraffic,Charypar_PhDThesis_2008}. This simulation included handling of reduced capacities due to traffic lights in an aggregate manner \citep[][p.139 ff]{Charypar_PhDThesis_2008}. It also supported modeling of gap back propagation at junctions \citep[][p.98 ff]{Charypar_PhDThesis_2008}.

Events were written do file by \gls{deqsim} and subsequently read by \gls{matsim}.  This represented a major framework performance bottleneck. \Gls{deqsim} was therefore replaced by a \gls{java} version, the \gls{jdeqsim} (see Section~\ref{sec:using-jdeqsim}).
 %% \kai{Andreas, gibt es da ein Kapitel zu?}. \ah{nicht wirklich}).

% ################################################################################################################
\section{Planomat}
\label{sec:planomat}
%\kai{Müssen wir das hier wirklich drinlassen?  Es hatte ja schon seinen Grund, dass wir das aus \gls{matsim} entfernt haben.}
%\ah{Ist die Frage, ob man die Historie noch mitschleppen will oder nur den aktuellen Stand beschreiben will. Bin in diesem Fall etwas unschlüssig. Planomat war für zwei grosse Projekte hier ein wichtiges Teil, deshalb wird man gute Argumente hören dafür es drinzulassen. Für den aktuellen Stand ist es aber tatsächlich nicht mehr relevant.}

Chapter~\ref{ch:extensionpoints} explains how \gls{matsim} can be extended. One long-standing extension point is the \lstinline{PlanStrategy} extension point (Section~\ref{sec:replanning-extension-point}). It allows the addition of ``innovative'' strategy modules (see Section~\ref{sec:strategymodules}), above and beyond those available by default.

One such replanning model was Planomat \citep[][]{MeisterBalmerEtc2006planomatIatbr, %MeisterEtAl_STRC_2006, %same as iatbr
Meister_PhDThesis_2011}. It replaced the randomizing modules for (departure) time innovation (Section~\ref{sec:timechoice}) and for mode innovation (Section~\ref{sec:modechoice}), with a module that computed a joint best reply for both choice dimensions internally, using a Genetic Algorithm. Thus, it evaluated not just one random alternative per iteration, as standard \gls{matsim} would do, but multiple alternatives within one single iteration, to obtain an (at least locally) optimal solution. Planomat was successfully applied in the project ``KTI Frequencies'' for time and mode innovation for sub-tours \citep[][p.10]{BalmerEtAl_ResRep_datapuls_2010}.
Unfortunately, there were three interacting problem complexes with Planomat:
%
\begin{itemize}\styleItemize
\item Any strategy module generating best reply plans must be able to compare plans and select a better one, at least along the considered choice dimensions. This is typically achieved by giving such a module an objective function which needs to be optimized. For example, a fastest path router minimizes the travel time; a generalized cost router minimizes the generalized travel cost.

All best reply modules here face the challenge that they cannot run the full \gls{mobsim} ($=$ network loading $=$ synthetic reality) every time they need such information. As a result, all best reply modules are forced to build some internal synthetic reality model.  

Planomat did this by running its plans through a simplified \gls{mobsim} of its own. This \gls{mobsim} was a re-implementation of the most important aspects for the core \gls{mobsim}.
%
%% Planomat 
%% %did not re-use core \gls{matsim} structural information, but 
%% built its own model of the environment.  For example, it would not use the \gls{matsim} core scoring function provided by the infrastructure, but compute its own score based on the events. \kai{Thibaut or Andreas H., is this correct?} \ah{at least for the KTI frequencies project based on Planomat this is true.}  \thibaut{Uhm, not really, this is exactly the opposite: it would go through the new plan, and ``fake'' events, that it would pass to the scoring function. It caused problems, because the way plans were replayed could easily differ from what the mobsim did (it for instance did not use the ``activity duration interpretation'' parameter, or was unable to consider tolls, as they are represented by money events thrown during the mobsim step), which was pretty hard to check (I remember having passed a lot of time trying to figure out the proper way to consider start, end time and duration in activities).}
%% \ah{It also used its own scoring function, right?}
%
Unfortunately, however, this meant that Planomat would not automatically pick up any change or addition to the core \gls{mobsim}.
%
Consequently,  Planomat's idea of a good plan often diverged from \gls{matsim}'s, especially when \gls{matsim} extensions were used.
%
In other words, any addition to the \gls{matsim} system: \eg tolls, or opening/closing time restrictions, or differentiating link travel times by turning directions, would have to be mirrored inside Planomat.
%
\item Planomat always tended to return the same solution: understandable from a best-reply module, but it becomes a problem when what the module thinks is a best reply starts to differ from what the \gls{matsim} core thinks.  

While an innovative strategy that deliberately generates diversity can be useful even when not fully consistent with the \gls{matsim} core \citep{NagelKickhoeferJoubert2014HeterogeneousVoTsPROCEDIA}, this cannot function with a non-diverse innovative strategy, since it then insists on returning only suboptimal plans.
%
\item In addition, Planomat used the \gls{matsim} core router in a way that hindered further software development of the core router. Essentially, Planomat used \gls{matsim} classes and methods not designed for re-use, but just happened to be public.

It was thus an obstacle for a major \gls{matsim} core router re-design, undertaken by T.~Dubernet (se Section~\ref{sec:router-extension-point}).
\end{itemize}

The combination of these three issues meant that Planomat was eventually discarded; moving it to the new router infrastructure would have entailed a major piece of one-time work. After that, maintaining Planomat's best-reply capability would have been a permanent work-intensive obligation. It was thus decided instead to invest our scarce resources in the design of a better core allowing extensions to survive without much manual intervention. Although this will always be work in progress, Chapter~\ref{ch:extensionpoints} explains our substantial progress toward pluggable extensibility.

%\kai{pls note following new paragraph:} 
However, it must be noted that the improved software architecture does not resolve the general conceptual problem; best reply modules somehow need to follow core system development. Chapter~\ref{ch:psim} discusses a newer alternative that re-uses \gls{mobsim} output for plan evaluation without having to run the full \gls{mobsim} every time. An alternative approach, based on plan diversity, is investigated by \cite{NagelKickhoeferJoubert2014HeterogeneousVoTsPROCEDIA}. Additionally, Chapter~\ref{ch:discretechoice} discusses aspects of diversity in plan set generation.

%% The Planomat module cannot be invoked directly anymore; it is not available in current releases anymore but can be downloaded from SVN history.
%% \kai{hier stand noch das mit der svn history.  ich vermute mal, dass man das wirklich noch aus svn bekommt, aber wollen wir das wirklich nochmal ausprobieren?  Falls nicht, wollen wir das wirklich schreiben?}
%% \ah{Vermutlich sollte man die Leute tatsächlich "zwingen" erstmal mit dem \gls{matsim}-Team zu sprechen, anstatt sie implizit zu ermuntern auf eigene Faust sich das Zeug zu besorgen und loszulegen.}

% ################################################################################################################
\section{PlanomatX}
\label{sec:planomatX}
PlanomatX was based on Planomat. It extended it by performing activity choice and adopting a Tabu Search approach \citep[][]{Feil_PhDThesis_2010}. To cope with the dimensionality curse (due to the added choice dimension), PlanomatX introduced schedule recycling, basically a warmstart concept. Because of problems when using the standard \gls{matsim} logarithmic utility function for activity choice, PlanomatX also derived an alternative utility function from \citet[][]{Joh_PhDThesis_2004}. Rough estimates for its parameters based on an \gls{mnl} exist, but turned out to be problematic, as shown in Section~\ref{sec:s-shaped}.

PlanomatX, derived from Planomat, suffered from the same maintenance problems and was eventually abandoned for the same reasons.

% ################################################################################################################

%\kai{habe das queueSimulation Kapitel mal auskommentiert; falls nicht ok, sollten wir da nochmal drüber diskutieren.}
% \ah{ok}

%% \section{queueSimulation}
%% \label{sec:queueSimulation}

%% \kai{Anders als bei planomat finde ich, dass wir dies hier auch einfach komplett weglassen können.  Es gibt m.E.\ keine Veröffentlichungen, die nur Sinn machen, wenn man dies hier weiß; soweit wir es getestet haben, waren die beiden mobsims immer äquivalent.}

%% For some time, there existed two different versions of the QSim, the more advanced and experimental one called ``QSim'', and the more basic one named ``simulation'' or ``queueSimulation''.   Essentially, the basic version was forked off just before QSim exploded in terms of functionality and extensibility, in order to have a much simpler reference mobsim available.  

%% Over time, however, changes in the core made it necessary to adapt the basic queue simulation to those changes.  Many of these changes had to do with agent behavior, where the basic queue simulation was not able to differentiate, say, between activity end and departure.  In order to reduce the maintenance burden, eventually the agent concept in the basic queue simulation was retrofitted to the agent concept of the more advanced QSim.  In the end this meant that the basic queue simulation was slowly drifing away from the original fork, while on the other hand not being used and thus not tested by the \gls{matsim} core team.

%% It was thus eventually decided to remove the basic queue simulation from the repository, and concentrate on \lstinline{QSim}.
%% %% The queueSimulation was another mobsim forked from QSim. It cannot be invoked directly anymore; it is not available in current releases anymore but can be downloaded from SVN history. 
%% The \lstinline|simulation| config 
%% file section, which you might come across in old config files, is deprecated as well. 
%% %\ah{stimmt das so?}\michaz{Yes. IMO let's not even mention it. The fact that it was forked from QSim at all is a technicality which nobody wants to know.})


% ################################################################################################################
%\section{Discussions}
%% ===================================================================================
%\subsection{Planomat}
%\label{sec:planomat}
%\kai{Müssen wir das hier wirklich drinlassen?  Es hatte ja schon seinen Grund, dass wir das aus \gls{matsim} entfernt haben.}
%\ah{Ist die Frage, ob man die Historie noch mitschleppen will oder nur den aktuellen Stand beschreiben will. Bin in diesem Fall etwas unschlüssig. Planomat war für zwei grosse Projekte hier ein wichtiges Teil, deshalb wird man gute Argumente hören dafür es drinzulassen. Für den aktuellen Stand ist es aber tatsächlich nicht mehr relevant.}
%
%% ===================================================================================
%\subsection{PlanomatX ... Activity Choice}
%\label{sec:activitychoice}
%
%\kai{Müssen wir das hier wirklich drinlassen?  Es hatte ja schon seinen Grund, dass wir das aus \gls{matsim} entfernt haben.}
%\ah{Ähnlich oben, nur dass ich hier die Argumente dann nicht so gut nachvollziehen kann.}
%
%\kai{Ich habe das wie folgt in Erinnerung: (1) PlanomatX war mit heißer Nagel gestrickt.  Es gab nie ein Bemühem um code re-use, weder im code von Feil selber noch in bezug auf \gls{matsim} Infrastruktur.  Jedes "computational experiment" wurde definiert durch eine weitgehenden Kopie der gesamten code base.  Wir sahen uns nicht dazu in der Lage, hier dasjenige Subset zu extrahieren, welches einer sinnvollen Funktionalität entsprach.  (2) Durch die vielen Kopien war der playground beim refactoring massiv im Weg.  Es ist schon ok, wenn man bestimmte code Zeilen pro playground ein- oder zweimal austauschen muss, aber 14x ist zu viel.  (3) Michael Balmer sagte, dass er den Resultaten, insbesondere denen in den hinteren Kapiteln der Diss, nicht trauen würde (sonst hätten wir vielleicht ein paar test cases drumherum gebaut und versucht, es zu retten).}
%
%\kai{Ich sehe das von daher gesehen interessanterweise eher umgekehrt wie Du: Meiner Intuition nach hätte eher der normale planomat bleiben können.  Hier kann ich mich nicht dran erinnern, dass ich eine eigene Meinung hatte.  Als wesentliches technisches Problem habe ich in Erinnerung, dass planomat die Tendenz hatte, immer die gleiche Lösung zu finden, völlig unabhängig von den Präferenzen des Agenten (z.B.\ value of time).}
%
%\kai{Wenn es wg.\ der Historie ist, würde ich es in einer separaten Section unterbringen.}
%
%\kai{Falls es Euch wirklich wichtig ist, dann kann man ein ``contrib'' Kapitel ``planomat'' einfügen, analog zu den anderen contribs (accessibility, roadpricing, emissions, locationchoice, ...).  Wenn alles gut gelaufen wäre, dann wären auch planomat/planomatX jetzt contribs.  Bei PlanomatX wurde, wie gesagt, von Feil nie genug geliefert, dass das Chancen gehabt hätte.  Bei Planomat ist mir, wie gesagt, der Prozess nicht ganz klar.}
%
%\kai{Vermutlich brauchen wir irgendwann eine oder mehrere Skype Bereinigungstreffen, oder? :-)}
%
%\ah{Da habe ich mich missverständlich ausgedrückt: Die Argumente es \textbf{zu behalten} kann ich für PlanomatX nur schwer nachvollziehen, was mir bei Planomat besser gelingt.
%
%kwa ist aber auch mehr für "Current State" denn für "History". Allerdings wird wohl Planomat(X) von Balac wieder reaktiviert werden (müssen). D.h., irgendwo eine kurze Section zu "`Former Important Modules"' wäre evtl. nicht schlecht.}

% ===================================================================================
% ################################################################################################################

% Local Variables:
% mode: latex
% mode: reftex
% mode: visual-line
% TeX-master: "../../main"
% comment-padding: 1
% fill-column: 9999
% End: 
