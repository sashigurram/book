\chapter{Roadpricing \who{Nagel?}}
\label{ch:roadpricing}
% ##################################################################################################################

\hfill \textbf{Author:} Kai Nagel

\begin{center} \includegraphics[width=0.25\textwidth, angle=0]{figures/MATSimBook.png} \end{center}

% ##################################################################################################################
Modules in the config: 
\begin{itemize}
	\item \lstinline|roadpricing|
\end{itemize}

Usage: Contrib via config \kai{not sure what that means}

Literature: \citet[][]{RieserEtAl_TechRep_VSP_2007, Rieser_unpub_IVT_2008, GretherEtAl_ERSA_2008, RieserEtAl_TRBTDF_2008} \who{Rieser, Chakirov, Grether, Kickhöfer, Nagel}

% ##################################################################################################################

\section{Introduction}

Roadpricing is an often-discussed policy measure \citep[e.g.][]{...}.  Its implementation in \acrshort{matsim} is conceptually straightforward: Essentially, for each vehicle entering a link at a given time, the appropriate toll is computed, and charged to the vehicle's driver.  The scoring function will pick this up by the term (see Equation~(\ref{eq:tdisutility})
\[
S_{trav,car,q} = ... + \beta_{c} \cdot (- \tau) ... \ ,
\]
where $\tau$ is the sum of all toll payments, and $\beta_{c}$ is the marginal utility of money (also see Chapter~\ref{ch:economicEval}).  The driver then takes this into account for decision-making, e.g.\ with respect to route choice, departure time choice, mode choice, location choice, etc., trading off toll payments with other elements of his or her function.

It should be clear that this picks automatically up all kinds of heterogeneities, for example:
\begin{compactitem}

\item Travelling at a different time may lead to a different toll, but possibly also to different schedule delay costs. \kai{wo wird das mit dem schedule delay erklärt?}

\item Different vehicle types may be charged different tolls \citep{KickhoeferNagel2012EmissionInternalization}.

\item Different travellers may have different values of time \citep{NagelKickhoeferJoubertHeterogeneousVoTs}, which may even vary by time-of-day.
  
\end{compactitem}

\section{Invocation}

\subsection{Minimal}

The least amount of infrastructure is necessary when running roadpricing from the command line.  For this, the \acrshort{matsim} jar, its libraries, \emph{and} the roadpricing jar need to be downloaded either from a release or from the nightly builds. \kai{explained where?}  After unzipping all zipfiles, the necessary command approximately is
\begin{lstlisting}
java -Xmx2000m -cp MATSim.jar:roadpricing-.../roadpricing-...jar org.matsim.roadpricing.run.Main config.xml  
\end{lstlisting}
where \lstinline$config.xml$ needs to contain a section
\begin{xml}
	<module name="roadpricing" >
		<param name="tollLinksFile" value="<path>/<tollfilename>" />
	</module>
\end{xml}
The toll file looks like this:
\begin{xml}
<roadpricing type="link" name="abc">
   <links>
      <link id="11">
         <cost start_time="05:00" end_time="10:00" amount="1." />
         <cost start_time="17:00" end_time="20:00" amount="1." />
      </link>             
      <link id="12" />
   </links>

   <!--this is for all links with no cost entry above:-->
   <cost start_time="05:00" end_time="10:00" amount="2.00"/>

</roadpricing>
\end{xml}
As one can see, there is a section where each link can be entered separately.  There can also be a separate cost structure for each link.  All links that are listed completely without a separate cost structure fall back on the general cost structure listed at the end.

\subsection{As ``script in Java''}

The above \lstinline$org.matsim.roadpricing.run.Main$ class can also be used as a starting point for one's own ``script in Java'':
\begin{lstlisting}
public static void main(String[] args) {
	// load the config, telling it to "materialize" the road pricing section:
	Config config = ConfigUtils.loadConfig( args[0], 
           new RoadPricingConfigGroup() ) ;
	
	// load the scenario:
	Scenario scenario = ScenarioUtils.loadScenario(config) ;

	// instantiate the controler:
	Controler controler = new Controler(scenario) ;

	// instantiate road pricing and add it as controler listener:
	RoadPricing roadPricing = new RoadPricing() ;
	controler.addControlerListener( roadPricing ) ;

	// run the controler:
	controler.run() ;
}
\end{lstlisting}




% Local Variables:
% mode: latex
% mode: reftex
% mode: visual-line
% TeX-master: "../../main"
% comment-padding: 1
% fill-column: 9999
% End: 
