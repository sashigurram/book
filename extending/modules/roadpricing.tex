\chapter{Roadpricing \who{Nagel?}}
\label{ch:roadpricing}
% ##################################################################################################################

\hfill \textbf{Author:} Kai Nagel

\begin{center} \includegraphics[width=0.25\textwidth, angle=0]{figures/MATSimBook.png} \end{center}

% ##################################################################################################################
Modules in the config: 
\begin{itemize}
	\item \lstinline|roadpricing|
\end{itemize}

Usage: Contrib via config \kai{not sure what that means}

Literature: \citet[][]{RieserEtAl_TechRep_VSP_2007, Rieser_unpub_IVT_2008, GretherEtAl_ERSA_2008, RieserEtAl_TRBTDF_2008} \who{Rieser, Chakirov, Grether, Kickhöfer, Nagel}

% ##################################################################################################################

\section{Introduction}

Roadpricing is an often-discussed policy measure.  

\section{Invocation}

\subsection{Minimal}

The least amount of infrastructure is necessary when running roadpricing from the command line.  For this, the \acrshort{matsim} jar, its libraries, \emph{and} the roadpricing jar need to be downloaded either from a release or from the nightly builds. \kai{explained where?}  After unzipping all zipfiles, the necessary command approximately is
\begin{lstlisting}
java -Xmx2000m -cp MATSim.jar:roadpricing-.../roadpricing-...jar org.matsim.roadpricing.run.Main config.xml  
\end{lstlisting}
where \verb$config.xml$ needs to contain a section
\begin{lstlisting}
	<module name="roadpricing" >
		<param name="tollLinksFile" value="<path>/<tollfilename>" />
	</module>
\end{lstlisting}
The toll file looks like this:
\begin{lstlisting}
<roadpricing type="link" name="abc">
   <links>
      <link id="11">
         <cost start_time="05:00" end_time="10:00" amount="1." />
         <cost start_time="17:00" end_time="20:00" amount="1." />
      </link>             
      <link id="12" />
   </links>

   <!--this is for all links with no cost entry above:-->
   <cost start_time="05:00" end_time="10:00" amount="2.00"/>

</roadpricing>
\end{lstlisting}
As one can see, there is a section where each link can be entered separately.  There can also be a separate cost structure for each link.  All links that are listed completely without a separate cost structure fall back on the general cost structure listed at the end.

\subsection{As ``script in Java''}

The above \verb$org.matsim.roadpricing.run.Main$ class can also be used as a starting point for one's own ``script in Java'':
\begin{lstlisting}
public static void main(String[] args) {
	// load the config, telling it to "materialize" the road pricing section:
	Config config = ConfigUtils.loadConfig( args[0], 
           new RoadPricingConfigGroup() ) ;
	
	// load the scenario:
	Scenario scenario = ScenarioUtils.loadScenario(config) ;

	// instantiate the controler:
	Controler controler = new Controler(scenario) ;

	// instantiate road pricing and add it as controler listener:
	RoadPricing roadPricing = new RoadPricing() ;
	controler.addControlerListener( roadPricing ) ;

	// run the controler:
	controler.run() ;
}
\end{lstlisting}




% Local Variables:
% mode: latex
% mode: reftex
% mode: visual-line
% TeX-master: "../../main"
% comment-padding: 1
% fill-column: 9999
% End: 
