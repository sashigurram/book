\chapter{MATSim For UrbanSim}
\label{ch:matsim4urbansim}
% ##################################################################################################################

\hfill \textbf{Author:} Kai Nagel

\begin{center} \includegraphics[width=0.25\textwidth, angle=0]{figures/MATSimBook.png} \end{center}

\section{Basic Information}
\label{sec:matsim4urbansim-stdInfo}

\createStandardInformationBasic{%
\url{http://matsim.org/javadoc} $\to$ matsim4urbansim
%
}{%
%
The module is invoked from a live UrbanSim implementation.
%
}{%
%
In the UrbanSim config file.
%
}{%
\citet[][]{Nicolai2011Couplinganurbansimulationmodelwithatravelmodel}; \citet{NicolaiNagelHiResAccessibilityMethod}; \citet{NicolaiNagel2013HandbookIntegrationOfTransportLanduse}
%
}

%%%%%%%%%%%%%%%%%%%%%%%%%%%%%%%%%%%%%%%%%%%%
%%%%%%%%%%%%%%%%%%%%%%%%%%%%%%%%%%%%%%%%%%%%
\section{Summary}


This is an adapter package in order to use MATSim as a travel model plug-in to \acrshort{urbansim} \citep[e.g.][see \url{http://www.urbansim.org}]{WaddellEtc2003UrbanSim}.
%
\acrshort{urbansim} is a well-known land use simulation.  It has, for example, submodels for residential location choice, commercial location choice, or development and building construction, and thus creates synthetic scenarios of possible urban or regional developments under various conditions and constraints.  As is well known, the traffic infrastructure plays a role in such developments; for example, highly accessible areas are more attractive both as residences as well as for commercial activities.  Since accessibility is reduced by congestion, and congestion can only be modelled realistically by having a sophisticated model of the interaction of demand and supply, \acrshort{urbansim} does not have a travel model of its own, but delegates that task to external models, such as MATSim.

In order to use MATSim4UrbanSim, one first needs to have a running \acrshort{urbansim} installation.  From there, one can add MATSim to that installation; see Section \ref{sec:matsim4urbansim-stdInfo} to find more information.  The basic MATSim parameters are configured from the \acrshort{urbansim} configuration file by adding an appropriate section; again, see Section \ref{sec:matsim4urbansim-stdInfo} to find more information.  It is possible to add a standard MATSim config file, thus allowing to use extended MATSim features, including those that were added after the adapter package was designed.

The module has not been maintained since 2013 and it is unclear how much of it currently works.

%%%%%%%%%%%%%%%%%%%%%%%%%%%%%%%%%%%%%%%%%%%%
%%%%%%%%%%%%%%%%%%%%%%%%%%%%%%%%%%%%%%%%%%%%
% Local Variables:
% mode: latex
% mode: reftex
% mode: visual-line
% TeX-master: "../../main"
% comment-padding: 1
% fill-column: 9999
% End: 




