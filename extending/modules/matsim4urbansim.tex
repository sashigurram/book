\chapter{MATSim For UrbanSim}
\label{ch:matsim4urbansim}
% ##################################################################################################################

\hfill \textbf{Author:} Kai Nagel

\begin{center} \includegraphics[width=0.25\textwidth, angle=0]{frontmatter/figures/MATSimBook} \end{center}

% ##################################################################################################################
\section{Basic Information}
\label{sec:matsim4urbansim-stdInfo}

\ah{Ich hätte das viel lieber in discontinued models, entspricht ja auch dem Stand der Dinge. Man könnte da dann dranschreiben, dass das Modul grundsätzlich reaktiviert werden kann und evtl. wird.}

\createStandardInformationBasic{%
\url{http://matsim.org/javadoc} $\to$ matsim4urbansim
%
}{%
%
The module is invoked from a live UrbanSim implementation.
%
}{%
%
In the UrbanSim config file.
%
}{%
\citet[][]{Nicolai2011Couplinganurbansimulationmodelwithatravelmodel}; \citet{NicolaiNagel2012HiResAccessibilityMethodInBook}; \citet{NicolaiNagel2013HandbookIntegrationOfTransportLanduse}
%
}

% ##################################################################################################################
\section{Summary}
This is an adapter package in order to use \gls{matsim} as a travel model plug-in to \gls{urbansim}, a well-known land use simulation \citep[e.g.][see \url{http://www.urbansim.org}]{WaddellEtc2003UrbanSim}.
\gls{urbansim} has, for example, submodels for residential location choice, commercial location choice, or development and building construction, and thus creates synthetic scenarios of possible urban or regional developments under various conditions and constraints. 
As is well known, the traffic infrastructure plays a role in such developments; for example, highly accessible areas are more attractive both as residences as well as for commercial activities. 
Since accessibility is reduced by congestion, and congestion can only be modeled realistically by having a sophisticated model of the interaction of demand and supply, \gls{urbansim} does not have a travel model of its own, but delegates that task to external models, such as \gls{matsim}.

In order to use MATSim4UrbanSim, one first needs to have a running \gls{urbansim} installation. 
From there, one can add \gls{matsim} to that installation; see Section~\ref{sec:matsim4urbansim-stdInfo} to find more information. 
The basic \gls{matsim} parameters are configured from the \gls{urbansim} configuration file by adding an appropriate section; again, see Section~\ref{sec:matsim4urbansim-stdInfo} to find more information. 
It is possible to add a standard \gls{matsim} \gls{configfile}, thus allowing to use extended \gls{matsim} features, including those that were added after the adapter package was designed.

The module has been applied by \citet[][]{ZoelligRenner_PhDThesis_2014}. 
However, basically, the module has not been maintained since 2013 and it is unclear how much of it currently works.

% ##################################################################################################################
% Local Variables:
% mode: latex
% mode: reftex
% mode: visual-line
% TeX-master: "../../main"
% comment-padding: 1
% fill-column: 9999
% End: 




