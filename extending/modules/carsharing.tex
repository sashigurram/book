\chapter{Carsharing}
\label{ch:carsharing}
% ##################################################################################################################

\hfill \textbf{Authors:} Francesco Ciari, Milos Balać

\begin{center} \includegraphics[width=0.25\textwidth, angle=0]{figures/MATSimBook.png} \end{center}

\editdone{This text has undergone the professional edit. Please no grammatical changes anymore! They are most-probably wrong.}

\createStandardInformation{todo}{todo}{todo}{\citet[][]{Ciari_PhDThesis_2012, CiariEtAl_Transportation_2014}}

% ##################################################################################################################
\section{Background}
The basic idea of carsharing is simple; a fleet of cars can be shared by several users, who can rent a car when they need it, eliminating the necessity of ownership. The possibility to rent short-term is the main difference from traditional car rental. This basic concept can be implemented in various ways;  in the last few years, several new business models have evolved in the market. There are three main operational models:
%
\begin{compactitem}
	\item Round-trip based: cars are parked at dedicated stations. They can be picked up from a station and left at the same station after use.
	\item One-way: cars are parked at dedicated stations. They must be picked up from a station and can be left at any station after use.
	\item Free-floating: cars are parked in any parking slot within a defined service area. They can be picked up and left anywhere within this area after use. 
\end{compactitem}
%
From a transport planning perspective, the very nature of carsharing---the importance of its availability at precise points in time and space---does not fit with traditional models, which are based on vehicles-per-hour flows. It is crucial to represent the availability of vehicles at the local level,  therefore representing individual travel with high spatial and temporal resolution. At the same time, when analyzing the carsharing choice, it is extremely important how a trip/activity is embedded in the whole activity chain. This combination of features can be found in MATSim, making it t a suitable framework for carsharing modeling. 

% ##################################################################################################################
\section{Modeling of Carsharing Demand in MATSim}
Carsharing, as modal option in MATSim, was originally introduced in a simplified manner and only in its round trip-based version as part of a dissertation work \citep[][]{Ciari_PhDThesis_2012}. Several improvements have been introduced since then and currently all three main types of carsharing can be simulated.

% ===================================================================================
%\subsection{Carsharing Travel} 
% -------------------------------------------------------------------
%\subsubsection{Round-Trip Based Carsharing}
\subsection{Round-Trip Based Carsharing}
The use of round-trip carsharing by an agent in the simulation is modeled in the following steps: 
\begin{enumerate}
	\item agent finishes his activity, finds the closest available car and reserves it (making it unavailable for other agents),
	\item agent walks to the station where he has reserved a vehicle,
	\item agent drives the car (interaction with other vehicles is modeled),
	\item parks the car at the next activity,
	\item after finishing his activity, the agent takes the car and drives to the next activity...
	\item before reaching the last activity in the subtour, agent ends the rental and leaves the vehicle at the starting station making it available to other agents,
	\item walks to the activity,
	\item carries out the rest of the daily plan.
\end{enumerate}

% -------------------------------------------------------------------
%\subsubsection{One-Way Carsharing}
\subsection{One-Way Carsharing}
For one-way carsharing, the steps are similar, but with several significant differences:
\begin{enumerate}
	\item agent finishes his activity, finds the closest station with an available car and reserves the vehicle (making it unavailable for other agents),
	\item agent walks to the station where it has reserved the car (takes the car and frees a parking spot at the station),
	\item agent finds the closest station to his destination, with a free parking spot and reserves it (making it unavailable for others)
	\item agent drives the car to the reserved parking spot (interacting with other vehicles in the network),
	\item parks the car on the reserved parking spot and ends the rental,
	\item walks to the next activity
	\item carries out the rest of the daily plan.
\end{enumerate}

% -------------------------------------------------------------------
%\subsubsection{Free-Floating Carsharing}
\subsection{Free-Floating Carsharing}
 Free-floating carsharing by an agent is simulated using similar steps, but the rental ends with the end of one trip:
\begin{enumerate}
	\item rent the nearest car.
	\item	walk from start activity to the rented car.
	\item	drive to the next activity (interaction with other vehicles are modeled).
	\item	park the car close to the next activity.
	\item	end the rental (and make the car available for other rentals).
\end{enumerate}

% -------------------------------------------------------------------
%\subsubsection{Generalized Cost of Carsharing Travel}
\subsection{Generalized Cost of Carsharing Travel}
The function representing generalized cost of travel for car sharing traveling from activity $q-1$ to activity $q$ is:
%
\begin{equation}
S_{trav, q, cs} = \alpha_{cs} + \beta_{c,cs} \cdot c_{t} \cdot t_r + \beta_{c,cs} \cdot c_d \cdot d + \beta_{t, walk} \cdot (t_a + t_e) + \beta_{t,cs} \cdot  t
\end{equation}
%
The same equation is used for all modeled forms of carsharing, but the parameter values will be different. 
%
The first term $\alpha_{cs}$ is a constant which can be used as calibration parameter and will also usually be different for different types of carsharing (and for different contexts).
%
The second and third terms refer to the time-dependent and the distance-dependent part of the fee, respectively. 
%
$t_r$ is the total reservation time and $c_t$ represents the monetary cost for one hour reservation time.
%
$d$ is the total reservation distance and $c_d$ is the marginal monetary cost for one kilometer of travel. 
%
The parameter $\beta_{c,cs}$ represents the marginal utility of an additional unit of money spent on traveling with carsharing. 
%
The fourth term is the walking path to and from the station (access time $t_a$ and egress time $t_e$) and is evaluated as a normal walk leg. 
%
The parameter $\beta_{t,cs}$ represents the marginal utility of an additional unit of time spent on traveling with carsharing, where $t$ is the actual (in vehicle) travel time. 

% ===================================================================================
\section{Carsharing Membership}
Carsharing is a membership program.  To be able to access a specific carsharing service, individuals must become member of that carsharing program. A logit model has been estimated for Switzerland \citep[][]{CiariWeis_EURO_2014} and implemented in MATSim as part of the carsharing module. The variables of the models are mainly individual socio-demographic characteristics.  
An important feature of the model, however, is that carsharing accessibility is explicitly considered, both from home and from work. Thus, accessibility A of person p is calculated with the following formula:
%
\begin{equation}
A(n) = \mathrm{ln} \left(\sum_{s=1}^m X_s \cdot \mathrm{e}^{-\beta \cdot d_{sh}}\right) + \mathrm{ln} \left(\sum_{s=1}^{m} X_s \cdot \mathrm{e}^{-\beta \cdot d_{sw}}\right)
\end{equation}
%
The weight parameter for distances is set to 0.2 as in \citet[][]{Weis_PhDThesis_2012} \ah{ref correct?}, and more details on it are given below. Assuming $m$ as the number of stations in the system, $d_{sh}$ and $d_{sw}$, are calculated for each station as the distance between the station $s$ and the home and work location of person $n$ respectively; and $X_s$ is the number of cars at station $s$. 
The model is not directly transferable to other regions, but a different model can be easily implemented in the Java code created. 

% ##################################################################################################################
\section{Validation}
The simulation model has been calibrated in order to reproduce the actual modal share for carsharing in the Zürich region of Switzerland. It was made using booking data from the Swiss operator Mobility. With the same data, the results were validated along several dimensions. Since Mobility  offers only round-trip based carsharing, only  this model could be validated. Dimensions included in the validation process were: distance from the last activity to the pick-up station, departure times, purpose of the rental (main purpose of the subtour) and temporal length of the rental. 

% ##################################################################################################################
\section{Applications}
After a long phase of creating and improving the module to simulate carsharing in all its forms, work has been recently carried out addressing specific carsharing operations issues. Examples are: evaluating the impact of a free-floating carsharing program introduction in Berlin on travel demand \citep[][]{CiariEtAl_TRR_2014} and Zürich \citep[][]{CiariEtAl_Transportation_2014} and investigating the relationship between demand and supply in both round-trip and one-way systems \citep[][]{BalacEtAl_TRB_2015}.

% ##################################################################################################################
%\ah{\citet[][]{CiariEtAl_TechRep_IVT_2013, CiariEtAl_TechRep_IVT_2014, CiariEtAl_TRB_2009}}



