% ##################################################################################################################
\chapter{Tampa, Florida: High Resolution Simulation of Urban Travel and Network Performance for Estimating Mobile Source Emissions}
\label{ch:tampa}
\hfill \textbf{Authors:} Sashikanth Gurram, Abdul R. Pinjari, and Amy L. Stuart

% ##################################################################################################################
\section{Introduction}
Mobile sources are significant contributors to ambient traffic-related air pollution associated with adverse health impacts in urban areas. Thus, it is important to accurately characterize mobile source emissions and population exposure to those emissions; this requires a high-resolution simulation of urban travel. In this study, using activity-based travel demand modeling and \gls{matsim}-based dynamic traffic assignment modeling, we demonstrate a large-scale, high-resolution simulation of resident population travel activity and highway network performance in Tampa, Florida. Such high resolution simulation outcomes are useful in estimating mobile source emissions and human exposure to those emissions. 

% ##################################################################################################################
\section{Study Area}
Hillsborough County, a large section of the Tampa Bay region in Florida, is our study area. The county's geographic context is presented in Figure~\ref{fig:tampa-fig1}. The freeway road I-275, acts as a major commuter corridor connecting the area north of Tampa to the central business district to the south. The freeway roads I-75 and I-4 run north-south and east-west, respectively, and serve as major highways for intra-city, inter-city and inter-state travel. The county has a diverse mix of air pollution sources and population demographics, few public transportation options, an unsatisfactory air quality record and a sprawling urban form. These make it an interesting test case from an air pollution perspective.

 %------------
\createfigure%
{The study area of Hillsborough County, Florida}%
{The study area of Hillsborough County, Florida}%
{\label{fig:tampa-fig1}}%
{\includegraphics[width=0.7\textwidth, angle=0]{./scenarios/figures/tampa-fig1.jpg}}%
{\citet[][]{GurramEtAl_AQAH_2015}}
 %------------

% ##################################################################################################################
\section{Modeling Framework}
Figure~\ref{fig:tampa-fig2} depicts the modeling framework used to simulate urban population activity and transportation network performance for the study region. 
An activity-based model (ABM) of travel demand (DaySim) is coupled with \gls{matsim}, here applied as a dynamic traffic assignment (DTA) model. 
The DaySim framework was originally developed for the Sacramento region \citep[][]{BradleyEtAl_JOCM_2010} and calibrated for the Tampa Bay region, using local household travel data \citep[][]{GliebeEtAl_TRB_2014}. 
An appealing feature of DaySim is its use of fine, parcel-level representation of space, which leads to high spatial resolution in the simulated activity locations.  
Similar to other \glspl{abm}, inputs to DaySim include detailed population demographics, land-use patterns and transportation system characteristics in the study region. The demographic inputs come from a population synthesizer called \lstinline|PopGen| \citep[][]{PendyalaEtAl_2011} that generates a synthetic population of individuals and households to match aggregate-level distributions of both household- and person-level characteristics from the U.S. Census. 
Demographic variables not controlled in PopGen (\eg household car ownership and individual employment characteristics) were estimated using econometric models based on local data. 
Taking all the above as inputs, DaySim simulates the daily activity and travel patterns of all residents in the study region, including the timing, duration and location of activities and the mode of travel between different activity locations. 
We ran the model on an eight-core Windows machine with a 2.8\,GHz Intel Xeon processor and 24\,\gls{gb} \gls{ram}. 
The run time was approximately 5\,hours for the entire Tampa Bay population of about 3\,million individuals.

 %------------
\createfigure%
{The transportation modeling framework}%
{The transportation modeling framework}%
{\label{fig:tampa-fig2}}%
{\includegraphics[width=0.7\textwidth, angle=0]{./scenarios/figures/tampa-fig2.jpg}}%
{}
 %------------

DaySim does not simulate travel route information between different activity locations. 
However, information on travel routes and network performance (\ie link speeds and volumes) is essential for estimating emissions and human exposure to those emissions. 
Therefore, \gls{matsim} was used to simulate travel routes and network performance %\citep[][]{BalmerEtAl_TRBTDF_2008}. 
\citep[][]{BalmerEtAl2008ITMStructurePerformance}.
To do so, outputs from the Tampa \gls{abm} were processed using SPSS and \gls{java} programming to provide the initial set of plans for \gls{matsim}. 
Similarly, the ArcGIS road network file for the Tampa Bay area was processed to create network inputs for \gls{matsim}. 
Since most travel in Tampa is by automobile (with close to 90\,\% mode share), only these trips were simulated in \gls{matsim}. 
It is worth noting, however, that a large number of automobile trips were simulated. 
Specifically, 9.7\,million trips made by approximately 2.3\,million residents of the study region during a 24\,hour period were simulated. 
The simulation was run for 300\,iterations, with the storage capacity factor for the links set to 3. 
Additionally, maximum plan memory size for each agent was set to 3. The \lstinline|BestScore| and \lstinline|ReRoute| replanning modules were used with a probability of 0.9 and 0.1, respectively. 
To undertake this large-scale and computationally intensive simulation, 48\,parallel processors each with 25\,\gls{gb} of \gls{ram} from a university research computing cluster setup were utilized, requiring 5.2\,days total run time. 
Link-level outputs from the simulation, including hourly traffic volumes and travel times, were written to a \lstinline|linkstats| file;  trip-level route information was written to a \lstinline|plans| file.

% ##################################################################################################################
\section{Results}
Diurnal patterns of link-level passenger car volumes and travel speeds for Hillsborough County are presented in Figure~\ref{fig:tampa-fig3} (in the form of bi-hourly averages). As expected, traffic volumes, shown in Figure~\ref{fig:tampa-fig3} a), are higher during the morning (7 to 9\,am) and the evening (4 to 7\,pm) peak hours than the rest of the day. Additionally, traffic volumes during evening peak hours are higher than volumes during morning peak hours, perhaps partly because the evening commute has a higher propensity for trip chaining compared to the morning commute \citep[][]{ChuYL_TRR_2003}. Travel speeds shown in Figure~\ref{fig:tampa-fig3} b) correspond to the diurnal pattern of traffic volumes, with lower speeds during the morning and evening peak hours.

Spatially, higher volumes are observed along major freeway corridors---I-75, I-275, and I-4, as expected. High traffic volumes are also observed along the road network near suburban locations, including Brandon, Citrus Park and Town 'N' Country. Accordingly, travel speeds are lower in these suburban locations along with the North Tampa area, University area and a few sections of the freeway corridors. 

The root mean squared error between the estimated traffic volumes and observed traffic volumes at eight different traffic counting stations is 0.41. Further, the error between estimated and observed traffic flows for inter-city roads was higher than those for intra-city roads, presumably because the current model system does not consider long-distance (or inter-city) travel, visitors' travel and freight movement in detail. Nevertheless, the high temporal and spatial resolution of the population activity (including individuals' travel routes) and network performance (\ie link volumes and speeds) simulated using the model system is promising for future detailed estimation of traffic pollutant emissions and human exposures to those emissions.  

 %------------
\createfigure%
{Simulated bi-hourly varying a) passenger car volumes and b) travel speeds (mph) for Hillsborough County on a typical weekday}%
{Simulated bi-hourly varying a) passenger car volumes and b) travel speeds (mph) for Hillsborough County on a typical weekday}%
{\label{fig:tampa-fig3}}%
{\includegraphics[width=1.5\textwidth, angle=90]{./scenarios/figures/tampa-fig3.jpg}}%
{}
 %------------

% ##################################################################################################################
\section{Future Work}
The next steps of this study include addition of inter-city, visitor and freight travel to the model system. 
Utilizing the fine-resolution, link-level traffic volume and speed outputs from \gls{matsim}, EPA’s MOVES software is being used to estimate mobile source emissions. 
Mobile source emissions can be combined with other sources of emission and meteorological data, using a pollutant dispersion model, to estimate diurnal cycles of hourly varying pollutant concentrations. 
The resulting pollutant concentrations will be combined with the diurnal locations of individuals (obtained from the \gls{abm} and \gls{matsim}) to estimate individual-level exposure to traffic-related pollutants, such as nitrogen oxides. 
Such individual-level exposure measures will be utilized to estimate demographic group-level exposures for assessment of inequality in exposure to traffic-related air pollution, as we have done previously using travel survey data \citep[][]{GurramEtAl_AQAH_2015}. 
The model system described above will be used to obtain estimates of population exposure for alternative scenarios of urban land-use design and transport policies.

% ##################################################################################################################
\section{Conclusion}
In this study, we simulated urban travel using activity-based travel demand modeling and dynamic traffic assignment, to obtain network performance measures, including link-level traffic volumes and speeds, at a high spatial and temporal resolution for Hillsborough County in Florida. 
As expected, simulated traffic volumes are higher and travel speeds lower during morning and evening peak hours. 
Spatially, higher volumes and lower speeds are observed along the freeway corridors and suburban locations than other locations. 
Model performance (vis-à-vis observed traffic patterns) is better for inter-city roads than intra-city roads, highlighting the need for better modeling of long distance passenger travel and freight movement. 
When the \gls{abm}-\gls{dta} framework built in this study is expanded to consider mobile source emissions and pollutant dispersion, the resulting transportation and air pollution modeling system will be useful for understanding interactions between urban transportation design, air pollution and population exposure to pollution.

% ##################################################################################################################

% Local Variables:
% mode: latex
% mode: reftex
% mode: visual-line
% TeX-master: "../main"
% comment-padding: 1
% fill-column: 9999
% End: 
